%%%%%%%%%%%%%%%%%%%%%%%%%%%%%%%%%%%%%%%%%%%%%%%%%%%%%%%%%%%%%%%%%%%%%%%%%
%%
%W  pt.tex             ACE standalone documentation          Colin Ramsay
%W                                                            Greg Gamble
%%
%H  $Id$
%%

%%  Colin Ramsay - 21 Sep 1999  is the true author of this document
%%  Greg Gamble  - 28-29 Feb 2000 made minor modifications:
%%                * modifications detailed in CHANGES
%%                * put this header at the top for CVS

%%  The Proof-table chapter.
%%
%%  5   10   15   20   25   30   35   40   45   50   55   60   65   70   75
%%..|....|....|....|....|....|....|....|....|....|....|....|....|....|....|

TBA:
Put the relation proving et al.\@ (incl DI) as a separate, \#ifdef'd,
  add-on package to \ace\ - with its own chapter in the manual?
We can do other things with this data structure; i.e., find a presentation
  for the subgroup in terms of the given subgroup generators.

\section{Example}

The Fibonacci group 
 $$ F(2,5) = \langle a,b,c,d,e \mid abC, bcD, cdE, deA, eaB \rangle $$
  is known to be the cyclic group of order $11$, $C_{11}$ \cite{Can}.
A proof of this can be readily extracted from an index one coset 
  enumeration of $[F(2,5) : \langle a \rangle]$, as discussed in \cite{Lee2}.
We use this as an example to illustrate, in detail, the basic technique for
  producing proofs from the workings of an enumeration.
%
For an enumeration of $[F(2,5) : \langle a \rangle]$, it is obvious that a
  total of more than one coset must be defined, and it is straightforward
  to establish that two are not sufficient, since there are only eight
  possible choices to check for the definition of coset \#$2$.
So the three-coset enumeration given is minimal.

For definiteness, we assume that the relators are processed in the order
  given, relators are processed from left to right until a gap or their
  end and then processed from right to left, deductions are made from left
  to right when a relator closes, deductions are stored in a stack (we 
  assume that both a deduction and its inverse are stored, with the 
  deduction at the top to make explicit the processing order), new
  coincidences are stored in a queue after performing a union-find, and
  that table rows are processed in column order (i.e., 
  $a$, $A$, \dots, $e$, $E$).

In what follows, we trace through the entire sequence of operations, and
  explicitly list the proof word associated with each (non-definition)
  table entry.
The relator, possibly cycled and/or inverted, is given in square brackets
  at the \tit{end} of the proof word.
Cosets appear as subscripts.
Subgroup generators are given in double square brackets.
For example, if $\alpha,\beta,\gamma$ are cosets at some stage of an
  enumeration and it is known that $\alpha a = \beta$ \amp 
  $\gamma C = \alpha$, then the relator $abC$ yields the deduction 
  $\beta b = \gamma$, which has proof word 
  $_{\beta}A_{\alpha}c_{\gamma}[Cab]_{\gamma}$;
  the meaning of this proof word is:
  $\beta A=\alpha, \alpha C=\gamma, \gamma Cab = \gamma$.
Coincidences in proof words will be indicated using `$0$' (e.g., 
  $_\alpha 0_\beta$), and we adopt the convention of writing the 
  higher-numbered coset (i.e., the one to be `replaced') on the left when
  $\alpha = \beta$.

\tsc{Step 1}:
Processing the subgroup generator yields the deduction $1a=1$ and its
  inverse $1A=1$.
We associate the words $_1[[a]]_1$ \amp $_1[[A]]_1$, respectively, with
  these entries.
Testing these deductions yields nothing, and leaves both the stack and the
  queue empty.

\tsc{Step 2}:
Make the definition $1E=2$ and its inverse $2e=1$, and push $2e$ and then
  $1E$ onto the stack.
Popping and processing $1E$ yields nothing.
Popping and processing $2e$ yields, successively, the deductions and
  proof words $1d=2$ ($_1a_1E_2[eAd]_2$), $2D=1$ ($_2e_1A_1[aED]_1$),
  $1B=2$ ($_1A_1E_2[eaB]_2$) \amp $2b=1$ ($_2e_1a_1[AEb]_1$).
The deductions are pushed in the order $2D$, $1d$, $2b$ \amp $1B$.
Testing these deductions yields nothing, and leaves both the stack and the
  queue empty.

\tsc{Step 3}:
Make the definition $1c=3$ and its inverse $3C=1$, and push $3C$ and then
  $1c$ onto the stack.
Processing $1c$ yields $3D=2$ ($_3C_1B_2[bcD]_2$) \amp $2d=3$
  ($_2b_1c_3[CBd]_3$), and stacks $2d$ then $3D$.
Processing $3D$ and then $2d$ yields nothing.
Processing $3C$ yields $1b=3$ ($_1A_1c_3[Cab]_3$) \amp $3B=1$
  ($_3C_1a_1[AcB]_1$), and stacks $3B$ and then $1b$.
Processing $1b$ yields $3c=2$ ($_3B_1d_2[Dbc]_2$) \amp $2C=3$
  ($_2D_1b_3[BdC]_3$), and stacks $2C$ \amp $3c$.
Processing $3c$ yields $3E=3$ ($_3D_2C_3[cdE]_3$) \amp $3e=3$
  ($_3c_2d_3[DCe]_3$), and stacks $3e$ \amp $3E$.
Processing $3E$ yields nothing.
Processing $3e$ yields $3A=2$ ($_3E_3D_2[deA]_2$) \amp $2a=3$
  ($_2d_3e_3[EDa]_3$), and stacks $2a$ \amp $3A$.
Processing $3A$ yields nothing.
Processing $2a$ yields $1e=2$ ($_1b_3A_2[aBe]_2$) \amp $2E=1$
  ($_2a_3B_1[bAE]_1$), and stacks $2E$ \amp $1e$.
Processing $1e$ yields nothing.
Processing $2E$ yields that $3=1$, with proof word $_3C_1e_2D_1[dEc]_1$.

\begin{table}
\hrule
\caption{Coset table at beginning of step $4$}
\label{tab:step4}
\smallskip
\renewcommand{\arraystretch}{0.875}
\begin{tabular*}{\textwidth}{@{\extracolsep{\fill}}lrrrrrrrrrr} 
\hline\hline
coset & $a$ & $A$ & $b$ & $B$ & $c$ & $C$ & $d$ & $D$ & $e$ & $E$ \\ 
\hline
 1 & 1 & 1 & 3 & 2 & 3 & - & 2 & - & 2 & 2 \\
 2 & 3 & - & 1 & - & - & 3 & 3 & 1 & 1 & 1 \\
 3 & - & 2 & - & 1 & 2 & 1 & - & 2 & 3 & 3 \\
\hline\hline
\end{tabular*}
\end{table}

\tsc{Step 4}:
At this point the stack contains, from bottom to top, $3B$ \amp $2C$, and
  we are currently processing $2E(=1)$.
The current state of the table is given in Table~\ref{tab:step4}.
We suspend deduction processing, and switch over to coincidence processing.
We remove $3=1$ from the queue, and process the information (and its
  inverse) in row \#$3$ of the table, `moving' it to row \#$1$.
Because \ace\ uses the first two columns to store the coincidence
  information, the order in which we process the consequences of a primary
  coincidence is a little counter-intuitive; so we'll list the actions
  in detail.

Process columns 1/2 of $3=1$:
Column $a$ yields nothing.
For column $A$, we first clear its inverse entry at $2a$ (it will be set,
  if necessary, during later coincidence-consequence processing).
We now find the coincidence $2=1$ (the proof word, $_2 a_3 0_1$, uses
  the entry for $2a$ which we've just cleared, but we'll assume that we
  still know what it was, since it hasn't actually been `overwritten'
  as yet!), which is `remembered'\kern-1.5pt, but not yet queued.
The inverse entry for $1A$ (which is in $1a$) is already set, so no
  further action is required.

Process columns 1/2 of $2=1$:
We now queue the coincidence $2=1$, and process the information in the
  first two columns of row \#$2$ of the table.
In fact, since the $2a$ entry has been cleared, and the $2A$ entry was
  already empty, nothing has to be done.

Process remaining columns of $3=1$:
Column $b$ yields nothing.
%
For column $B$, the inverse entry for $3B$ (i.e., $1b$) is cleared, and the
  coincidence $2=1$ is rediscovered (but does nothing).
The inverse entry for $1B$ (which is in $2b$) is already set, so no
  further action is required.
%
For column $c$, the inverse entry for $3c$ (i.e., $2C$) is cleared.
The fact that $1c=3$, and $3=1$, causes the coincidence $2=1$ to be
  rediscovered, but this does nothing.
The inverse entry for $1c$ (which is in $3C$) is already set, so no
  further action is required.
%
For column $C$, the inverse entry for $3C$ (i.e., $1c$) is cleared.
Since the $1C$ entry is blank, we deduce that $1C=1$ ($_1 0_3 C_1$) \amp
  $1c=1$ ($_1 c_3 0_1$), and stack $1c$ then $1C$.
%
Column $d$ yields nothing.
%
For column $D$, the inverse entry for $3D$ (i.e., $2d$) is cleared.
Since the $1D$ entry is blank, we deduce that $1D=2$ ($_1 0_3 D_2$) \amp
  $2d=1$ ($_2 d_3 0_1$), and stack $2d$ then $1D$.
%
For column $e$, the inverse entry for $3e$ (i.e., $3E$) is not cleared,
  since it equals $3$ (which is actually $1$!).
We now find that $2=1$, but this yields nothing.
The inverse entry for $1e$ (which is in $2E$) is already set, so no
  further action is required.
%
The processing for column $E$ is a `mirror-image' to that for column $e$,
  and so yields nothing.

Process remaining columns of $2=1$:
At this point the coset table is as shown in Table~\ref{tab:step4f}.
Coset \#$3$ is dead and has been fully processed; i.e., no row of the table
  (other than \#$3$) contains a $3$.
Coset \#$2$ is coincident with coset \#$1$, and has been fully processed
  in columns $1$ \amp $2$ only (so there are no $2$'s in columns $1$ \amp
  $2$ of row \#$1$).
%
For column $b$, the inverse entry for $2b$ (i.e., $1B$) is cleared.
Since the $1b$ entry is blank, we deduce that $1b=1$ ($_1 0_2 b_1$) \amp
  $1B=1$ ($_1 B_2 0_1$), and stack $1B$ then $1b$.
%
Columns $B$, $c$ \amp $C$ yields nothing.
%
For column $d$, the inverse entry for $2d$ (i.e., $1D$) is cleared.
Since $2$ is actually $1$, we discover the vacuous coincidence $1=1$.
The inverse entry for $1d$ (which is in $2D$) is already set, so no
  further action is required.
%
For column $D$, the inverse entry for $2D$ (i.e., $1d$) is cleared.
Since the $1D$ entry is blank, we deduce that $1D=1$ ($_1 0_2 D_1$) \amp
  $1d=1$ ($_1 d_2 0_1$), and stack $1d$ then $1D$.
%
For column $e$, the inverse entry for $2e$ (i.e., $1E$) is cleared.
Since $2$ is actually $1$, we discover the vacuous coincidence $1=1$.
The inverse entry for $1e$ (which is in $2E$) is already set, so no
  further action is required.
%
For column $E$, the inverse entry for $2E$ (i.e., $1e$) is cleared.
Since the $1E$ entry is blank, we deduce that $1E=1$ ($_1 0_2 E_1$) \amp
  $1e=1$ ($_1 e_2 0_1$), and stack $1e$ then $1E$.

\tsc{Step 5}:
At this point we resume the processing of the deduction $2E$.
However, the only coset remaining active is \#$1$, and all of its entries
  are $1$.
So all remaining deductions are either discarded (since they involve dead
  cosets) or yield nothing (since all relator scans complete), and the
  enumeration is complete.
At this point, the coset table is the same as shown in
  Table~\ref{tab:step4f}, except that all the entries in the first row
  are $1$.

\begin{table}
\hrule
\caption{Coset table at final stage of step $4$}
\label{tab:step4f}
\smallskip
\renewcommand{\arraystretch}{0.875}
\begin{tabular*}{\textwidth}{@{\extracolsep{\fill}}lrrrrrrrrrr} 
\hline\hline
coset & $a$ & $A$ & $b$ & $B$ & $c$ & $C$ & $d$ & $D$ & $e$ & $E$ \\ 
\hline
 1 & 1 & 1 & - & 2 & 1 & 1 & 2 & 2 & 2 & 2 \\
 2 & - & - & 1 & - & - & - & 1 & 1 & 1 & 1 \\
 3 & - & 2 & - & 1 & 2 & 1 & - & 2 & 3 & 3 \\
\hline\hline
\end{tabular*}
\end{table}

\tsc{Notes:}
Of course, the tables shown in Tables~\ref{tab:step4} \amp
  \ref{tab:step4f} are actually split between the `original' \ace\ table
  and the auxiliary data structures maintained by the PT package.
%
Note that multiple coincidences can cause more complicated proof
  words, which this example does not illustrate.
%
Two further examples, complete with their working, can be found in 
  \cite{Lee4}.
Note however, that the proof words there (called `substitution words') do
  not explicitly include the relevant relator.

The proof words currently associated with the entries in the coset table
  are shown in Table~\ref{tab:prf}.
Recall that `$0$' denotes a coincidence, and note that words without a
  $0$ or a $[\cdot]$ term denote definitions.
In practice, proof words are never actually deleted or overwritten, since
  we may need them to extract a proof.
Instead, each table entry has associated with it a `stack' of proof words,
  and the appropriate one must be selected when expanding a current or
  previous table entry.

\begin{table}
\hrule
\caption{Final proof words}
\label{tab:prf}
\smallskip
\renewcommand{\arraystretch}{0.900}
\begin{tabular*}{\textwidth}{@{\extracolsep{\fill}}rlll} 
\hline\hline
 & \multicolumn{3}{c}{coset} \\
\cline{2-4}
generator & 1           & 2                 & 3 \\ 
\hline
$a$ & $_1[[a]]_1$       & $_2d_3e_3[EDa]_3$ & - \\
%
$A$ & $_1[[A]]_1$       & -                 & $_3E_3D_2[deA]_2$ \\
%
$b$ & $_1A_1c_3[Cab]_3$ & $_2e_1a_1[AEb]_1$ & - \\
    & $_1 0_2 b_1$      & -                 & - \\
%
$B$ & $_1A_1E_2[eaB]_2$ & -                 & $_3C_1a_1[AcB]_1$ \\
    & $_1 B_2 0_1$      & -                 & - \\
%
$c$ & $_1c_3$           & -                 & $_3B_1d_2[Dbc]_2$ \\
    & $_1 c_3 0_1$      & -                 & - \\
%
$C$ & $_1 0_3 C_1$      & $_2D_1b_3[BdC]_3$ & $_3C_1$ \\
%
$d$ & $_1a_1E_2[eAd]_2$ & $_2b_1c_3[CBd]_3$ & - \\
    & $_1 d_2 0_1$      & $_2 d_3 0_1$      & - \\
%
$D$ & $_1 0_3 D_2$      & $_2e_1A_1[aED]_1$ & $_3C_1B_2[bcD]_2$ \\
    & $_1 0_2 D_1$      & -                 & - \\
%
$e$ & $_1b_3A_2[aBe]_2$ & $_2e_1$           & $_3c_2d_3[DCe]_3$ \\
    & $_1 e_2 0_1$      & -                 & - \\
%
$E$ & $_1E_2$           & $_2a_3B_1[bAE]_1$ & $_3D_2C_3[cdE]_3$ \\
    & $_1 0_2 E_1$      & -                 & - \\
\hline\hline
\end{tabular*}
\end{table}

As an example of the use of the PT, consider the fact that $1b=1$; that
  is, $b \in \langle a \rangle$.
To prove this formally, we exhibit a word which is equivalent to $b$ and
  which reduces to a product of subgroup generators (i.e., $a$'s).
Expanding $_1b_1$ yields $_10_2b_1$, and expanding both parts of this
  yields $_10_3 A_2 e_1 a_1[AEb]_1$.
Successively expanding this now gives
  $$\begin{array}{l}
  _1[CeD]_1 d_2 E_1 c_3 E_3 D_2[deA]_2 e_1 [[a]]_1 [AEb]_1 , \\
  _1[CeD]_1 a_1E_2[eAd]_2 a_3B_1[bAE]_1 c_3 D_2C_3[cdE]_3 C_1B_2[bcD]_2
    [deA]_2 e_1 [[a]]_1 [AEb]_1 , \\
  _1[CeD]_1 [[a]]_1 E_2[eAd]_2 d_3e_3[EDa]_3 C_1a_1[AcB]_1 [bAE]_1 c_3
    C_1B_2[bcD]_2 D_1b_3[BdC]_3  \\
    \quad\quad [cdE]_3 C_1 A_1E_2[eaB]_2 [bcD]_2 [deA]_2 e_1 [[a]]_1
    [AEb]_1 .
  \end{array}$$
At this point we can cancel $_1c_3C_1$ to simply $_1$, and continue
  $$\begin{array}{l}
  _1[CeD]_1 [[a]]_1 E_2[eAd]_2 b_1c_3[CBd]_3 c_2d_3[DCe]_3 [EDa]_3 C_1
    [[a]]_1 [AcB]_1 [bAE]_1 A_1E_2 \\
    \quad\quad [eaB]_2 [bcD]_2 e_1A_1[aED]_1 A_1c_3[Cab]_3 [BdC]_3 
    [cdE]_3 C_1 [[A]]_1 E_2[eaB]_2 [bcD]_2 \\
    \quad\quad [deA]_2 e_1 [[a]]_1 [AEb]_1 .
  \end{array}$$
Another three steps now yields the final proof word
  $$\begin{array}{l}
  _1[CeD]_1[[a]]_1E_2[eAd]_2e_1[[a]]_1[AEb]_1c_3[CBd]_3C_1[[a]]_1[AcB]_1
    [[a]]_1E_2[eAd]_2[Dbc]_2e_1 \\
    \quad\quad [[a]]_1[AEb]_1c_3[CBd]_3[DCe]_3[EDa]_3C_1
    [[a]]_1[AcB]_1[bAE]_1[[A]]_1E_2[eaB]_2[bcD]_2 \\
    \quad\quad e_1[[A]]_1[aED]_1[[A]]_1c_3[Cab]_3[BdC]_3[cdE]_3C_1[[A]]_1
    E_2[eaB]_2[bcD]_2[deA]_2e_1 \\
    \quad\quad [[a]]_1[AEb]_1 .
  \end{array}$$
If this entire word is freely reduced, then we recover the word $b$, while
  if we freely reduce after deleting the relators, then we obtain the
  word $a^3$\kern-1.5pt.
Since free reductions and relator deletions are reversible, this word 
  proves that $b=a^3$\kern-1.5pt.

%%%%%%%%%%%%%%%%%%%%%%%%%%%%%%%%%%%%%%%%%%%%%%%%%%%%%%%%%%%%%%%%%%%%%%%%%
%%
%E
