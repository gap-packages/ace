
% changes.tex - Colin Ramsay - 17 Dec 01
%
%   5   10   15   20   25   30   35   40   45   50   55   60   65   70   75
% ..|....|....|....|....|....|....|....|....|....|....|....|....|....|....|

This chapter documents the changes made to \ace\ 3.000 as part of the
  upgrade to \ace\ 3.001.
This includes changes made to development versions of \ace\ which came
  after 3.000 but before the official 3.001 release \amp code freeze, and
  temporary code introduced for one reason or another and then deleted.
Some minor bugs have been fixed, some internal reorganisation to make the
  code `cleaner' was done, some changes in response to feedback from
  users/installers of \ace\ were made, and some new/enhanced options are
  available.
In addition, the development of \pace\ 1.000 \amp \peace\ 1.000 -- which
  were `based' on \ace\ 3.000, and preceded \ace\ 3.001 -- suggested
  several `improvements' to \ace.

There are no substantive changes, and \ace\ 3.001 is back-compatable with 
  \ace\ 3.000.
The behaviour of \ace, as far as the user at any of the three interface
  levels is concerned, remains the same.
(As far as I am aware, these statements are not false, but check the list
  of changes carefully if you have any problems.)

Some general changes were :-

i)  The removal/tidying-up of the various DTT (Debug/Test/Trace) code which
    was added at one time or another to assist in development.
The addition of various new bits of DTT code.

ii) The (extensive) comments in the source were `edited to enhance
    clarity'.

To some extent, \ace\ 3.001, \pace\ 1.000 \amp \peace\ 1.000 were developed
  in parallel from \ace\ 3.000.
Each of the changes to \ace\ listed below is flagged to indicate whether or
  not is has been carried over to the current versions of \pace/\peace\ as
  well.
The first symbol(s) is/are for \pace\ \amp the second for \peace; 
  ``?'' means don't know/care,
  ``y'' means yes,
  ``n'' means no, and
  ``x'' not applicable.

The changes/corrections consisted of :- 

undated (code reorganisation) :- y,x \\
When a coinc is queued (\amp cols 1/2 processed), the correct
  representative is picked up (the CREP macro).
When we come to process the remaining columns, the rep've may have become
  invalid.
So we could (it's not mandatory) calculate the new rep've, in an attempt
  to prevent moving data multiple times.
Some tests showed that this is not beneficial, so the calculation of
  rep'ves at the top of the processing loop in \_coinc() was removed.
Ditto for the ability to path compress there.
(See the comments in coinc.c in the 3.000 \amp 3.001 source code.)
Of course, we \emph{must} retain the use of rep'ves in \_cols12(), and the 
  ability to path compress there (if requested).

7 Mar 99 (minor bug) :- y,y \\
At several places in enum.c, we did ``msgnext = msgctrl'' instead of 
  ``msgnext = msgincr''.
This could cause spurious additional progress messages, if msgctrl was 1
  (ie, messaging was active).

5 Jul 99 (name change) :- y,x \\
Pre-ACE3, \ace\ stood for ``adaptive coset enumerator'', now it stands for
  ``advanced coset enumerator''.
The ``adaptive'' was because \ace\ could modify its behaviour in the
  presence of a big collapse, which could be \emph{very} slow.
During the development of ACE3, a better (more advanced!) understanding of
  deduction handling was gained.
This led to the introduction of the `pmod' \amp `psiz' parameters, and
  removed the need for the adaptive flag.
Of course, there are many situations where \ace\ could profitably change
  its enumeration strategy on the fly based on recent events, and we
  reserve the right to implement some of these in future versions and
  change \ace's name back to its original meaning.

12 Aug 99 (sloppyness) :- x/y,y \\
The entry for `loop' on the help screen included a spurious initial ``/''
  in the argument part.
The argument of `no' was changed from ``-1/0/..'' to ``-1/0/1..'' to more
  accurately reflect the trichotomy of meaning.

3 Sep 99 (optimisation problem) :- y,y \\
Some compilers complained when trying to optimise the main control loop for
  the despatcher in `al2\_cmdloop()', since it's very long.
Of course, optimising this is not relevant, since it only has to run at
  human speed; but still ...
The original single ``if (.) {.} else if (.) {.} ... else {.}'' has been 
  split into separate ``if (.) {.}'' statements, and ``continue''
  statements used to skip to the end of the enclosing ``while'' statement
  from the end of each ``if''.
This stops some, but \emph{not} all, compilers from complaining.

8 Sep 99 (minor bug) :- y,y \\
At various places where we want neatly aligned output, I'd swallowed the
  separating space(s) into the field width; eg, I'd used ``\%5d\%5d''
  instead of `` \%4d \%4d'' for the `fprintf' format strings.
This could be a problem if our numbers become big, since we'd lose the
  spacing between them.
The main concern here is packages such as GAP which use the output of \ace\
  as input.
All such usages have been removed; so that the numbers are guaranteed to be
  separated, although perhaps not vertically aligned.
(In C, the field width is a \emph{minimum}, and will be adjusted up if the 
  number is too big.)

21 Sep 99 (sloppyness) :- y,y \\
All the various ``\#-{}- xxx -{}-'' messages have been tuned so that
  they're all the same width \amp the text in them is central.

22 Sep 99 (stupid bug) :- x,x \\
The `rc' command did not correctly restore the original list of subgroup
  generators if this list was non-empty; it left in the first of the
  added generators.
This was due to bad loop limits, and has been fixed.

18 Oct 00 (tidy up) :- y,x \\
At various times, test code has been added to \ace\ to allow the printing
  of the definition sequence used in an enumeration.
This has many uses:
  prune the sequence to try to find a `smallest' possible one;
  extract a formal proof of (sub)group membership;
  generate Schreier generators for the subgroup.
With the development of \peace, it was decided not to formally incorporate
  any of this work in \ace\ (contrary to earlier intentions).
So, all traces of this feature have been removed from \ace\ 3.001.

6 Dec 00 (enhanced functionality) :- n,n \\
The `sr' command only allowed brief (arg 0) or full (arg 1) printout of the
  enumeration details.
Now, some commands accept as a valid value an empty argument, and so the 
  `echo data' mode cannot be used to print their current value.
In particular, this applied to the `enum', `rel', `gen', `subgrp' commands.
The `sr' command has been enhanced, with arguments 2--5, to allow these 
  items to be printed individually.

11 Dec 00 (redo/sgdone bug) :- x,x \\
In those redo modes where a CL is done, this is done \emph{before} the SG
  phase.
If the CL phase causes a finite index (ie, a collapse to 1), then the
  redo returns before the SG, leaving the sgdone flag (incorrectly) clear.
The problem was solved by setting the sgdone flag in the \_chk1() routine
  in coinc.c.
See the ``test019*'' test files in the distribution.
(Philosophical aside: the whole question of how to handle `redo', `cc', etc
  needs rethinking, since the CL phase can be \emph{slow}!)

20 Dec 00 (code tidyup) :- y,x \\
i) The SAVED00 macro in the SAVED macro, which keeps statistics on the max
  dedn stack size, was incorrectly placed in that it wasn't always called;
  it's been moved.
(Philosophical aside: a call to SAVED which causes a stack overflow and a
  call to al0\_dedn() can produce an intermediate sized stack bigger than
  the one recorded by SAVED00, due to the stack pruning in al0\_dedn().
Since the miscount at a DS can be at most 1, we don't worry about it.) \\
ii) An experimental dedn mode \#5, which stacked dedns 
  non-deterministically based on a nominated percentage, was tested (see
  `snippets.c').
It's `interesting', but it's not part of \ace\ 3.001, so all traces of it
  have been removed.
(Philosophical aside: such code more properly should be in \pace, which is
  the preferred vehicle for experiments regarding dedn handling.)

22 Dec 00 (Linux/gcc bug \amp coset rep've bug) :- y,x \\
i) On some Linux/gcc combinations, \ace\ died with a segmentation fault
  when trying to construct coset representatives (eg, `pr', `oo' commands).
At the time, extensive rummaging failed to find the cause of the fault.
Since its presence/absence seemed to depend on which version of gcc was
  used, it was concluded that there were bugs in various versions of gcc.
(The `new' versions of gcc which accompany each `new' distribution of
  Linux are notoriously buggy.) \\
ii) It turns out that the al1\_bldrep() routine uses a variable scol which
  could be used uninitialised.
The particular set of circumstances which could cause this weren't
  considered (ie, thought of) when the routine was tested.
The problem is easily solved by setting \amp testing scol appropriately.
See the ``test021*'' test files in the distribution.
 
2 Mar 01 (max bug) :- y,y \\
After a `begin' which overflows, we can reset `max' before running a
  `cont'.
If maxrow1<nextdf the new value of `max' is rounded up in al2\_start() (in 
  control.c) so that no coset table data is lost.
In an attempt to be clever, it is rounded up to nextdf, to allow one more
  coset (there will be nextdf-1 cosets currently).
Sadly, if the overflow happened with nextdf = maxrow+1 = tabsiz+1, this
  allows cosets past the end of the table.
On some machine/OS/compiler combinations, this could cause a crash.
It's easily fixed by abandoning attempts to be clever, and simply
  (re)setting maxrow to nextdf-1.
See the ``test023*'' test files in the distribution.

2 Mar 01 (enhanced rc functionality) :- x,x \\
i) The printout during a run was altered to include the current iteration
  number. \\
ii) A 0 argument, meaning any non-trivial index, was added.

