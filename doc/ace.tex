%%%%%%%%%%%%%%%%%%%%%%%%%%%%%%%%%%%%%%%%%%%%%%%%%%%%%%%%%%%%%%%%%%%%%%%%%
%%
%W  ace.tex                ACE documentation             Alexander Hulpke
%W                                                      Joachim Neub"user
%W                                                            Greg Gamble
%%
%H  $Id$
%%
%Y  Copyright (C) 2000, School of Math & Comp. Sci., St Andrews, Scotland
%%

\def\ACE{{\sf ACE}}
%%%%%%%%%%%%%%%%%%%%%%%%%%%%%%%%%%%%%%%%%%%%%%%%%%%%%%%%%%%%%%%%%%%%%%%%%
\Chapter{The ACE Share Package}

The  \lq{}Advanced Coset Enumerator' {\ACE}:

\begintt
ACE coset enumerator (C) 1995-1999 by George Havas and Colin Ramsay
    http://www.csee.uq.edu.au/~havas/ace3.tar.gz
\endtt

can  be called  from within  {\GAP} through  an interface,  written by
Alexander Hulpke and Greg Gamble, which is described in this manual.

The interface links to an external binary and therefore is only usable
under UNIX (see Section~"Installing the ACE share package" for how  to
install {\ACE}).  It will not work  on  Windows  or the Macintosh.  It
requires {\GAP}~4.

{\ACE} can be used through this interface in different ways:

%%%%%%%%%%%%%%%%%%%%%%%%%%%%%%%%%%%%%%%%%%%%%%%%%%%%%%%%%%%%%%%%%%%%%%%
\Section{Using ACE as a Default for Coset Enumerations and some General
Warnings}

After loading  {\ACE}   (see Section~"Loading the ACE share package"),
if you  want to use the {\ACE}  coset enumerator as a  default for all
coset  enumerations  done  by   {\GAP}  (which  may  also  get  called
indirectly),  you can  achieve  this by  setting  the global  variable
`TCENUM' to `ACETCENUM'.

\begintt
gap> TCENUM:=ACETCENUM;;
\endtt

This    sets    the    function    `CosetTableFromGensAndRels'    (see
Section~"ref:Coset  Tables  and  Coset  Enumeration"  in  the   {\GAP}
Reference  Manual)  to  be  the  function  `CallACE'   (described   in
Section~"CallACE"), which then can be  called  with  all  the  options
defined for the {\ACE} interface,  not  just  the  options  `max'  and
`silent'. If `TCENUM'  is  set  to  `ACETCENUM'  without  any  further
action, the `default' strategy of the {\ACE} enumerator will  be  used
(see Chapter~"Strategy Options for the  ACE  Enumerator").  An  {\ACE}
strategy is merely a special option of {\ACE} that sets  a  number  of
the options described in Chapter~"Options for ACE"  all
at once. One can set options globally using the function `PushOptions'
(see Chapter~"ref:Options Stack"  in  the  {\GAP}  Reference  Manual);
options pushed onto `OptionsStack' this way,  remain  there  until  an
explicit `PopOptions()' call is made. However, the usual way  to  pass
options is  behind  a  colon  following  a  function's  arguments  (as
described  for  the  function  `CallACE'  in  Section~"CallACE",   for
example); options passed  this  way  are  local,  and  disappear  from
`OptionsStack' after the function has executed  successfully.  Beware,
however the function will see any other options on the  `OptionsStack'
(in particular, any pushed  there  via  `PushOptions',  prior  to  the
function call),  unless  over-ridden  by  an  option  passed  via  the
function. Also note that duplication of  option  names  for  different
programs   may   lead   to    misinterpretations.    You    can    use
`DisplayOptionsStack' (see  Chapter~"ref:DisplayOptionsStack"  in  the
{\GAP} Reference Manual) to ensure  that  there  is  no  such  danger.
However, considering how much of a potential  mine-field  a  non-empty
`OptionsStack' might be for the unwary user, we provide

\>FlushOptionsStack()

which simply executes `PopOptions()' until `OptionsStack' is empty.

It  is  important  to  realize  that  {\ACE}'s   options   (even   the
non-strategy options) are not orthogonal, i.e.\  the  order  in  which
they are put to {\ACE} can be important. For this reason, except for a
few options that have no effect on the course of an  enumeration,  the
order in which options are passed to the {\ACE} interface is preserved
when those same options are passed to the  {\ACE}  binary.  Also  note
that except for limitations  imposed  by  {\GAP}  e.g.\  clashes  with
{\GAP} keywords and blank spaces not allowed in keywords, the  options
of the {\ACE} interface are the  same  as  for  the  binary;  so,  for
example, the options can appear in upper or  lower  case  (or  indeed,
mixed case) and most may be abbreviated. Note that  any  options  that
the {\ACE} binary doesn't understand are simply ignored and a  warning
is sent  to  standard  error  (which  should  appear  in  your  {\GAP}
session). If this occurs, the input fed to {\ACE} and the output  from
{\ACE} will be files in a temporary directory; the full path to  these
files  is  recorded  in  the  record   fields   `ACEinfo.infile'   and
`ACEinfo.outfile', respectively.

You can switch back to the coset  enumerator  built  into  the  {\GAP}
library by assigning `TCENUM' to `GAPTCENUM'.

\begintt
gap> TCENUM:=GAPTCENUM;;
\endtt

%%%%%%%%%%%%%%%%%%%%%%%%%%%%%%%%%%%%%%%%%%%%%%%%%%%%%%%%%%%%%%%%%%%%%%
\Section{Calling ACE Directly}

If on the other hand you do not want to set  up  {\ACE}  globally  for
your coset enumerations, you may call the {\ACE}  interface  directly,
which will allow you to decide for yourself, for each such call, which
options you want to use for running {\ACE}. Please note  the  warnings
regarding options in the preceding section. By such  a  call,  a  file
with your input data in {\ACE}  readable  format  will  be  handed  to
{\ACE} and you will get the answer back in some {\GAP} format. At that
moment however the {\ACE} job is terminated, that is, you cannot  send
any further questions or requests about the  result  of  that  job  to
{\ACE}; you are not using {\ACE} interactively. For an interactive use
of {\ACE} from {\GAP} see Chapter~"Using ACE Interactively".

Calling the {\ACE} interface directly is done by

\>CallACE( <fgens>, <rels>, <sgens> : <options> )

Here <fgens> is  a list of free generators, <rels> a  list of words in
these generators  giving relators for a finitely  presented group, and
<sgens> the list  of subgroup generators, again expressed  as words in
the free generators. All these are given in the standard {\GAP} format
(See Chapter~"ref:Finitely  Presented Groups" of  the {\GAP} Reference
Manual).

Behind the colon any  selection  of  the  options  available  for  the
interface (see Chapters~"Options for ACE"  and~"Strategy  Options  for
ACE") can be given, separated by commas like record  components.  Note
that strategies are simply special options that set a  number  of  the
options, detailed in Chapter~"Options for ACE", all at once. These can
be used e.g.~to preset limits of space and time to be used, to  modify
input and output and to modify the enumeration procedure.

Please  see Section~"Using ACE as a Default for Coset Enumerations and 
some General Warnings" for a discussion  regarding  global  and  local
passing of options, and the non-orthogonal nature of {\ACE}'s options.

The  function   calls the  {\ACE}  coset  enumerator  and returns  the
(standardized) coset table obtained.

If the coset enumeration does not  finish  in  the  preset  limits  it
raises an error, unless the `silent' option (see~"Option  silent")  is
set, in which case it returns `fail'.

The example given  below  is  the  call  for  a  presentation  of  the
Fibonacci group F(2,7) for  which  we  shall  discuss  the  impact  of
various options in Chapter~"Examples". Observe  that  in  the  example
below, no options  are  passed,  which  means  that  {\ACE}  uses  the
`default' strategy (see Chapter~"Strategy Options for ACE").

\begintt
gap> F:= FreeGroup( "a", "b", "c", "d", "e", "f", "g");;
gap> a:= F.1;; b:= F.2;; c:= F.3;; d:= F.4;; e:= F.5;; f:= F.6;; g:= F.7;;
gap> fgens:= [a, b, c, d, e, f, g,];;
gap> rels:= [
> a*b*c^-1,
> b*c*d^-1,
> c*d*e^-1,
> d*e*f^-1,
> e*f*g^-1,
> f*g*a^-1,
> g*a*b^-1];;
gap> CallACE(fgens, rels, [c]);;
\endtt

(The  variable  `ACETCENUM.CosetTableFromGensAndRels'  is assigned  to
`CallACE'.)

If you only want to  test, whether a coset enumeration terminates, and
don't want to  transfer the whole coset table  to {\GAP}. Instead, you
can call

\>ACEStats( <fgens>, <rels>, <sgens> : <options> )

which returns a record `rec(index=<i>,  cputimeStr=<cs>,  cputime=<c>,
maxcosets=<m>, totcosets=<t>]' where <i> is the index of the  subgroup
$\langle <sgens> \rangle$ in $\langle <fgens> \mid <rels> \rangle$ (or
$0$, whenever the enumeration does not succeed); <cs> is the total CPU
time used for the enumeration (in seconds) as a string;  <c>  is  also
the total CPU time, but as a  rational  number;  <m>  is  the  maximum
number  of  \lq{}alive'  coset  numbers  at  any  one  time   in   the
enumeration; and <t> is the total number of coset  numbers  that  were
defined in the  enumeration.  See  also  Section~"Options  controlling
Input and Output". The same warnings noted in Section~"Using ACE as  a
Default for Coset Enumerations and some General  Warnings",  regarding
options, apply.

%%%%%%%%%%%%%%%%%%%%%%%%%%%%%%%%%%%%%%%%%%%%%%%%%%%%%%%%%%%%%%%%%%%%%%
\Section{Writing ACE Standalone Input Files}

If you want to use ACE as a standalone with its own  syntax,  you  can
write  a  {\GAP}  input  and  can  use  `CallACE'  with   the   option
`ACEinfile:=<filename>' (see~"Option ACEinfile"). This will  keep  the
input file for the {\ACE} standalone produced by the {\GAP}  interface
under the file name <filename> (and just return `fail')  so  that  you
can perform interactive work in the standalone.

%%%%%%%%%%%%%%%%%%%%%%%%%%%%%%%%%%%%%%%%%%%%%%%%%%%%%%%%%%%%%%%%%%%%%%
\Section{Calling ACE Interactively}

Further features  of the {\ACE} standalone  can be used  from GAP only
through an interactive link.

%%%%%%%%%%%%%%%%%%%%%%%%%%%%%%%%%%%%%%%%%%%%%%%%%%%%%%%%%%%%%%%%%%%%%%

\Section{Acknowledgement}

Large parts of this manual,  in  particular  the  description  of  the
options for running {\ACE}, are directly copied  from  the  respective
description in the manual~\cite{Ram99} for the standalone  version  of
{\ACE} by Colin Ramsay.

%%%%%%%%%%%%%%%%%%%%%%%%%%%%%%%%%%%%%%%%%%%%%%%%%%%%%%%%%%%%%%%%%%%%%%%%%
\Chapter{Installing and Loading the ACE Share Package}

%%%%%%%%%%%%%%%%%%%%%%%%%%%%%%%%%%%%%%%%%%%%%%%%%%%%%%%%%%%%%%%%%%%%%%
\Section{Installing the ACE share package}

To  install, unpack  the  archive file  in  a directory  in the  `pkg'
hierarchy  of your  version  of  {\GAP}~4. (This  might  be the  `pkg'
directory of the {\GAP}~4 home  directory; it is however also possible
to keep an additional `pkg' directory in your private directories, see
Section~"ref:Installing  Share Packages"  of  the {\GAP}~4   Reference
Manual for details  on how to do this.) Go to  the newly created `ACE'
directory and  call `./configure <path>'  where <path> is the  path to
the {\GAP} home  directory. So for example if  you install the package
in the main `pkg' directory call

\begintt
./configure ../..
\endtt

This  will fetch  the  architecture  type for  which  {\GAP} has  been
compiled last and create a `Makefile'. Now simply call

\begintt
make
\endtt

to compile the binary and to install it in the appropriate place.

Note that the  current version of the configuration  process only sets
up  directory paths.  If you  need a  different compiler  or different
compiler options, you need  to edit `src/Makefile.in' prior to calling
`make' yourself.

If you use this installation of {\GAP} on different hardware platforms
you will have to compile the binary for each platform separately. This
is  done  by calling  `configure'  and  `make'  for the  package  anew
immediately   after  compiling  {\GAP}   itself  for   the  respective
architecture.  If your version of  {\GAP} is already compiled (and has
last  been compiled  on  the same  architecture)  you do  not need  to
compile {\GAP} again, it is  sufficient to call the `configure' script
in the {\GAP} home directory.

The manual you are currently reading describes how to use  the  {\ACE}
share package; it can be  found  in  the  `doc'  subdirectory  of  the
package.

The  subdirectory  `standalone-doc'  contains the  file  `ace3001.dvi'
which holds a version of the user manual for the {\ACE} standalone; it
forms part of~\cite{Ram99}).  You  should consult it if  you are going
to  switch to  the {\ACE}  standalone, e.g.~in  order to  directly use
interactive facilities.

The  `src' subdirectory  contains a  copy  of the  original source  of
{\ACE}.  (The  only modification  is  that  a  file `Makefile.in'  was
obtained from  the different `make.xyz' and  will be used  to create a
`Makefile'.)  You  can replace  the source by  a newer  version before
compiling.

%%%%%%%%%%%%%%%%%%%%%%%%%%%%%%%%%%%%%%%%%%%%%%%%%%%%%%%%%%%%%%%%%%%%%%
\Section{Loading the ACE share package}

To use the {\ACE} package you have to request it explicitly.  This  is
done by calling

\begintt
gap> RequirePackage( "ace" );
\endtt

See Section~"ref:RequirePackage" in the {\GAP} Reference Manual.

If {\GAP} cannot find a working binary, the call  to  `RequirePackage'
will fail.

If you want to load the {\ACE} package by default, you  can  put  this
`RequirePackage' command into your `gaprc' file (see  Section~"ref:The
.gaprc file" in the {\GAP} Reference Manual).

%%%%%%%%%%%%%%%%%%%%%%%%%%%%%%%%%%%%%%%%%%%%%%%%%%%%%%%%%%%%%%%%%%%%%%%%%
\Chapter{Some Basics}

Throughout this manual for  the  use  of  {\ACE}  as  a  {\GAP}  share
package, we shall assume that the reader already knows the basic ideas
of coset enumeration, as can be  found  for  example  in~\cite{Neu82}.
There, a simple proof is given for the fact that a  coset  enumeration
for a subgroup of finite index in  a  finitely  presented  group  must
eventually terminate with the correct result, provided the enumeration
process obeys a simple condition (Mendelsohn's  condition)  formulated
in Lemma~1 and Theorem~2 of~\cite{Neu82}. This basic condition  leaves
room for a great variety of \lq{}strategies'  for  coset  enumeration;
two \lq{}classical' ones, have been known  for  a  long  time  as  the
\lq{}Felsch strategy' and  the  \lq{}HLT  strategy'  (for  Haselgrove,
Leech and Trotter). Extensive experimental studies on many  strategies
can  be  found  in~\cite{CDHW73},  \cite{Hav91},   \cite{HR99a},   and
\cite{HR99b}, in particular.

A few basic points should be particularly understood:

\beginlist

\item{--} \lq{}Subgroup(generator) and relator tables' that  are  used
in the description of coset enumeration in \cite{Neu82}, and to  which
we will also occasionally refer in this manual,  do  *not*  physically
exist in the implementation of coset  enumeration  in  {\ACE}.  For  a
terminology that is closer to the actual implementation  and  also  to
the  formulations  in  the  manual  for  the  {\ACE}  standalone   see
\cite{CDHW73} and \cite{Hav91}.

\item{--} Coset enumeration proceeds by defining  \lq{}coset  numbers'
that really denote possible  representatives  for  cosets  written  as
words in the generators of the group. At the time of their  generation
it is not guaranteed that any two of these words do  indeed  represent
different cosets.

\item{--} It is customary in  talking about coset enumeration to speak
of \lq{}cosets' when really coset  numbers are meant. While  we try to
avoid this in this interface manual, in certain word combinations such
as \lq{}coset application' we will follow this custom.

\item{--} The   definition   of   a   coset   number   may   lead   to
\lq{}deductions' from the \lq{}closing of rows in subgroup or  relator
tables'. These are kept in a \lq{}deduction stack'.

\item{--} Also   it  may  be  found  that  (different)  words  in  the
generators defining different coset numbers really  lie  in  the  same
coset of the given subgroup. This is called  a  \lq{}coincidence'  and
will eventually lead to the elimination of the larger of the two coset
numbers.  Until  this   elimination   has   been   performed   pending
coincidences are kept in a \lq{}queue of coincidences'.

\item{--} A definition that will actually close a row in a subgroup or
relator table will immediately yield twice  as  many  entries  in  the
coset  table  as  other  definitions.  Such  definitions  are   called
\lq{}preferred definitions', the places  in  rows  of  a  subgroup  or
relator table that they close are also referred  to  as  \lq{}gaps  of
length one' or minimal gaps. Such gaps can be found  at  little  extra
cost when \lq{}relators are traced from a given coset number'.  {\ACE}
keeps a selection of them in a \lq{}preferred  definition  stack'  for
use in some definition strategies (see~\cite{Hav91}).

\endlist

It will also be necessary to understand some further basic features of
the  implementation and  the corresponding  terminology which  we will
explain in the sequel.

%%%%%%%%%%%%%%%%%%%%%%%%%%%%%%%%%%%%%%%%%%%%%%%%%%%%%%%%%%%%%%%%%%%%%%%%%
\Section{Enumeration  Style}

The first main decision for any coset enumeration is in which sequence
to make definitions. When a new coset number has  to  be  defined,  in
{\ACE} there are basically three possible methods to choose from:

\beginlist

\item{--} One may fill the next empty entry  in  the  coset  table  by
scanning from the left/top of the coset table towards the right/bottom
-- that is, in order row by row. This is called  *C-style  definition*
(for *C*oset Table Based definition)  of  coset  numbers.  In  fact  a
procedure needs to follow a method like this to some  extent  for  the
proofs that coset enumeration eventually terminates  in  the  case  of
finite index (see~\cite{Neu82}).

\item{--} In *R-style definition* (for *R*elator Based definition) the
order  in which  cosets are  defined is  explicitly prescribed  by the
order in which rows of (the subgroup generator tables and) the relator
tables are filled by making definitions.

\item{--} One may choose  definitions  from  a  *Preferred  Definition
Stack*. In this stack possibilities for definition  of  coset  numbers
are stored that will close a certain row of  a  relator  table.  Using
these \lq{}preferred definitions' is sometimes also referred to  as  a
\lq{}minimal gaps strategy'. The  idea  of  using  these  is  that  by
closing a row in a relator table, thus, one  will  immediately  get  a
consequence. We will come back to the obvious question  of  where  one
obtains this \lq{}preferred definition stack'.

\endlist

The *enumeration style* is mainly determined by  the  balance  between
C-style definitions and R-style definitions, which  is  controlled  by
the values of the `ct' and `rt' options (see~"Option  ct"  and~"Option
rt").

However this still leaves us with  plenty of freedom for the design of
definition  strategies,  freedom which  can, for example,  be  used to
great advantage in Felsch-type strategies. Though it is  not  strictly
necessary, Felsch-type  programs generally start off  by ensuring that
each of the given subgroup generators  forms a cycle at coset 1 before
embarking  on   further  enumeration.  The  use  of   this  and  other
possibilities leads to the following table of *enumeration styles*.


% \begin{table}
% \hrule
% \caption{The styles}
% \label{tab:sty}
% \smallskip
% \renewcommand{\arraystretch}{0.875}
% \begin{tabular*}{\textwidth}{@{\extracolsep{\fill}}crrlc} 
% \hline\hline
% & \ttt{Rt} value & \ttt{Ct} value & style name & \\
% \hline
% & $<\!0$ & $<\!0$ & R/C & \\
% & $<\!0$ & $0$    & R*  & \\
% & $<\!0$ & $>\!0$ & Cr  & \\
% & $0$    & $<\!0$ & C   & \\
% & $0$    & $0$    & R/C (defaulted) & \\
% & $0$    & $>\!0$ & C  & \\
% & $>\!0$ & $<\!0$ & Rc & \\
% & $>\!0$ & $0$    & R  & \\
% & $>\!0$ & $>\!0$ & CR & \\
% \hline\hline
% \end{tabular*}
% \end{table}
\begintt
Rt value     Ct value     style name
---------------------------------------

   0           >0         C
  <0           >0         Cr
  >0           >0         CR

  >0            0         R
  <0            0         R*
  >0           <0         Rc
  <0           <0         R/C
   0            0         R/C (default)

---------------------------------------
\endtt

In *C-style*  most definitions are made in the  next empty coset table
slot  and  are (in  principle)  tested  in  all essentially  different
positions in the relators; i.e.~this is a Felsch-like mode.

However in C-Style some  definitions may be made following a preferred
definition strategy, see the `pmode' and `psize' options below.

*Cr-style* is like  C-style except that a single R-style  pass is done
after the initial C-style pass.

In *CR-style* alternate passes of C-style and R-style are performed.

In *R-style*  all   the  definitions  are   made  via  relator  scans; 
i.e.~this is an HLT-like mode.

*R\*-style*  makes  definitions the  same as  R-style,  but  tests all
definitions as for C-style.

*Rc-style* is like R-style, except  that a single C-style pass is done
after the initial R-style pass.

In  *R/C-style*  we  run  in  R-style  until  an  overflow, perform  a
lookahead on the entire table, and then switch to CR-style.

*Defaulted R/C-style*  is the default  style, used if  you call {\ACE}
without specifying  options. In it we  use R/C-style with  `ct' set to
1000 and `rt' set to  approximately $2000$ divided by the total length
of the relators  in an attempt to balance R and  C definitions when we
switch to CR-style.

%%%%%%%%%%%%%%%%%%%%%%%%%%%%%%%%%%%%%%%%%%%%%%%%%%%%%%%%%%%%%%%%%%%%%%
\Section{Finding Deductions, Coincidences, and Preferred Definitions} 

*TO BE WRITTEN*

%%%%%%%%%%%%%%%%%%%%%%%%%%%%%%%%%%%%%%%%%%%%%%%%%%%%%%%%%%%%%%%%%%%%%%

\Chapter{Options for ACE}

{\ACE} offers a wide range of options to  direct  and  guide  a  coset
enumeration, most of which are  available  from  {\GAP}  through  this
interface. We describe most of the options available via the interface
in this chapter; other options, termed strategies, are defined in  the
following chapter. Strategies are merely  special  options  of  {\ACE}
that set a number of options described in this chapter, all at once.

Except for limitations imposed by {\GAP}  e.g.\  clashes  with  {\GAP}
keywords and blank spaces not allowed in keywords, the options of  the
{\ACE} interface are the same as for the binary; so, for example,  the
options can appear in upper or lower case (or indeed, mixed case)  and
most may be abbreviated. Below we only list the options in  all  lower
case, and in their longest form; where  abbreviation  is  possible  we
give the shortest abbreviation in the  option's  description  e.g.~for
the `mendelsohn' option we state that  its  shortest  abbreviation  is
`mend', which means `mende', `mendel' etc.,  and  indeed,  `Mend'  and
`MeND', are all valid abbreviations of that option. Some options  have
synonyms e.g.~`cfactor' is an alternative for `ct'.

It  is  important  to  realize  that  {\ACE}'s   options   (even   the
non-strategy options) are not orthogonal, i.e.\  the  order  in  which
they are put to {\ACE} can be important. For this reason, except for a
few options that have no effect on the course of an  enumeration,  the
order in which options are passed to the {\ACE} interface is preserved
when those same options are passed to the {\ACE} binary.  One  of  the
reasons for the non-orthogonality of options is to  protect  the  user
from obtaining invalid enumerations from bad combinations of  options;
the general rule is that the later option prevails.

Any option that the {\ACE} binary doesn't understand is simply ignored
and a warning is sent to standard error (which should appear  in  your
{\GAP} session). If this occurs, the  input  fed  to  {\ACE}  and  the
output from {\ACE} will be files in a temporary  directory;  the  full
path to these files is recorded in the record fields  `ACEinfo.infile'
and `ACEinfo.outfile', respectively.

If  `CallACE'  (see  Section~"CallACE")  is  given  no  options,   the
`default'  strategy  (see  Chapter~"Strategy  Options  for  ACE")   is
selected. If the `default' strategy does not suffice, most  usually  a
user will select one of the  other  strategies  from  among  the  ones
listed in Chapter~"Strategy Options for ACE", and possibly modify some
of the options by selecting from the options in this chapter. It's not
illegal to select more than one strategy, but it's  not  sensible;  as
mentioned above, the later one prevails.

Please see Section~"Using ACE as a Default for Coset Enumerations  and
some  General  Warnings"  for  a  general  discussion  of  the  option
mechanism and of the pitfalls one may encounter,  in  the  passing  of
options. There are two ways options may be passed to {\ACE}:

\beginlist

\item{--} Options may be set globally using the function `PushOptions'.

\item{--} Options may be appended to the argument list of any function
call,  separated by a  colon from  the argument  list;  they  are then
passed on recursively to any subsequent inner function call, which may
in turn have options of their own.

%\item{--} If {\ACE} is to be used interactively .....
%*TO BE FILLED IN*

\endlist

Options are given like record components, separated by commas. Options
that take a value are given as an assignment  to  the  name  (such  as
`time:=<val>'). Options passed without assigning a value, are  treated
by {\GAP} as the setting of a boolean option to the value  `true'.  If
the  option  is  a  known  {\ACE}  option,  `CallACE'  will  make  the
appropriate translation to the value expected by  the  {\ACE}  binary;
the appropriate translation may be to pass to the binary: no value, 0,
or 1. In general, this should not concern  the  user  since  `CallACE'
will take care of it. Options announced as unknown (options are echoed
to the screen, if the `echo' option is passed) will *always* be passed
to {\ACE} as no-value options; the user can over-ride  this  behaviour
simply by assigning the intended value. The reason for passing unknown
options in this way, is that the {\ACE} interface should  still  work,
albeit with erroneous warning messages, with an  updated  binary.  The
{\ACE} options known to the {\ACE} interface are  the  fields  of  the
record `KnownACEoptions'; each field (known {\ACE} option) is assigned
to a list of the form `[<i>, <ListOrFunction>]',  where  `<i>'  is  an
integer representing the  shortest  abbreviation  of  the  option  and
`<ListOrFunction>' is either a list of (known)  allowed  values  or  a
boolean function that may be used to determine if the given value is a
(known) valid value e.g.

\begintt
gap> KnownACEoptions.compaction;
[ 3, [ 0 .. 100 ] ]
\endtt

indicates that the option `compaction' may be  abbreviated  to  `com'
and the (known) valid values are in the (integer) range 0 to 100; and

\begintt
gap> KnownACEoptions.ct;
[ 2, <Operation "IS_INT"> ]
\endtt

indicates that there is essentially no abbreviation of `ct' (since its
shortest abbreviation is of length 2),  and a value of  `ct' is  known
to be valid if `IsInt' returns true for that value.

*All* options (except `silent', `messfile', `outfile' and  `echo')  on
the `OptionsStack', when the {\ACE} interface is called via `CallACE',
are passed to the {\ACE} binary. Even options flagged as  \lq{}unknown
or possibly bad' (such warnings are issued when the `echo'  option  is
used) are  passed  to  the  {\ACE}  binary.  When  the  {\ACE}  binary
encounters an option that it doesn't understand it issues a warning to
`stderr' and simply ignores it;  so  options  accidentally  passed  to
{\ACE} are unlikely to pose problems. The options `silent', `messfile'
(which is translated to  `ao'  for  the  binary)  and  `outfile'  were
introduced for {\GAP} use; and `echo', though  an  {\ACE}  option,  is
actually handled by `CallACE' and not passed to the binary.

So  for example to  use the  `hard' strategy  option and  increase the
workspace to $10^7$ words for the example given above, one could call:

\begintt
gap> CallACE(fgens,rels,[c]:hard,workspace:=10^7);;
\endtt

Except for a few additional  option  synonyms  (e.g.~`messfile'  is  a
{\GAP}-introduced synonym for the `ao' {\ACE} option),  and  just  two
multiple-word {\ACE} options (the `pure r'  and  `pure  c'  strategies
have  become  `purer'  and  `purec',  respectively,  for  the   {\GAP}
interface), and  a  small  number  of  additional  non-{\ACE}  options
(e.g.~`silent' and `outfile'), all option names for the interface  are
the same as for the {\ACE} standalone.  The  {\ACE}  binary  has  many
multiple-word options but all  except  `pure  r'  and  `pure  c'  have
single-word  alternatives;  only  the  single-word  option  names  are
available via  the  {\GAP}  interface.  Some  abbreviations  that  are
possible for the {\ACE} binary (e.g.~`fi' for `fill',  and  `rec'  for
`recover') are *not* allowed via  the  {\GAP}  interface,  since  they
clash with {\GAP} keywords. Again, most  of  the  preceding  paragraph
will not be of importance to the {\GAP} user; these comments have been
made merely for the user who wishes to reconcile  what  the  interface
does with how the {\ACE} binary  performs,  and  is  of  most  use  to
standalone users.

%%%%%%%%%%%%%%%%%%%%%%%%%%%%%%%%%%%%%%%%%%%%%%%%%%%%%%%%%%%%%%%%%%%%%%%%%
\Section{General Options that Modify the Enumeration Process}

\beginitems
\>`asis'{Option asis}&
Do not reduce relators. (Shortest abbreviation: `as'.)

By default, {\ACE} freely  and cyclically reduces the relators, freely
reduces  the  subgroup generators,  and  sorts  relators and  subgroup
generators in length-increasing  order.  If you do not  want this, you
can switch it off by setting the `asis' option.

*Notes:* As well as allowing you  to use the presentation *as* it *is*
given,  this  is  useful for  forcing  definitions  to  be made  in  a
prespecified  order,  by  introducing  dummy  (i.e.,  freely  trivial)
subgroup generators.   (Note that the  exact form of  the presentation
can  have a significant  impact on  the enumeration  statistics.)  For
some fine points of the influence of `asis' being set on the treatment
of involutory generators see the {\ACE} standalone manual.

\>`ct:=<val>'{Option ct}
\>`cfactor:=<val>'{Option cfactor}&
Number of C-style definitions per pass; `<val>' should be an  integer. 
(Shortest abbreviation of `cfactor' is `c'.)

The absolute value of `<val>' sets the
number of C-style definitions per  pass through  the enumerator's main
loop. The sign of `<val>'  sets the  style. The possible  combinations
of the values of `ct' and  `rt'  (described below)  are  given in  the
table of  enumeration styles in Section~"Enumeration Style".

\>`rt:=<val>'{Option rt}
\>`rfactor:=<val>'{Option rfactor}&
Number of R-style definitions per pass; `<val>' should be an  integer. 
(Shortest abbreviation of `rfactor' is `r'.)

The absolute value of `<val>' sets the
number of R-style definitions per  pass through  the enumerator's main
loop. The sign of `<val>'  sets the  style. The possible  combinations
of the values of `ct' (described above)  and  `rt'  are  given in  the
table of  enumeration styles in Section~"Enumeration Style".

\>`no:=<val>'{Option number}&
The number of group relators to include in the subgroup;  
`<val>' should be an integer greater than or equal to $-1$.

It is sometimes helpful to include the group relators into the list of
the subgroup generators,  in the sense that they  are applied to coset
number  \#1 at the  start of  an enumeration.  A value  of 0  for this
option  turns this  feature off  and  the (default)  argument of  $-1$
includes all the relators.  A positive argument includes the specified
number of relators, in order. The `no' option affects only the C-style
procedures.

\>`mendelsohn'{Option mendelsohn}&
Turns on mendelsohn processing. (Shortest abbreviation: `mend'.)

Mendelsohn style processing during relator scanning/closing is  turned
on by giving this option. Off is the default, and  here  relators  are
scanned only from  the  start  (and  end)  of  a  relator.  Mendelsohn
\lq{}on' means that all (different) cyclic permutations of  a  relator
are scanned.

The effect  of Mendelsohn style  processing is case-specific.   It can
mean the difference  between success or failure, or  it can impact the
number  of  cosets   required,  or  it  can  have   no  effect  on  an
enumeration's statistics.

*Note:* Processing all cyclic permutations of the relators can be very
time-consuming,  especially if  the  presentation is  large.  So,  all
other things being equal, the  Mendelsohn flag should normally be left
off.

%%%%%%%%%%%%%%%%%%%%%%%%%%%%%%%%%%%%%%%%%%%%%%%%%%%%%%%%%%%%%%%%%%%%%
\Section{Options Modifying C-Style Definitions}

The  next  three  options  are  relevant  only  for   making   C-style
definitions. Making definitions in C-style, that is filling the  coset
table line by line, it can be very advantageous to  switch  to  making
definitions from the preferred definition stack. Possible  definitions
can be extracted from this stack in various ways and the  two  options
`pmode'  and  `psize'  (see~"Option   pmode"   and   ~"Option   psize"
respectively) regulate this. However it should be  clearly  understood
that making all definitions from a preferred definition stack one  may
violate the condition of Mendelsohn's theorem, and the  option  `fill'
(see~"Option fill") can be used to avoid this.

\>`fill:=<val>'{Option fill}
\>`ffactor:=<val>'{Option ffactor}&
Controls the preferred definition strategy by setting the fill factor;
`<val>' must be a non-negative integer.
(Shortest abbreviation of `fill' is `fil', and  shortest  abbreviation
of `ffactor' is `f'.)

Unless prevented by  the fill factor, gaps of  length one found during
deduction   testing  are  preferentially   filled  (see~\cite{Hav91}).
However,  this potentially  violates the  formal requirement  that all
rows in the coset table  are eventually filled (and tested against the
relators).   The fill  factor is  used  to ensure  that some  constant
proportion of the coset table  is always kept filled.  Before defining
a coset  number to  fill a  gap of length  one, the  enumerator checks
whether `fill' times  the completed part of the table  is at least the
total  size of  the  table and,  if  not, fills  coset  table rows  in
standard order (C-style) instead of filling gaps.

An  argument of  0  selects  the default  value  of $\lfloor  5(n+2)/4
\rfloor$,  where $n$  is the  number of  columns in  the  table.  This
default  fill factor  allows  a moderate  amount  of gap-filling.   If
`fill' is  1, then there is  no gap-filling.  A large  value of `fill'
can cause  what is in effect  infinite looping (resolved  by the coset
enumeration failing).   However, in general,  a large value  does work
well.  The  effects of the various gap-filling  strategies vary widely.
It is  not clear  which values are  good general defaults  or, indeed,
whether any strategy is always ``not too bad''.

\>`pmode:=<val>'{Option pmode}&
Option for preferred definitions;  `<val>' should be in the integer
range 0 to 3. (Shortest abbreviation: `pmod'.)

The  value of  the  `pmode' option  determines  which definitions  are
preferred.  If  the argument is  0, then Felsch style  definitions are
made using  the next empty table  slot.  If the  argument is non-zero,
then gaps of length one found during relator scans in Felsch style are
preferentially  filled  (subject to  the  value  of  `fill').  If  the
argument  is 1,  they are  filled  immediately, and  if it  is 2,  the
consequent deduction  is also made  immediately (of course,  these are
also put on the deduction stack).  If the argument is 3, then the gaps
of length one are noted in the preferred definition queue.

Provided such a gap survives (and no coincidence occurs, which  causes
the queue to be discarded) the next coset number will  be  defined  to
fill the oldest gap of length one. The default value is either 0 or 3,
depending on the strategy selected (see Section "Strategy Options  for
ACE"). If you want to know more details, read the code.


\>`psize:=<val>'{Option psize}&
Size of preferred definition queue; `<val>' *must*  be $2^n$, for some
integer $n>0$. (Shortest abbreviation: `psiz'.)

The  preferred definition  queue is  implemented as  a  ring, dropping
earliest entries. An argument of 0 selects  the default size of $256$.
Each  queue slot takes two words (i.e., 8 bytes),  and the  queue  can
store up to $2^n-1$ entries.

%%%%%%%%%%%%%%%%%%%%%%%%%%%%%%%%%%%%%%%%%%%%%%%%%%%%%%%%%%%%%%%%%%%%%
\Section{Options for R-Style Definitions}

\>`row:=<val>'{Option row}&
Set the \lq{}row filling' option; `<val>' is either 0 or 1.

By default, \lq{}row filling' is on (i.e.~`true' or 1). To turn it off
set `row' to either `false' or 0 (both are translated to 0 when passed
to  the {\ACE}  binary).   When making  HLT-style  definitions, it  is
normal to scan each row of  the coset table after its coset number has
been applied to  all relators, and make definitions  to fill any holes
in that row of the  coset table encountered. This will, in particular,
guarantee  that   the  condition  of  Mendelsohn's   Theorem  will  be
fulfilled.  Failure  to do  so can cause  even simple  enumerations to
overflow.

\>`lookahead:=<val>'{Option lookahead}&
Lookahead; `<val>' should be in the integer range 0 to 4.
(Shortest abbreviation: `look'.)
  
Although HLT-style strategies  are fast, they are local,  in the sense
that  the  implications   of  any  definitions/deductions  made  while
applying cosets may not become  apparent until much later.  One way to
alleviate this problem is to perform lookaheads occasionally; that is,
to  test the  information  in  the table,  looking  for deductions  or
concidences.  {\ACE} can perform a lookahead when the table overflows,
before the compaction routine is  called.  Lookahead can be done using
the entire  table or  only that  part of the  table above  the current
coset, and  it can  be done R-style  (scanning coset numbers  from the
beginning  of relators)  or C-style  (testing all  definitions  in all
essentially different positions).

The following are the effects of the possible values of `lookahead':

\beginlist

\item{--} A value of 0 disables lookahead;
\item{--} 1 does a partial table lookahead, R-style; 
\item{--} 2 does a whole table lookahead, C-style; 
\item{--} 3 does a whole table lookahead, R-style; 
\item{--} 4 does a partial table lookahead, C-style.  

\endlist

The default is 1 if the `hlt' strategy is used and  0  otherwise;  see
Section~"Strategy Options for ACE".

%%%%%%%%%%%%%%%%%%%%%%%%%%%%%%%%%%%%%%%%%%%%%%%%%%%%%%%%%%%%%%%%%%%%%%
\Section{Options for Deduction Handling}

\>`dmode:=<val>'{Option dmode}&
Deduction mode; `<val>' should be in the integer range 0 to 4.
(Shortest abbreviation: `dmod'.)

A completed table  is only valid if every table  entry has been tested
in all essentially different  positions in all relators.  This testing
can either be done directly  (Felsch strategy) or via relator scanning
(HLT strategy).  If it is  done directly, then more than one deduction
(i.e., table  entry) can be waiting  to be processed at  any one time.
So the untested deductions are stored in a stack.  Normally this stack
is fairly small but, during a collapse, it can become very large.

This command allows the user  to  specify  how  deductions  should  be
handled. The value <val> has the following interpretations:

\beginlist

\item{--} $0$:  
discard deductions if there is no stack space left;

\item{--} $1$: 
as for $0$, but purge any redundant coset numbers on the  top  of  the
stack at every coincidence;

\item{--} $2$: 
as for 0, but purge all redundant coset  numbers  from  the  stack  at
every coincidence;

\item{--} $3$:
discard the entire stack if it overflows; and

\item{--} $4$:
if the stack overflows, double the stack size and purge all  redundant
coset numbers from the stack.

\endlist

The default deduction mode is either $0$  or  $4$,  depending  on  the
strategy selected (see Section~"Strategy Options for ACE"), and it  is
recommended that you stay with the default. If you want to  know  more
details, read the well-commented C code.

*Notes:*
If deductions are discarded for any reason, then a final relator check
phase  will be run  automatically at  the end  of the  enumeration, if
necessary, to check the result.

\>`dsize:=<val>'{Option dsize}&
Deduction stack size; `<val>' should be a non-negative integer.
(Shortest abbreviation: `dsiz'.)

Sets the  size of  the (initial) allocation  for the  deduction stack.
The size is  in terms of the number of  deductions, with one deduction
taking two words (i.e., 8 bytes).  The default size, of $1000$, can be
selected  by  a value  of  0.   See the  `dmode' entry  for a  (brief)
discussion of deduction handling.

\enditems

%%%%%%%%%%%%%%%%%%%%%%%%%%%%%%%%%%%%%%%%%%%%%%%%%%%%%%%%%%%%%%%%%%%%%%%%%
\Section{Technical Options}

The following options do not affect how the coset enumeration is done,
but how it  uses the computer's resources. They  might thus affect the
runtime as  well as  the range of  problems that  can be tackled  on a
given machine.

\beginitems

\>`workspace:=<val>'{Option workspace}&
Workspace size in words (default $10^6$);
`<val>' should be an expression that evaluates to a positive  integer,
or a string of digits ending in a  `k',  `M'  or  `G'  representing  a
multiplication  factor  of  $10^3$,  $10^6$  or  $10^9$,  respectively
e.g.~both `workspace := 2 * 10^6' and `workspace :=  "2M"'  specify  a
workspace  of  $2\times10^6$  words.  Actually,  if   the   value   of
`workspace' is entered as a string, each of `k', `M' or  `G'  will  be
accepted in either upper or lower case. (Shortest abbreviation: `wo'.)

By default, {\ACE} has a physical table size of $10^6$ words (i.e., $4
\times 10^6$ bytes in the  default 32-bit environment).  The number of
coset numbers in the table is  the table size divided by the number of
columns.   Although  the  number   of  coset  numbers  is  limited  to
$2^{31}-1$ (if the C `int' type is 32 bits), the table size can exceed
the $4$GByte 32-bit limit if a suitable machine is used.

\>`time:=<val>'{Option time}&
Maximum execution time in seconds; `<val>' must be an integer  greater
than or equal to $-1$. (Shortest abbreviation: `ti'.)

The `time' command  puts a time limit (in seconds) on  the length of a
run. The default is $-1$  which is no  time limit. If the  argument is
$\ge0$ then the total elapsed time for this call is checked at the end
of each pass through the enumerator's main loop, and if it's more than
the limit the run is stopped and the current table returned.

Note that a limit of $0$ performs exactly one pass  through  the  main
loop, since $0 \ge 0$.

%\begintt
%???????????????????????????????????????????????????????????????????
%*GREG:
%
%DOES THIS HAVE RELEVANCE TO THE INTERACTIVE MODE?*
%
%If the enumerator is run in the continue mode, this allows a form of
%  ``single-stepping''
%
%??????????????????????????????????????????????????????????????????
%\endtt

The time  limit is approximate, in  the sense that  the enumerator may
run for a longer, but never a shorter, time.  So, if there is, e.g., a
big collapse (so that the time round the loop becomes very long), then
the run may run over the limit by a large amount.

*Notes:*
The time  limit is  CPU-time, not wall-time.   As in all  timing under
Unix, the clock's granularity (usually  $10$ mSec) and the system load
can affect  the timing;  so the  number of main  loop iterations  in a
given time may vary.

\>`loop:=<val>'{Option loop}&
Loop limit; `<val>' should be a non-negative integer.

The core enumerator is organised as a state machine,  with  each  step
performing an \lq{}action' (i.e., lookahead, compaction) or a block of
actions   (i.e.,   $|`ct'|$   coset   definitions,   $|`rt'|$    coset
applications). The number of  passes  through  the  main  loop  (i.e.,
steps) is counted, and the enumerator can make an  early  return  when
this count hits the value of `loop'. A  value  of  $0$,  the  default,
turns this feature off.

*Guru Notes:*
You can do lots of really neat things using this feature, but you need
some understanding of the internals of {\ACE} to get real benefit from
it.

\>`path'{Option path}&
Turns on path compression.

To correctly  process  multiple  concidences,  a  union-find  must  be
performed. If both path compression and weighted union are used,  then
this can be done in essentially linear time (see, e.g., \cite{CLR90}).
Weighted union alone, in the worst-case, is worse than linear, but  is
subquadratic. In practice, path compression  is  expensive,  since  it
involves many coset table accesses. So, by default,  path  compression
is turned off; it can be turned on by `path'. It has no effect on  the
result, but may affect the running time and the internal statistics.

*Guru Notes:*
The whole question of the best way to handle large coincidence forests
is problematic.  Formally, {\ACE} does  not do a weighted union, since
it is constrained to replace the higher-numbered of a coincident pair.
In practice,  this seems  to amount to  much the same  thing!  Turning
path  compression on  cuts down  the  amount of  data movement  during
coincidence processing at the expense of having to trace the paths and
compress them.  In general, it does not seem to be worthwhile.

\>`compaction:=<val>'{Option compaction}&
Percentage of dead cosets to trigger compaction;
`<val>' should be an integer (percentage) in the integer range 0 to 100.
(Shortest abbreviation: `com'.)

The option `compaction'  sets  the  percentage  of  *dead*~\index{dead
coset (number)} coset numbers needed  to  trigger  compaction  of  the
coset table, during an enumeration. A *dead* coset (number) is a coset
found to be coincident with a lower numbered coset. The default is  10
or 100, depending on the strategy used (see Section~"Strategy  Options
for ACE").

Compaction recovers the space allocated to  coset  numbers  which  are
flagged as dead. It results in a table where all the active cosets are
numbered contiguously from \#1, and with the remainder  of  the  table
available for new cosets.

The coset table is compacted when a definition of a  coset  number  is
required, there is no space for a  new  coset  number  available,  and
provided that the given percentage of the coset  table  contains  dead
coset numbers. For example, if `compaction'  =  $20$  then  compaction
will occur only if 20\% or more of the cosets in the table  are  dead.
An argument of 100 means that compaction is never performed, while  an
argument of 0 means always compact,  no  matter  how  few  dead  coset
numbers there are (provided there is at least one, of course).

Compaction may be performed  multiple times during an enumeration, and
the table that results from an  enumeration may or may not be compact,
depending on whether or not there have been any coincidences since the
last compaction (or  from the start of the  enumeration, if there have
been no compactions).

*Notes:*
In some strategies (e.g., `hlt') a lookahead phase may be  run  before
compaction is attempted. In  other  strategies  (e.g.,  `sims  :=  3')
compaction may be performed while there  are  outstanding  deductions;
since deductions are discarded during compaction,  a  final  lookahead
phase will (automatically) be performed.

Compacting a table \lq{}destroys'  information  and  history,  in  the
sense that the coincidence list is deleted, and the table entries  for
any dead cosets are deleted.

%??????????????????????????????????????????????????????????????????????
%
%*GREG:
%WILL THIS BE RELEVANT FOR INTERACTIVE USE?*
%
%If messaging is enabled 
%(see~"Options controlling Input and Output"),
%then a progress  message (labelled {\tt CO}) is  printed each time the
%compaction routine is  actually called (as opposed to  each time it is
%potentially called).
%
%??????????????????????????????????????????????????????????????????????

\>`max:=<val>'{Option max}&
Maximum number of cosets to be defined;
`<val>' should be $0$ or an integer greater than or equal to 2.

Sets the maximum  number of coset numbers to  be defined.  By default,
all  of the workspace  is used,  if necessary,  in building  the coset
table.  So the table size is  an upper bound on how many coset numbers
can be active at any one time.   The `max' option allows a limit to be
placed on  how much of the  physical table space is  made available to
the enumerator.   Enough space for  at least two coset  numbers (i.e.,
the subgroup and one other) must  be made available.  An argument of 0
selects all of the workspace.

\enditems

%%%%%%%%%%%%%%%%%%%%%%%%%%%%%%%%%%%%%%%%%%%%%%%%%%%%%%%%%%%%%%%%%%%%%%

\Section{Options controlling Input and Output}

\beginitems

\>`silent'{Option silent}& 
Inhibits an error return.

If the coset enumeration does not finish within the preset limits,  an
error is raised by the interface to {\GAP}, unless the option `silent'
has been set, in which case `fail' is returned. Note that `silent'  is
a feature of the {\GAP} interface only, it has no counterpart  in  the
{\ACE} standalone.

\>`ACEinfile:=<filename>'{Option ACEinfile}&
Creates an {\ACE} input file <filename> for use with  the  standalone,
only; <filename> should be a string.

If this option is used, {\GAP} creates an  input  file  with  filename
<filename> only,  and  then  exits  (i.e.~the  {\ACE}  binary  is  not
called). This option is provided for users who wish to  work  directly
(and interactively) with the {\ACE} standalone; and is, of  course,  a
feature of the interface only, that has no counterpart in  the  {\ACE}
standalone. The full path to the input file normally  used  by  {\ACE}
(i.e.~when  option   `ACEinfile'   is   not   used)   is   stored   in
`ACEinfo.infile'.   This   option   used   to    be    performed    by
`outfile'\index{Option outfile (deprecated: use ACEinfile)}, which was
deemed confusing and so is now deprecated.

\>`messages:=<val>'{Option messages}
\>`monitor:=<val>'{Option monitor}&
Sets the verbosity of output; <val> should be an integer.
(Shortest  abbreviation  of  `messages'  is   `mess',   and   shortest
abbreviation of `monitor' is `mon'.)

By default, <val> = 0, for which {\ACE} prints out only a single  line
of information, giving the result of each  enumeration.  If  <val>  is
non-zero then the presentation and the parameters are  echoed  at  the
start of the run, and messages  on  the  enumeration's  status  as  it
progresses are also printed out. The absolute value of <val> sets  the
frequency of the progress messages, with a negative sign turning  hole
monitoring on. Note that, hole monitoring is expensive, so don't  turn
it on unless you really need it.  (See  option  `ACEoutfile'  and  the
discussion below on the meanings of {\ACE}'s output messages,  if  you
intend to peruse all this output.)

\>`ao:=<filename>'{Option ao}
\>`ACEoutfile:=<filename>'{Option ACEoutfile}&
Redirects (`a'lters) `o'utput to <filename>; <filename>  should  be  a
string.

Normally {\ACE}'s output is directed to a temporary  file  whose  full
path is stored in `ACEinfo.outfile', which  is  parsed  to  produce  a
coset table or a list  of  statistics.  This  option  causes  {\ACE}'s
output to be directed to <filename> instead,  presumably  because  the
user wishes to see (and keep) data output by the {\ACE} binary,  other
than the coset table output from `CallACE' or the statistics output by
`ACEStats'. Below we discuss the meaning of the additional data to  be
found in the {\ACE} binary's output.  The  option  `ACEoutfile'  is  a
{\GAP}-introduced synonym for `ao', that is translated to `ao'  before
submission to the {\ACE} binary. Do not use option  `ACEoutfile'  when
running    the    standalone    directly.    An     older     synonym,
`messfile'\index{Option  messfile  (deprecated:  use  ACEoutfile)}  is
deprecated. Happily, `ao' can  also  be  regarded  as  mnemonical  for
`ACEoutfile'.

\enditems

*The meanings of {\ACE}'s output messages*

For the rest of this section we discuss the meanings of  the  messages
that appear in the output of the {\ACE} binary (which is held  in  the
file `ACEinfo.outfile', or `ACEoutfile' (or  `ao')  if  the  user  has
chosen to use this option). Note that the banner stating  the  version
number and date, which will be observed if one runs the standalone  is
re-directed to the file `ACEinfo.banner'.

What is first observed in the {\ACE} output file is a heading like:

\begintt
  #-- ACE 3.000: Run Parameters ---
\endtt

(where `3.000' may be replaced be some later version number)  followed
by the \lq{}input parameters' developed from the arguments and options
passed to `CallACE' or `ACEStats'.  After these appears a separator:

\begintt
  #--------------------------------
\endtt

followed by  any  *progress  messages*  (progress  messages  are  only
printed if `messages' is non-zero; recall that by default `messages' =
0), followed by a *results message*. In the case of  `CallACE',  these
messages are followed by yet more progress messages (if `messages'  is
non-zero) and a coset  table.  Finally,  the  {\ACE}  output  file  is
terminated by {\ACE}'s exit banner, which should look something like:

\begintt
=========================================
ACE 3.000        Sun Mar 12 17:25:37 2000
\endtt

Both *progress messages* and  the  *results  message*  consist  of  an
initial tag followed by a list of statistics. All messages have values
for the statistics `a', `r', `h', `n', `h',  `l'  and  `c'  (excepting
that the second `h', the one following  the  `n'  statistic,  is  only
given if hole monitoring has been turned on by setting `messages'  $\<
0$, which as noted above is expensive and  should  be  avoided  unless
really needed). Additionally, there may appear the statistics: `m' and
`t' (as for the results message); `d'; or `s', `d' and `c' (as for the
`DS' progress message). The meanings of  the  various  statistics  and
tags will follow later. The following is a sample progress message:

\begintt
AD: a=2 r=1 h=1 n=3; h=66.67% l=1 c=+0.00; m=2 t=2
\endtt

with tag `AD' and values for the statistics `a', `r',  `h',  `n',  `h'
(appears because `messages' $\<  0$),  `l',  `c',  `m'  and  `t'.  The
following is a sample results message:

\begintt
INDEX = 12 (a=12 r=16 h=1 n=16; l=3 c=0.01; m=14 t=15)
\endtt

which, in this case, declares a successful enumeration of  the  cosets
of a subgroup of index 12 within a group, and, as it turns out, values
for the same statistics as the sample progress message.

In the following table we  list  the  statistics  that  can  follow  a
progress or results message tag, in order:

\begintt
--------------------------------------------------------------------
statistic   meaning
--------------------------------------------------------------------
a           number of active cosets
r           number of applied cosets
h           first (potentially) incomplete row
n           next coset definition number
h           percentage of holes in the table (if `messages'$ \< 0$) 
l           number of main loop passes
c           total CPU time
m           maximum number of active cosets
t           total number of cosets defined
s           new deduction stack size (with DS tag)
d           current deduction stack size, or
              no. of non-redundant deductions retained (with DS tag)
c           no. of redundant deductions discarded (with DS tag)
--------------------------------------------------------------------
\endtt

Now that we have discussed the various  meanings  of  the  statistics,
it's time to discuss the various types of progress messages possible.

*Progress Messages*

A progress message (and its tag) indicates the function just completed
by the enumerator. In the following table, the possible message `tag's
appear in the first column. In the `action' column,  a  `y'  indicates
the function is aggregated and counted. Every time this count  reaches
the value of `messages', a message line is printed and  the  count  is
zeroed. Those tags flagged  with  a  `y*'  are  only  present  if  the
appropriate option was included when the {\ACE} binary was compiled (a
default compilation includes the appropriate options; so normally read
`y*' as `y').

Tags with an `n' in the `action' column indicate the function  is  not
counted, and cause a message line to be output every time they  occur.
They also cause the action count to be reset.

\begintt
------------------------------------------------------------------
tag   action      meaning
------------------------------------------------------------------
AD         y      coset 1 application definition (`SG'/`RS' phase)
RD         y      R-style definition
RF         y      row-filling definition
CG         y      immediate gap-filling definition
CC         y*     coincidence processed
DD         y*     deduction processed
CP         y      preferred list gap-filling definition
CD         y      C-style definition
Lx         n      lookahead performed (type `x')
CO         n      table compacted
CL         n      complete lookahead (table as deduction stack)
UH         n      updated completed-row counter
RA         n      remaining cosets applied to relators
SG         n      subgroup generator phase
RS         n      relators in subgroup phase
DS         n      stack overflowed (compacted and doubled)
------------------------------------------------------------------
\endtt

% \begin{table}
% \hrule
% \caption{Possible progress messages}
% \label{tab:prog}
% \smallskip
% \renewcommand{\arraystretch}{0.875}
% \begin{tabular*}{\textwidth}{@{\extracolsep{\fill}}lll} 
% \hline\hline
% message & action & meaning \\
% \hline
% \ttt{AD} & y  & coset \#1 application definition 
% 			(\ttt{SG}/\ttt{RS} phase) \\
% \ttt{RD} & y  & R-style definition \\
% \ttt{RF} & y  & row-filling definition \\
% \ttt{CG} & y  & immediate gap-filling definition \\
% \ttt{CC} & y* & coincidence processed \\
% \ttt{DD} & y* & deduction processed \\
% \ttt{CP} & y  & preferred list gap-filling definition \\
% \ttt{CD} & y  & C-style definition \\
% \ttt{Lx} & n  & lookahead performed (type \ttt{x}) \\
% \ttt{CO} & n  & table compacted \\
% \ttt{CL} & n  & complete lookahead (table as deduction stack) \\
% \ttt{UH} & n  & updated completed-row counter \\
% \ttt{RA} & n  & remaining cosets applied to relators \\
% \ttt{SG} & n  & subgroup generator phase \\
% \ttt{RS} & n  & relators in subgroup phase \\
% \ttt{DS} & n  & stack overflowed (compacted and doubled) \\
% \hline\hline
% \end{tabular*}
% \end{table}

*Results Messages*

The possible results are given in the following table; any result  not
listed represents an internal error and  should  be  reported  to  the
{\ACE} authors.

% The level column is omitted ... since it won't mean anything to a
% GAP user
\begintt
result tag           meaning 
------------------------------------------------------------------
INDEX = x            finite index of `x' obtained
OVERFLOW             out of table space
SG PHASE OVERFLOW    out of space (processing subgroup generators)
ITERATION LIMIT      `loop' limit triggered
TIME LIMT            `ti' limit triggered
HOLE LIMIT           `ho' limit triggered
INCOMPLETE TABLE     all cosets applied, but table has holes
MEMORY PROBLEM       out of memory (building data structures)
---------------------------------------------------------------------
\endtt

% \begin{table}
% \hrule
% \caption{Possible enumeration results}
% \label{tab:rslts}
% \smallskip
% \renewcommand{\arraystretch}{0.875}
% \begin{tabular*}{\textwidth}{@{\extracolsep{\fill}}lll} 
% \hline\hline
% result & level & meaning \\
% \hline
% \ttt{INDEX = x}         & 0 & finite index of \ttt{x} obtained \\
% \ttt{OVERFLOW}          & 0 & out of table space \\
% \ttt{SG PHASE OVERFLOW} & 0 & out of space (processing subgroup
% 				generators) \\
% \ttt{ITERATION LIMIT}   & 0 & \ttt{loop} limit triggered \\
% \ttt{TIME LIMT}         & 0 & \ttt{ti} limit triggered \\
% \ttt{HOLE LIMIT}        & 0 & \ttt{ho} limit triggered \\
% \ttt{INCOMPLETE TABLE}  & 0 & all cosets applied, but table has holes \\
% \ttt{MEMORY PROBLEM}    & 1 & out of memory (building data structures) \\
% \hline\hline
% \end{tabular*}
% \end{table}

*Notes*

Hole monitoring is expensive, so don't turn it on  unless  you  really
need it. If you wish to print out the presentation  and  the  options,
but not the progress messages, then set `messages' non-zero, but  very
large. (You'll still get the `SG', `DS', etc. messages,  but  not  the
`RD', `DD', etc. ones.) You can set `messages' to $1$, to monitor  all
enumerator actions, but be warned  that  this  can  yield  very  large
output files.

%%%%%%%%%%%%%%%%%%%%%%%%%%%%%%%%%%%%%%%%%%%%%%%%%%%%%%%%%%%%%%%%%%%%%%
\Section{Options for Experimentation}

With each of the options in this section you will probably want to set
the option `ACEoutfile' (see~"Option ACEoutfile").  In  fact,  if  you
haven't already done so,  read  the  entire  previous  section.

\beginitems

\>`aep:=<val>'{Option aep}&
Runs  the enumeration for `a'll `e'quivalent `p'resentations;
<val> is in the integer range 0 to 7.

\enditems

The `aep' option runs  an  enumeration  for  combinations  of  relator
ordering, relator rotations, and relator inversions.

The argument <val> is considered as a binary number.  Its  three  bits
are treated as flags, and control relator rotations (the  $2^0$  bit),
relator inversions (the $2^1$ bit) and relator  orderings  (the  $2^2$
bit),  respectively;  where  $1$  means  \lq{}active'  and  $0$  means
\lq{}inactive'. (See below for an example).

The `aep' option first performs a \lq{}priming run' using the  options
as they stand. In particular, the `asis' and  `messages'  options  are
honoured.

It then turns `asis' on and `messages' off  (i.e.~sets  `messages'  to
0), and generates and tests the  requested  equivalent  presentations.
The maximum and minimum values attained by `m' (the maximum number  of
coset numbers defines at any stage) and `t' (the total number of coset
numbers defined) are tracked, and each  time  a  new  \lq{}record'  is
found, the relators used and the summary result line is  printed.  See
the previous section for a discussion of the statistics `m'  and  `t'.
To observe these messages you will probably want  to  set  the  option
`ACEoutfile' (see~"Option ACEoutfile").

The order in which the  equivalent  presentations  are  generated  and
tested has no particular significance, but note that the  presentation
as given *after* the initial priming run) is the  *last*  presentation
to be generated and tested, so that  the  group's  relators  are  left
`unchanged' by running the `aep' option, (not that  a  non-interactive
user cares).

As discussed by Cannon, Dimino, Havas  and  Watson  \cite{CDHW73}  and
Havas and Ramsay \cite{HR99b} such equivalent presentations can  yield
large variations in the number of cosets required in  an  enumeration.
For this command, we are interested in this variation.

After  the  final  presentation  is  run,   some   additional   status
information messages are printed to the {\ACE} output file:

\beginlist
\item{--}  the number of runs which yielded a finite index; 
\item{--}  the total number of runs (excluding the priming run); and 
\item{--}  the range of values observed for `m' and `t'.
\endlist

As an example (drawn from the discussion in \cite{HR99a}) consider the
enumeration of the  $448$  cosets  of  the  subgroup  $\langle  a^2,Ab
\rangle$ of the group
$$ (8,7 \mid 2,3) 
    = \langle a,b \mid a^8 = b^7 = (ab)^2 = (Ab)^3 = 1 \rangle. $$
There are $4!=24$  relator  orderings  and  $2^4=16$  combinations  of
relator or inverted relator. Exponents are  taken  into  account  when
rotating relators, so the relators given give rise to 1, 1,  2  and  2
rotations respectively, for a total of $1.1.2.2=4$  combinations.  So,
for  `aep'  =  $7$   (resp.~$3$),   $24.16.4=1536$   (resp.~$16.4=64$)
equivalent presentations are tested.

*Notes:*
There is no way to stop the `aep'  option  before  it  has  completed,
other than killing the task. So do a reality check beforehand  on  the
size of the search space and the time for each enumeration. If you are
interested in  finding  a  \lq{}good'  enumeration,  it  can  be  very
helpful, in terms of running time, to put a tight limit on the  number
of cosets via the `max' option. You may also have to set  `compaction'
=  $100$  to  prevent  time-wasting  attempts  to  recover  space  via
compaction.  This  maximises  throughput  by  causing  the   \lq{}bad'
enumerations, which are in  the  majority,  to  overflow  quickly  and
abort. If you wish to explore  a  very  large  search-space,  consider
firing up many copies of {\ACE}, and starting each with a \lq{}random'
equivalent  presentation.  Alternatively,  you  could  use  the  `rep'
command.

\beginitems

\>`rep:=<val>'{Option rep}
\>`rep:=[<val>, <Npresentations>]'{Option rep}&
Runs  the enumeration for `r'andom `e'quivalent `p'resentations;
<val> is in the integer range 0 to 7;
<Npresentations> must be a positive integer.

\enditems

The `rep' (random equivalent  presentations)  option  complements  the
`aep'  option.  It  generates  and  tests   some   random   equivalent
presentations. The argument <val>  acts  as  for  `aep'.  It  is  also
possible to set the number <Npresentations>  of  random  presentations
used (by default, eight  are  used),  by  using  the  extended  syntax
`rep:=[<val>,<Npresentations>]'.

The routine first  turns  `asis'  on  and  `messages'  off  (i.e.~sets
`messages' to 0), and then generates and tests the requested number of
random equivalent presentations. For each  presentation, the  relators
used and the  summary  result  line  are  printed.  To  observe  these
messages you  will  probably  want  to  set  the  option  `ACEoutfile'
(see~"Option ACEoutfile").

*Notes:*
The relator inversions and rotations are \lq{}genuinely'  random.  The
relator permuting is a little bit of a kludge, with the  \lq{}quality'
of the permutations tending to improve with successive  presentations.
When the `rep' command  completes,  the  presentation  active  is  the
*last* one generated, (not that the non-interactive user cares).

*Guru Notes:*
It might appear that neglecting to restore the  original  presentation
is an error. In fact, it is a useful feature! Suppose that  the  space
of equivalent presentations is too  large  to  exhaustively  test.  As
noted in the entry for `aep', we can start up multiple copies of `aep'
at random points in the  search-space.  Manually  generating  `random'
equivalent presentations to serve as starting-points  is  tedious  and
error-prone. The `rep' option provides a simple solution;  simply  run
`rep := 7' before `aep := 7'.

%%%%%%%%%%%%%%%%%%%%%%%%%%%%%%%%%%%%%%%%%%%%%%%%%%%%%%%%%%%%%%%%%%%%%%%%%
\Section{Other Options}

*YET ANOTHER SECTION TO BE WRITTEN*

Here we will list all the known {\ACE} options that ordinarily  should
not be used via `CallACE' or `ACEStats'.

%%%%%%%%%%%%%%%%%%%%%%%%%%%%%%%%%%%%%%%%%%%%%%%%%%%%%%%%%%%%%%%%%%%%%%%%%
\Chapter{Strategy Options for ACE}

It can be difficult to select appropriate options when presented  with
a new enumeration. The problem is  compounded  by  the  fact  that  no
generally applicable rules exist to  predict,  given  a  presentation,
which option settings are \lq{}good'. To help overcome  this  problem,
{\ACE} contains various commands which select  particular  enumeration
strategies. One or other of these strategies may work and, if not, the
results may indicate how  the  options  can  be  varied  to  obtain  a
successful enumeration.

Strategies are merely options that set a number of the options seen in
the previous chapter, all at  once;  those  provided  are:  `default',
`easy', `felsch', `hard', `hlt', `purec', `purer', and `sims'. All  of
these can be passed as boolean  options  from  the  {\ACE}  interface,
except `sims' which expects one of the integer values: 1, 3, 5, 7,  or
9. Also, `felsch' can accept a value of 0 or 1, where 0 has  the  same
effect as passing `felsch' with no value.

If no options are passed to {\ACE}, the `default' strategy is assumed,
which starts out presuming that the enumeration will be easy,  and  if
it turns out not to be so, {\ACE} switches to a strategy designed  for
more difficult enumerations. The  other  straightforward  options  for
beginning users are `easy'  and  `hard'.  Thus,  `easy'  will  quickly
succeed or fail (in the context of the given resources); `default' may
succeed quickly, or if not will try the `hard'  strategy;  and  `hard'
will run more slowly, from the beginning trying to succeed.

Strategy options are  parsed  in  *exactly*  the  same  way  as  other
options; order *is* important. It is usual  to  specify  one  strategy
option and possibly follow it with a number of options defined in  the
previous chapter, possibly modifying some of the options  set  by  the
strategy option. Please refer to  the  introduction  of  the  previous
chapter and to Section~"Using ACE as a Default for Coset  Enumerations
and some General Warnings" for more  information  and  warnings  about
using options.

In the table below, we list the thirteen standard strategies,  showing
the options (of Chapter~"Options for ACE")  that  they
set, except that all  of  them  additionally  set  `path'  =  `false',
`psize' = $256$, and `dsize' = $1000$. Recall that `mend', `look'  and
`com' are abbreviations for  `mendelsohn'  (see~"Option  mendelsohn"),
`lookahead' (see~"Option  lookahead")  and  `compaction'  (see~"Option
compaction"), respectively.

% \begin{table}
% \hrule
% \caption{The Predefined Strategies}
% \label{tab:pred}
% \smallskip
% \renewcommand{\arraystretch}{0.875}
% \begin{tabular*}{\textwidth}{@{\extracolsep{\fill}}lrrrrrrrrrrrrr} 
% \hline\hline
%           & \multicolumn{13}{c}{parameter} \\ 
% \cline{2-14}
% strategy & path & row & mend & no & look & com & ct   & rt    & fill &
% pmode & psize & dmode & dsize \\ 
% \hline
% default    & 0   & 1   & 0    & -1 & 0    & 10  & 0    & 0     & 0  & 3    & 256  & 4    & 1000 \\
% easy   & 0   & 1   & 0    & 0  & 0    & 100 & 0    & 1000  & 1  & 0    & 256  & 0    & 1000 \\
% felsch:0& 0   & 0   & 0    & 0  & 0    & 10  & 1000 & 0     & 1  & 0    & 256  & 4    & 1000 \\
% felsch:1  & 0   & 0   & 0    & -1 & 0    & 10  & 1000 & 0     & 0  & 3    & 256  & 4    & 1000 \\
% hard   & 0   & 1   & 0    & -1 & 0    & 10  & 1000 & 1     & 0  & 3    & 256  & 4    & 1000 \\
% hlt    & 0   & 1   & 0    & 0  & 1    & 10  & 0    & 1000  & 1  & 0    & 256  & 0    & 1000 \\
% purec & 0   & 0   & 0    & 0  & 0    & 100 & 1000 & 0     & 1  & 0    & 256  & 4    & 1000 \\
% purer & 0   & 0   & 0    & 0  & 0    & 100 & 0    & 1000  & 1  & 0    & 256  & 0    & 1000 \\
% sims:1 & 0   & 1   & 0    & 0  & 0    & 10  & 0    & 1000  & 1  & 0    & 256  & 0    & 1000 \\
% sims:3 & 0   & 1   & 0    & 0  & 0    & 10  & 0    & -1000 & 1  & 0    & 256  & 4    & 1000 \\
% sims:5 & 0   & 1   & 1    & 0  & 0    & 10  & 0    & 1000  & 1  & 0    & 256  & 0    & 1000 \\
% sims:7 & 0   & 1   & 1    & 0  & 0    & 10  & 0    & -1000 & 1  & 0    & 256  & 4    & 1000 \\
% sims:9 & 0   & 0   & 0    & 0  & 0    & 10  & 1000 & 0     & 1  & 0    & 256  & 4    & 1000 \\
% \hline\hline
% \end{tabular*}
% \end{table}

\begintt
                                option
            ----------------------------------------------------------
strategy    row   mend  no  look  com    ct     rt  fill  pmode  dmode
----------------------------------------------------------------------
default       1  false  -1     0   10     0      0     0      3      4
easy          1  false   0     0  100     0   1000     1      0      0
felsch := 0   0  false   0     0   10  1000      0     1      0      4
felsch := 1   0  false  -1     0   10  1000      0     0      3      4
hard          1  false  -1     0   10  1000      1     0      3      4
hlt           1  false   0     1   10     0   1000     1      0      0
purec         0  false   0     0  100  1000      0     1      0      4
purer         0  false   0     0  100     0   1000     1      0      0
sims := 1     1  false   0     0   10     0   1000     1      0      0
sims := 3     1  false   0     0   10     0  -1000     1      0      4
sims := 5     1   true   0     0   10     0   1000     1      0      0
sims := 7     1   true   0     0   10     0  -1000     1      0      4
sims := 9     0  false   0     0   10  1000      0     1      0      4
----------------------------------------------------------------------
\endtt

Note that we explicitly (re)set all of the listed  enumerator  options
in all of the predefined strategies, even although some of  them  have
no effect. For example, the `fill' value is irrelevant  in  HLT  mode.
The idea behind this  is  that,  if  you  later  change  some  options
individually, then the enumeration retains the `flavour' of  the  last
selected predefined strategy.

Note also that other options which may effect an enumeration are  left
untouched by setting one of the predefined  strategies;  for  example,
the values of `max' (see~"Option max") and `asis' (see~"Option asis").
These options have an effect which  is  independant  of  the  selected
strategy.

Note that, apart from the `felsch := 0' and `sims  :=  9'  strategies,
all of the strategies are distinct, although some are very similar.

%%%%%%%%%%%%%%%%%%%%%%%%%%%%%%%%%%%%%%%%%%%%%%%%%%%%%%%%%%%%%%%%%%%%%%
\Section{The Strategies in Detail}

\beginitems

\>`default'{Option default}&
Selects the default strategy. (Shortest abbreviation: `def'.)

This   strategy   is   based   on   the   defaulted   R/C-style;   see
Section~"Enumeration Style". The idea here is that we assume that  the
enumeration is \lq{}easy' and start out in R-style. If  it  turns  out
not to be easy, then  we  regard  it  as  \lq{}hard',  and  switch  to
CR-style, after performing a lookahead on the entire table.

\>`easy'{Option easy}&
Selects an \lq{}easy' R-style strategy.

If this strategy is selected, we run in R-style (i.e., HLT), but  turn
`lookahead' and `compaction' off. For small and/or easy  enumerations,
this mode is likely to be the fastest.

\>`felsch'{Option felsch}
\>`felsch:=<val>'{Option felsch}&
Selects a Felsch strategy; <val> should be 0 or 1. 
(Shortest abbreviation: `fel'.)

Assigning `felsch' 0 or no value selects a pure Felsch strategy, and a
value of 1 selects a Felsch strategy with all relators in the subgroup
(`no = -1') and turns gap-`fill'ing on.

\>`hard'{Option hard}&  
Selects a mixed R-style and C-style strategy.

In many \lq{}hard' enumerations, a  mixture  of  R-style  and  C-style
definitions, all tested in all  essentially  different  positions,  is
appropriate. This option selects such a mixed strategy. The idea  here
is that most of the work is done C-style (with  the  relators  in  the
subgroup (`no = -1') and with gap-`fill'ing active),  but  that  every
$1000$ C-style definitions a further coset is applied to all relators.

*Guru  Notes:*
The $1000/1$ mix is not necessarily optimal, and some  experimentation
may be needed  to  find  an  acceptable  balance  (see,  for  example,
\cite{HR99b}). Note also that, the longer  the  total  length  of  the
presentation,  the  more  work  needs  to  be  done  for  each   coset
application to the relators; one coset application can result in  more
than $1000$ definitions, reversing the  balance  between  R-style  and
C-style definitions.

\>`hlt'{Option hlt}&
Selects the standard HLT strategy.

\>`purec'{Option purec}&
Sets the strategy to basic C-style (coset table based).

In this strategy there is no `compaction',  no  gap-`fill'ing  and  no
relators in subgroup (`no = 0').

\>`purer'{Option purer}&
Sets the strategy  to basic R-style (relator based).

In this  strategy  there  is  no  `mendelsohn',  no  `compaction',  no
`lookahead' and no `row'-filling.

\>`sims:=<val>'{Option sims}&
Sets a Sims strategy; <val> should be one of 1, 3, 5, 7 or 9.

In his book~\cite{Sim94},  Sims  discusses  ten  standard  enumeration
strategies. These are effectively HLT (with and  without  `mendelsohn'
set -- see~"Option mendelsohn") and Felsch, all either with or without
table standardisation as the enumeration  proceeds.  {\ACE}  does  not
implement  table  standardisation  during  an  enumeration,   although
incomplete tables can be standardised and  an  enumeration  continued.
The five odd-numbered non-standardising strategies of~\cite{Sim94} are
implemented ({\ACE}'s numbering is the same as  given  in  Section~5.5
of~\cite{Sim94}).  With  care,  it  is  possible  to   duplicate   the
statistics  given  in~\cite{Sim94};  some  examples   are   given   in
Chapter~"Examples". 
% again, in the appendix ...

%%%%%%%%%%%%%%%%%%%%%%%%%%%%%%%%%%%%%%%%%%%%%%%%%%%%%%%%%%%%%%%%%%%%%%%
%UPTOHERE 

\Chapter{Using ACE Interactively}

\>StartACE( <name>, <fgens>, <rels>, <sgens> [:<options>] ) F

`StartACE' takes a string <name>, a list <fgens> of free generators,
a list <rels> of relators and a list <sgens> of subgroup generators,
both in terms of the free generators, and optionally <options>
(see~"CallAce" for the format of the latter).

The return value is a {\GAP} object that represents the running {\ACE},
it is used as an argument in all functions described below.
<name> is used to print the return value in all subsequent outputs.

\>LeaveACE( <aceobj> ) F

`LeaveACE' takes an object <aceobj> as returned by `StartACE'
(see~"StartACE"), and terminates the corresponding {\ACE} job.
%T return an exit status if possible?

\begintt
*GREG:

THERE SHOULD PERHAPS BE THREE SECTIONS, 

ONE GIVING COMMANDS THAT INTERACTIVELY 'STEER' ACE

ONE WITH FUNCTIONS THAT ALLOW  TO SEND QUESTIONS ABOUT THE COSET TABLE
TO ACE.

ONE THAT ALLOWS OPERATIONS USING THE (POSSIBLY INCOMPLETE) COSET TABLE
TO GET FURTHER INFORMATION.

I HAVE  INSERTED INTO PLACEHOLDERS  FOR THESE SECTIONS  THE RESPECTIVE
STANDALONE COMMANDS WHICH  I WOULD LIKE TO HAVE  AVAILABLE THROUGH THE
INTERFACE*
\endtt


%%%%%%%%%%%%%%%%%%%%%%%%%%%%%%%%%%%%%%%%%%%%%%%%%%%%%%%%%%%%%%%%%%%%%
\Section{Steering ACE Interactively}

\>ACECheck( <aceobj> ) F
\>ACERedo( <aceobj> ) F

An extant  enumeration is redone,  using the current  options.  Any
existing information in the table  is retained, and the enumeration is
restarted from coset \#1 (i.e., the subgroup).

*Notes:*

This option is really intended  for the case where additional relators
and/or subgroup  generators have been introduced.   The current table,
which may be  incomplete or exhibit a finite  index, is still 'valid'.
However, the new data may  allow the enumeration to complete, or cause
a collapse to a smaller index.

\>ACEContinue( <aceobj> ) F

Continue the  current enumeration,  building upon the  existing table.
If a previous run stopped without producing a finite index you can, in
principle, change any  of the options and continue  on.  Of course,
if you make any changes  which invalidate the current table, you won't
be allowed to continue, although you may be allowed to redo.


\>ACERecover( <aceobj> ) F

Invokes the compaction routine on the table to recover the space  used
by the dead coset numbers. A `CO' message line is printed if any  rows
of the coset table were recovered, and a `co' line if none were.

*GREG: DOES THE NEXT SENTENCE MAKE SENSE FOR THE INTERFACE?*

This routine is  called automatically if the `cy',  `nc', `pr',or `st'
options are involved.


\>ACEAddGenerators( <aceobj>, <wordlist> ) F

Adds the words in the list <wordlist> to any subgroup generators already
present.
The enumeration must be restarted or redone, it cannot be continued.

\>ACEAddRelators( <aceobj>, <relatorlist> ) F

Adds the words in the list <relatorlist> to any relators already present.
The enumeration must be restarted or redone, it cannot be continued.

\>ACECosetCoincidence( <aceobj>, <int> ) F

Print out the representative of coset <int>, and add it to the
subgroup generators; i.e., equates this coset with coset \#1, the
subgroup.


%%%%%%%%%%%%%%%%%%%%%%%%%%%%%%%%%%%%%%%%%%%%%%%%%%%%%%%%%%%%%%%%%%%%%%%%
\Section{Interactive Questions}

% \>`dump' := [0/1/2[,0/1]]&
% 
% Dumps  the internal  variables of  {\ACE}.  The  first  argument selects
% which of  the three levels  of {\ACE} to  dump, and defaults  to Level
% $0$.  The second argument selects the level of detail, and defaults to
% $0$  (i.e., less  detail).  This  option  is intended  for gurus;  the
% source code should be consulted to see what the output means.

\>ACEPrintOptions( <aceobj> ) F

This command prints details of the options included  in the version of
{\ACE} you're  running; i.e.,  what compiler flags  were set  when the
executable  was built.   A  typical output,  illustrating the  default
build, is:

\begintt
Executable built:
  Sat Feb 27 15:57:59 EST 1999
Level 0 options:
  statistics package = on
  coinc processing messages = on
  dedn processing messages = on
Level 1 options:
  workspace multipliers = decimal
Level 2 options:
  host info = on
\endtt


\>ACEPrintDetails( <aceobj>[, <params> ) F

This  command  prints out  details  of  the  current presentation  and
options.  No second argument, or second argument `false', prints out
the group
and subgroup name, the group's relators and the subgroup's generators.
If the second argument is `true', then  the current setting of the
options is
also printed.  The printout is the  same as that produced at the start
of a run when messaging is on.

% *Notes:*
% The output is printed out in a form suitable for input, so that a record
% of a previous run can be used to replicate the run.
% Note that, due to the defaulting of some options and the special
% meaning attached to some values, a little care has to be taken in
% interpreting the options.
%T mention outfile option here?

*GREG: DOES THE FOLLOWING PASSAGE MAKE SENSE WITH THE INTERFACE?*
If you wish to *exactly* duplicate a run, you should use the output
of `sr' *after* the run completes.


\>ACEPrintStatistics( <aceobj> ) F

If the statistics package is compiled into the code (which it is by
default),
then print the statistics accumulated during the most recent enumeration.


%%%%%%%%%%%%%%%%%%%%%%%%%%%%%%%%%%%%%%%%%%%%%%%%%%%%%%%%%%%%%%%%%%%%%%%%
\Section{Operations with a Coset table}

\>ACECycles( <aceobj> ) F

returns a list of permutations corresponding to the generators,
i.e., the permutation representation.
%T only for complete tables?

\>ACENormalClosure( <aceobj>[, <conjugates>] ) F

If the second argument is missing or `false', test for normal closure.
If a subgroup  generator  does not  normalise  a  generator,  then this  is
returned.
If  the  argument  is `true',  then  any  non-normalising
conjugates are also added to the subgroup generators.

\>ACEOrder( <aceobj>, <int> ) F

This function searches for cosets with order modulo the subgroup a multiple
of the absolute value of the integer <int>.
If <int> is negative then the list of all cosets meeting the requirement is
returned.
If <int> is positive then the first coset meeting the requirement is
returned if such a coset exists, and `fail' otherwise.

\>ACEOrders( <aceobj> ) F

returns the list of orders modulo the subgroup of all cosets.


\>ACEStabilisingCosets( <aceobj>, <int> ) F

This function returns stabilising cosets of the subgroup.

\beginlist
\item{--} If <int> $> 0$, the list of the first <int> stabilising cosets
    is returned.
\item{--} If <int> $\< 0$, the list of pairs of the first <int> stabilising
    cosets, plus their representatives, is returned.
\item{--} If <int> = $0$, the list of pairs of all of the stabilising cosets,
    plus their representatives, is returned.
\endlist


\>ACEStandardTable( <aceobj> ) F

This function compacts and then standardises the table.
That is, for a given ordering of the generators in the columns of the
table, it produces a canonic table.
In such a table, a row-major scan encounters previously unseen cosets in
numeric order.

*GREG, WHAT DOES THE LAST SENTENCE MEAN???*

*Notes:*
In a canonic table, the coset representatives are in length plus (column
order) lexicographic order, and each is the minimum in this order.
Two tables are equivalent only if their canonic forms are the same.

*Guru Notes:*
In half of the ten standard enumeration strategies of Sims \cite{Sim94},
the   table   is   standardised   repeatedly.    This   is   expensive
computationally, but can result  in fewer cosets being necessary.  The
effect of  doing this  can be investigated  in {\ACE}  by (repeatedly)
halting the enumeration, standardising the table, and continuing.

\>ACETraceWord( <aceobj>, <int>, <word> ) F

Traces <word> through the coset table, starting at coset number <int>,
and returns the final coset number if the trace completes,
and `fail' otherwise.


\>ACERandomCoincidences( <aceobj>, <index>[, <attempts> ) F

This function attempts  to  find  nontrivial subgroups  with  index  a
multiple  of the first  argument <index> by  repeatedly putting  random
cosets
coincident with coset \#1 and seeing what happens.  The starting coset
table  must  be non-empty,  but  need  not  be complete.   The  second
argument <attempts> puts  a limit on  the number of  attempts,
with a  default of
eight.    For   each  attempt,   we   repeatedly   add  random   coset
representatives  to the  subgroup and  redo the  enumeration.   If the
table becomes too small, the attempt is aborted, the original subgroup
generators  restored,  and  another   attempt  made.   If  an  attempt
succeeds, then the new set of subgroup generators is retained.

*Guru  Notes:*  A  coset  can  have  many  different  representatives.
Consider running  `st' before  `rc', to canonicise  the table  and the
representatives.


%%%%%%%%%%%%%%%%%%%%%%%%%%%%%%%%%%%%%%%%%%%%%%%%%%%%%%%%%%%%%%%%%%%%%%%%%
\Chapter{Examples}

\begintt
*GREG:

HERE I WANT EVENTUALLY TO INSERT MORE EXAMPLES USING THE INTERFACE,
THIS IS JUST THE ONE ALEXANDER HAD IN HIS FIRST SHORT DRAFT*
\endtt

The following example calls `ACE' for up to 800 cosets using
Mendelsohn style relator processing and sets the message level to 500
\begintt
gap> g:=PerfectGroup(2^5*60,2);;
gap> f:=FreeGeneratorsOfFpGroup(g);;
gap> CallACE(f,RelatorsOfFpGroup(g),f{[1]}:mendelsohn,max:=800,mess:=500);;
ACE 3.000        Fri Aug 20 16:05:01 1999
=========================================
Host information:
  name = muir
  #-- ACE 3.000: Run Parameters ---
Group Name: G;
Group Generators: abcdefg;
Group Relators: (c)^2, (d)^2, (e)^2, (f)^2, (g)^2, aag, (b)^3, (cd)^2, 
  (ef)^2, (ce)^2, (cf)^2, (de)^2, (df)^2, Acae, Adaf, Aeac, Afad, Bfbe, 
  gAga, gBgb, (gc)^2, (gd)^2, (ge)^2, (gf)^2, Bebfe, Bcbgfd, Bdbfedc, 
  (ab)^5;
Subgroup Name: H;
Subgroup Generators: a;
Wo:1000000; Max:500; Mess:500; Ti:-1; Ho:-1; Loop:0;
As:0; Path:0; Row:1; Mend:1; No:28; Look:0; Com:10;
C:0; R:0; Fi:13; PMod:3; PSiz:256; DMod:4; DSiz:1000;
  #--------------------------------
SG: a=1 r=1 h=1 n=2; l=1 c=+0.00; m=1 t=1
RD: a=321 r=68 h=1 n=412; l=5 c=+0.00; m=327 t=411
CL: a=302 r=107 h=1 n=501; l=7 c=+0.01; m=327 t=500
DD: a=302 r=107 h=1 n=501; l=8 c=+0.00; d=386
CO: a=302 r=83 h=139 n=303; l=9 c=+0.01; m=327 t=500
DD: a=379 r=101 h=237 n=388; l=11 c=+0.00; d=3
DD: a=456 r=101 h=266 n=465; l=11 c=+0.00; d=5
INDEX = 480 (a=480 r=101 h=489 n=489; l=12 c=0.02; m=480 t=686)
\endtt

If `ACE' is made the standard coset enumerator the same method of passing
arguments may be used with all other commands and will affect coset
enumerations. As an example we use the `ACE' enumerator to compute the
permutation representation of a perfect group from the data library:

\begintt
gap> TCENUM:=ACETCENUM;;
gap> PerfectGroup(IsPermGroup,16*60,1:max:=50,mess);
gap> PerfectGroup(IsPermGroup,16*60,1:max:=50,mess);
ACE 3.000        Fri Aug 20 16:06:16 1999
=========================================
Host information:
  name = muir
INDEX = 16 (a=16 r=36 h=1 n=36; l=3 c=0.00; m=30 t=35)
A5 2^4
\endtt

(If you run the examples time, machine name or the version number of the
binary might differ.)


%%%%%%%%%%%%%%%%%%%%%%%%%%%%%%%%%%%%%%%%%%%%%%%%%%%%%%%%%%%%%%%%%%%%%%%%%
%%
%E

