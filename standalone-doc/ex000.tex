
This example uses input file \ttt{ex000.in}, and illustrates the basics of
  \ace.
Note how the input is generally insensitive to command synonyms,
  capitalisation, white space, and \ttt{:} \amp \ttt{;} characters.
%
When \ace\ starts up, it prints out its version, the date \amp time, and
  the name of the host on which it's running.
If we attempt to do an enumeration immediately we get an error, since the
  lack of generators means we can't build the (empty) coset table.

\bv\begin{verbatim}
ACE 3.000        Sat May  8 13:50:29 1999
=========================================
Host information:
  name = flute
end;
** ERROR (continuing with next line)
   can't start (no generators?)
\end{verbatim}\ev

After defining two generators, we can do an enumeration.
The default state is not to echo the presentation or print any messages;
  only the result line is printed.
The group is free, since there are no relators, and the subgroup is
 trivial.
So the enumeration overflows.

\bv\begin{verbatim}
gr:ab;          # A stupid comment
Begin
OVERFLOW (a=249998 r=83333 h=83333 n=249999; l=337 c=0.97; 
                                                       m=249998 t=249998)
\end{verbatim}\ev

The \ttt{sr} commands dumps out the presentation and the parameters for
  the run.
All of these are currently defaulted, apart from those dependent on there
  being two (non-involutionary) generators.

\bv\begin{verbatim}
sr:1;
  #-- ACE 3.000: Run Parameters ---
Group Name: G;
Group Generators: ab;
Group Relators: ;
Subgroup Name: H;
Subgroup Generators: ;
Wo:1000000; Max:249998; Mess:0; Ti:-1; Ho:-1; Loop:0;
As:0; Path:0; Row:1; Mend:0; No:0; Look:0; Com:10;
C:0; R:0; Fi:7; PMod:3; PSiz:256; DMod:4; DSiz:1000;
  #--------------------------------
\end{verbatim}\ev

This is the first part of the table.
Note that, as there are no relators, the table has separate columns for
  generator inverses.
So the default workspace of $1000000$ words allows
  a table of $249998 = 1000000/4 - 2$ cosets.
As row filling is on by default, the table is simply filled with
  cosets in order.
Note that a compaction phase is done before printing the table, but that
  this does nothing here (the lower-case \ttt{co} tag), since there are no 
 dead cosets.
The coset representatives are simply all possible freely reduced words, in
  length plus lexicographic order.

\bv\begin{verbatim}
pr:-1,12;
co: a=249998 r=83333 h=83333 n=249999; c=+0.00
 coset |      a      A      b      B   order   rep've
-------+---------------------------------------------
     1 |      2      3      4      5
     2 |      6      1      7      8       0   a
     3 |      1      9     10     11       0   A
     4 |     12     13     14      1       0   b
     5 |     15     16      1     17       0   B
     6 |     18      2     19     20       0   aa
     7 |     21     22     23      2       0   ab
     8 |     24     25      2     26       0   aB
     9 |      3     27     28     29       0   AA
    10 |     30     31     32      3       0   Ab
    11 |     33     34      3     35       0   AB
    12 |     36      4     37     38       0   ba
\end{verbatim}\ev

We now set things up to do the alternating group on five letters, of order
  $60$.
We turn messaging on, but set the interval high enough so that there will
  be no progress messages.

\bv\begin{verbatim}
Enum: A_5;
rel: a^2, b^3, ababababab;
subgr: trivial;
mess: 1000;    start;
\end{verbatim}\ev

The presentation and the parameters are echoed, the enumeration is
  performed, and then the results of the run are printed.
Note that the exponent of the \ttt{ababababab} word has been correctly
  deduced, and that \ttt{a} is treated as an involution.
So the table has only three columns now.
Definitions are HLT-style, and a total of $76$ cosets (incl.\@ the 
  subgroup) were defined.

\bv\begin{verbatim}
  #-- ACE 3.000: Run Parameters ---
Group Name: A_5;
Group Generators: ab;
Group Relators: (a)^2, (b)^3, (ab)^5;
Subgroup Name: trivial;
Subgroup Generators: ;
Wo:1000000; Max:333331; Mess:1000; Ti:-1; Ho:-1; Loop:0;
As:0; Path:0; Row:1; Mend:0; No:3; Look:0; Com:10;
C:0; R:0; Fi:6; PMod:3; PSiz:256; DMod:4; DSiz:1000;
  #--------------------------------
INDEX = 60 (a=60 r=77 h=1 n=77; l=3 c=0.00; m=66 t=76)
\end{verbatim}\ev

We now use a non-trivial subgroup, and monitor all the actions of the
  enumerator.

\bv\begin{verbatim}
Subgroup Name: C_5 ;
gen:ab;
Monit :1
END;
  #-- ACE 3.000: Run Parameters ---
Group Name: A_5;
Group Generators: ab;
Group Relators: (a)^2, (b)^3, (ab)^5;
Subgroup Name: C_5;
Subgroup Generators: ab;
Wo:1000000; Max:333331; Mess:1; Ti:-1; Ho:-1; Loop:0;
As:0; Path:0; Row:1; Mend:0; No:3; Look:0; Com:10;
C:0; R:0; Fi:6; PMod:3; PSiz:256; DMod:4; DSiz:1000;
  #--------------------------------
AD: a=2 r=1 h=1 n=3; l=1 c=+0.00; m=2 t=2
SG: a=2 r=1 h=1 n=3; l=1 c=+0.00; m=2 t=2
RD: a=3 r=1 h=1 n=4; l=2 c=+0.00; m=3 t=3
RD: a=4 r=2 h=1 n=5; l=2 c=+0.00; m=4 t=4
RD: a=5 r=2 h=1 n=6; l=2 c=+0.00; m=5 t=5
RD: a=6 r=2 h=1 n=7; l=2 c=+0.00; m=6 t=6
RD: a=7 r=2 h=1 n=8; l=2 c=+0.00; m=7 t=7
RD: a=8 r=2 h=1 n=9; l=2 c=+0.00; m=8 t=8
RD: a=9 r=2 h=1 n=10; l=2 c=+0.00; m=9 t=9
CC: a=8 r=2 h=1 n=10; l=2 c=+0.00; d=0
RD: a=9 r=5 h=1 n=11; l=2 c=+0.00; m=9 t=10
RD: a=10 r=5 h=1 n=12; l=2 c=+0.00; m=10 t=11
RD: a=11 r=5 h=1 n=13; l=2 c=+0.00; m=11 t=12
RD: a=12 r=5 h=1 n=14; l=2 c=+0.00; m=12 t=13
RD: a=13 r=5 h=1 n=15; l=2 c=+0.00; m=13 t=14
RD: a=14 r=5 h=1 n=16; l=2 c=+0.00; m=14 t=15
CC: a=13 r=6 h=1 n=16; l=2 c=+0.00; d=0
CC: a=12 r=6 h=1 n=16; l=2 c=+0.00; d=0
INDEX = 12 (a=12 r=16 h=1 n=16; l=3 c=0.00; m=14 t=15)
\end{verbatim}\ev

We now dump out the statistics accumulated during the run.
The run had \ttt{a=12} \amp \ttt{t=15}, so there must have been three
  coincidences (\ttt{qcoinc=3}).
Of these, two were primary coincidences (\ttt{rdcoinc=2}).
Since \ttt{t=15}, there must have been fourteen coset definitions;  one
  was during the application of coset \#1 (i.e., the subgroup) to the
  subgroup generator (\ttt{apdefn=1}), and the remainder during the
  application of the cosets to the relators (\ttt{rddefn=13}).

\bv\begin{verbatim}
STATistics;
  #- ACE 3.000: Level 0 Statistics --
cdcoinc=0 rdcoinc=2 apcoinc=0 rlcoinc=0 clcoinc=0
  xcoinc=2 xcols12=4 qcoinc=3
  xsave12=0 s12dup=0 s12new=0
  xcrep=6 crepred=0 crepwrk=0 xcomp=0 compwrk=0
xsaved=0 sdmax=0 sdoflow=0
xapply=1 apdedn=1 apdefn=1
rldedn=0 cldedn=0
xrdefn=1 rddedn=5 rddefn=13 rdfill=0
xcdefn=0 cddproc=0 cdddedn=0 cddedn=0
  cdgap=0 cdidefn=0 cdidedn=0 cdpdl=0 cdpof=0
  cdpdead=0 cdpdefn=0 cddefn=0
  #----------------------------------
\end{verbatim}\ev

Note how the pre-printout compaction phase now does some work (the
  upper-case \ttt{CO} tag), since there were coincidences, and hence dead
  cosets.
Note how \ttt{b/B} have been used as the first two columns, since these
  must be occupied by a generator/inverse pair or a pair of involutions.
The \ttt{a} column is also the \ttt{A} column, as \ttt{a} is an
  involution.

\bv\begin{verbatim}
  print  TABLE : -1, 12 ;
CO: a=12 r=13 h=1 n=13; c=+0.00
 coset |      b      B      a   order   rep've
-------+--------------------------------------
     1 |      3      2      2
     2 |      1      3      1       3   B
     3 |      2      1      4       3   b
     4 |      8      5      3       5   ba
     5 |      4      8      6       2   baB
     6 |      9      7      5       5   baBa
     7 |      6      9      8       3   baBaB
     8 |      5      4      7       5   bab
     9 |      7      6     10       5   baBab
    10 |     12     11      9       3   baBaba
    11 |     10     12     12       2   baBabaB
    12 |     11     10     11       3   baBabab
\end{verbatim}\ev

If we define the generator order to be that of the columns, then the table
  above is not in canonic form, and the coset representatives are not in
  order.
We now standardise the table; note the compaction phase before
  standardisation, although it does nothing in this particular case.
Note how, if we read through the table in row-major order, new cosets
  are introduced using the smallest available number, and that the 
  representatives are now in order and are minimal.

\bv\begin{verbatim}
st;
co/ST: a=12 r=13 h=1 n=13; c=+0.00
pr:-1,12;
co: a=12 r=13 h=1 n=13; c=+0.00
 coset |      b      B      a   order   rep've
-------+--------------------------------------
     1 |      2      3      3
     2 |      3      1      4       3   b
     3 |      1      2      1       3   B
     4 |      5      6      2       5   ba
     5 |      6      4      7       5   bab
     6 |      4      5      8       2   baB
     7 |      8      9      5       5   baba
     8 |      9      7      6       5   baBa
     9 |      7      8     10       3   babaB
    10 |     11     12      9       3   babaBa
    11 |     12     10     12       3   babaBab
    12 |     10     11     11       2   babaBaB
\end{verbatim}\ev

We now exit \ace, printing out the version and the date \amp time again.

\bv\begin{verbatim}
q
=========================================
ACE 3.000        Sat May  8 13:52:49 1999
\end{verbatim}\ev

