%%%%%%%%%%%%%%%%%%%%%%%%%%%%%%%%%%%%%%%%%%%%%%%%%%%%%%%%%%%%%%%%%%%%%%%%%
%%
%W  back.tex           ACE standalone documentation          Colin Ramsay
%W                                                            Greg Gamble
%%
%H  $Id$
%%

%%  Colin Ramsay - 11 May 1999  is the true author of this document
%%  Greg Gamble  - 28-29 Feb 2000 made minor modifications:
%%                * modifications detailed in CHANGES
%%                * put this header at the top for CVS

%%  The background chapter.
%%
%%  5   10   15   20   25   30   35   40   45   50   55   60   65   70   75
%%..|....|....|....|....|....|....|....|....|....|....|....|....|....|....|

\section{Terminology}

Although \ace\ can accept either letters or numbers for group generators,
  we generally use letters, since these are much easier to understand.
(Unless you need more than $26$ generators, or are using some form of 
  automatically generated presentation, you should adopt the same
  convention.)
Lower-case letters denote generators, with inverses being denoted by either
  upper-case letters or negative superscripts; e.g., $ABab$ and
  $a^{-1}b^{-1}ab$ are equivalent.
We use $1$ to denote the identity element and/or the subgroup (i.e.,
  coset \#1).

%\dots scanning, applying, closing.
%\dots dead coset(s), compact(ion).

\section{Groups notation}
For a subgroup $H$ of a group $G$, we represent by $[G:H]$ the set of right
  cosets of $H$ in $G$, i.e.
  $$ [G:H] = \{ Hx \mid x \in G \}. $$
Two cosets $Hx, Hy \in [G:H]$ are equal, i.e.~coincident, if and only if
  $xy^{-1} \in H$.
Also, any two cosets of $[G:H]$ are either coincident or disjoint.
The cardinality of $[G:H]$ is the number of distinct cosets in $[G:H]$,
  and is equal to the index $|G:H|$ of $H$ in $G$, 
  i.e.~$\bigl|[G:H]\bigr| = |G:H|$.
In fact, $[G:H]$ is a partition of $G$ into subsets of $G$, each of
  cardinality $|H|$, the order of $H$; and if $G$ is finite then
  $|G:H| = |G|/|H|$.

Some standard groups that arise in our examples are:
  \begin{quote}
    $S_n$, the full \emph{symmetric group} on $n$ letters;\\
    $A_n$, the full \emph{alternating group} on $n$ letters; and\\
    $C_n$, the \emph{cyclic group} of order $n$.
  \end{quote}

A group $G$ will often be defined via a \emph{presentation} of the form
  $$\langle \textit{generators} \mid \textit{relators} \rangle.$$
In this case, the elements of $G$ are words in the \emph{generators}
  and the \emph{relators} are a list of words that are equivalent to 
  the empty word (i.e.~identity element) in $G$. 
Actually, amongst the \emph{relators} we will also allow \emph{relations},
  which are equations of the form $w_1 = w_2$ (equivalent to the relator 
  $w_1w_2^{-1}$), where $w_1, w_2$ are words in the generators of $G$.
See the introduction of \cite{MKS} for a rather nice account of the
  terminology of presentations.

\section{History}

%Cosets were first defined and discussed, in abstract terms, by Galois and
%  Cayley.
The concept of a subgroup, and its cosets, has been known since the 
  beginnings of group theory.
One of the earliest (practical?) uses of cosets seems to have been by
  Moore \cite{Moo}, who gives presentations for $S_n$ \amp $A_n$ and
  proves them correct by, in effect, counting the $n$ cosets of 
  $[S_n:S_{n-1}]$ \amp $[A_n:A_{n-1}]$.
Dickson \cite[\S264]{Dic} presents a more accessible account, and 
  explicitly notes that ``these sets form a rectangular table''\kern-2pt.
%
To illustrate this, we paraphrase Dickson's proof for the case $S_4$.
(In order to be as faithful as possible to Dickson's account, we retain
his numbering of the cosets; an ACE enumeration would have labelled what
is $O_4$ below, as coset 1.)

Let $G_4$ be the abstract group
  $$ \langle b_1,b_2,b_3 \mid b_1^2,b_2^2,b_3^2,
       b_1b_3=b_3b_1, b_1b_2b_1=b_2b_1b_2,b_2b_3b_2=b_3b_2b_3 \rangle . $$
Now $S_4$ is generated by the transpositions $s_1=(1\,2)$, $s_2=(2\,3)$ \amp 
  $s_3=(3\,4)$.
Putting $s_i=b_i$, $1 \le i \le 3$, we see that these transpositions 
  satisfy the defining relations of $G_4$.
So $S_4$ is a quotient group of $G_4$, and $|G_4| \ge |S_4| = 4! = 24$.

That $|G_4| \le 24$, and so $G_4 \cong S_4$, is proved by induction.
Let $G_3$ be the subgroup of $G_4$ generated by $b_1$ \amp $b_2$.
(The actual induction is on the $b_i$.
For our purposes, we'll simply assume that $|G_3| \le 6$.)
Now consider the cosets $O_4=G_3$, $O_3=G_3b_3$, $O_2=G_3b_3b_2$ \amp 
  $O_1=G_3b_3b_2b_1$.
We'll show that these four cosets are merely permuted by the $b_i$, so
  that the index $|G_4 : G_3| \le 4$; hence $|G_4| \le 24$, as required.

Obviously, $O_4b_3=O_3$, $O_3b_2=O_2$ \amp $O_2b_1=O_1$.
Since the $b_i$ are involutions, then $O_3b_3=G_3b_3b_3=G_3=O_4$.
Similarly, $O_2b_2=O_3$ \amp $O_1b_1=O_2$.
Since $b_1$ \amp $b_2$ generate $G_3$, then $O_4b_1=O_4b_2=O_4$.
Now, since $b_1$ \amp $b_3$ commute, then 
  $O_3b_1=G_3b_3b_1=G_3b_1b_3=O_4b_1b_3=O_4b_3=O_3$.
Now consider $O_1b_3=G_3b_3b_2b_1b_3=G_3b_3b_2b_3b_1$.
Since $b_2b_3b_2=b_3b_2b_3$, then this can be written as 
  $G_3b_2b_3b_2b_1=O_4b_2b_3b_2b_1=O_4b_3b_2b_1=O_1$.
In a similar manner, $O_1b_2=O_1$ \amp $O_2b_3=O_2$.
Our coset table (see Table~\ref{tab:dic}) is now complete, so 
  $|G_4 : G_3| \le 4$.
Note that the $b_i$ give rise to transpositions $(O_i\,O_{i+1})$,
  which if $O_i$ is identified with $i$, for each $i$, give back the 
  transpositions $s_1=(1\,2)$, $s_2=(2\,3)$ \amp $s_3=(3\,4)$ that 
  generate $S_4$.
Thus, the permutations $(O_i\,O_{i+1})$ give us a \emph{faithful} permutation
  representation of $S_4$. 
It is \emph{faithful} in this case, since the largest subgroup of $S_3$ 
  (where we consider $S_3$ to be the naturally embedded subgroup of $S_4$
  generated by $(1\,2)$ and $(2\,3)$\,)
  that is \emph{normal} in $S_4$ is the identity subgroup.

\begin{table}
\hrule
\caption{The coset table for $[S_4:S_3]$}
\label{tab:dic}
\smallskip
\renewcommand{\arraystretch}{0.875}
\begin{tabular*}{\textwidth}{@{\extracolsep{\fill}}llll} 
\hline\hline
 & \multicolumn{3}{c}{Generators} \\
\cline{2-4}
coset                   & $b_1$ & $b_2$ & $b_3$ \\ 
\hline
 $O_4$ ($G_3$)          & $O_4$ & $O_4$ & $O_3$ \\
 $O_3$ ($G_3b_3$)       & $O_3$ & $O_2$ & $O_4$ \\
 $O_2$ ($G_3b_3b_2$)    & $O_1$ & $O_3$ & $O_2$ \\
 $O_1$ ($G_3b_3b_2b_1$) & $O_2$ & $O_1$ & $O_1$ \\
\hline\hline
\end{tabular*}
\end{table}

The construction of a coset table was systematised and popularised by Todd
  \amp Coxeter \cite{TC} (or see \cite[Chapter 2]{CM}).
The first known computer implementation was that of Haselgrove in $1953$.
This, along with other early implementations, is described by Leech
  \cite{Lee3}.
Detailed accounts of the techniques used in coset enumeration can be found
  in \cite{CDHW,Hav,Lee,Neu,Sim}.
Formal proofs of the correctness of various strategies for coset
  enumeration are given in \cite{Men,Neu,Sim}.

%%%%%%%%%%%%%%%%%%%%%%%%%%%%%%%%%%%%%%%%%%%%%%%%%%%%%%%%%%%%%%%%%%%%%%%%%
%%
%E
