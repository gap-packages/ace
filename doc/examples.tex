%%%%%%%%%%%%%%%%%%%%%%%%%%%%%%%%%%%%%%%%%%%%%%%%%%%%%%%%%%%%%%%%%%%%%%%%%
%%
%W  examples.tex        ACE appendix - examples          Alexander Hulpke
%W                                                      Joachim Neub"user
%W                                                            Greg Gamble
%%
%H  $Id$
%%
%Y  Copyright (C) 2000  Centre for Discrete Mathematics and Computing
%Y                      Department of Computer Science & Electrical Eng.
%Y                      University of Queensland, Australia.
%%

%%%%%%%%%%%%%%%%%%%%%%%%%%%%%%%%%%%%%%%%%%%%%%%%%%%%%%%%%%%%%%%%%%%%%%%
\Chapter{Examples}

In this chapter  we  collect  together  a  number  of  examples  which
illustrate the various ways in which the {\ACE} Share Package  may  be
used, and give some interactions with the `ACEExample' function. In  a
number of cases, we have set the `InfoLevel' of  `InfoACE'  to  3,  so
that all output from {\ACE} is displayed, prepended by  \lq{}`\#I  ''.
Recall that to also see the commands directed *to*  {\ACE}  (behind  a
\lq{}`ToACE> '' prompt), you will need to set the `InfoACE'  level  to
4. We have omitted the line

\beginexample
gap> RequirePackage("ace");
\endexample

which is,  of  course,  required  at  the  beginning  of  any  session
requiring {\ACE}. The  first  few  sections  are  the  nearest  {\GAP}
equivalent of the similar sections in the  Appendix  of  `ace3001.dvi'
(the standalone manual in directory `standalone-doc').

%%%%%%%%%%%%%%%%%%%%%%%%%%%%%%%%%%%%%%%%%%%%%%%%%%%%%%%%%%%%%%%%%%%%%%
\Section{Getting Started}

Each     of     the      functions      `ACECosetTableFromGensAndRels'
(see~"ACECosetTableFromGensAndRels"),  `ACEStats'   (see~"ACEStatsNI")
and `ACEStart' (see~"ACEStart"), may be called with  three  arguments:
<fgens> (the group  generators),  <rels>  (the  group  relators),  and
<sgens> (the subgroup generators). While it is legal for the arguments
<rels> and <sgens> to be empty  lists,  it  is  always  an  error  for
<fgens> to be empty, e.g.

\beginexample
gap> ACEStats([], [], [] : echo, max := 10);
ACEStats called with the following arguments:
 Group generators : [  ]
 Group relators : [  ]
 Subgroup generators : [  ]
Error : first argument defines an empty list of group generators at
Error( ": first argument defines an empty list of group generators" );
CALL_ACE( "ACEStats", arg[1], arg[2], arg[3] ) called from
<function>( <arguments> ) called from read-eval-loop
Entering break read-eval-print loop, you can 'quit;' to quit to outer loop,
or you can return to continue
brk> quit; # All we can do!                  
gap> # We set some options. So 'OptionsStack' will be non-empty ...
gap> DisplayOptionsStack();
[ rec(
      echo := true,
      max := 10 ) ]
gap> # We should clear the 'OptionsStack' before proceeding ...
gap> FlushOptionsStack();
gap> DisplayOptionsStack(); # Just to show 'OptionsStack' is now empty.
[  ]
gap>
\endexample

The function  `FlushOptionsStack'  is  described  in  Section~"General
Warnings regarding the Use of Options".

After defining generators, we can do an enumeration:

\beginexample
gap> F := FreeGroup("a", "b");;
gap> fgens := GeneratorsOfGroup(F);
[ a, b ]
\endexample

%%%%%%%%%%%%%%%%%%%%%%%%%%%%%%%%%%%%%%%%%%%%%%%%%%%%%%%%%%%%%%%%%%%%%%
*A first  `ACEStats'  example*{\undoquotes\atindex{ACEStats,  a  first
example}{@`ACEStats', a first example}}

By  default,  the  presentation  is  not  echoed;   use   the   `echo'
(see~"option echo") option if you want that.  Also,  by  default,  the
{\ACE} binary only prints a *results message*, but we won't  see  that
unless   we   set   `InfoACE'   to   a   level   of   at    least    2
(see~"SetInfoACELevel"):

\beginexample
gap> SetInfoACELevel(2);
\endexample

Calling `ACEStats' with arguments `fgens', `[]', `[]', defines a  free
froup with 2 generators, since the second argument  defines  an  empty
relator list; and since  the  third  argument  is  an  empty  list  of
generators, the  subgroup  defined  is  trivial.  So  the  enumeration
overflows:

\beginexample
ACEStats(fgens, [], []);
#I  OVERFLOW (a=249998 r=83333 h=83333 n=249999; l=337 c=0.22; m=249998 t=2499\
98)
rec( index := 0, cputime := 22, cputimeUnits := "10^-2 seconds", 
  activecosets := 249998, maxcosets := 249998, totcosets := 249998 )
\endexample

The  line  starting  with  \lq{}`\#I  ''  is  the  `Info'-ed  *results
message* from {\ACE};  see  Appendix~"The  Meanings  of  ACE's  Output
Messages" for details  on  what  it  means.  Observe  that  since  the
enumeration overflowed, {\ACE}'s result message  has  been  translated
into a {\GAP} record with `index' field 0.

To dump  out  the  presentation  and  parameters  associated  with  an
enumeration,  {\ACE}  provides  the  `sr'  (see~"option  sr")  option.
However, you won't see output of this  command,  unless  you  set  the
`InfoACELevel' to at least 3. Also, to ensure the reliability  of  the
output of the `sr' option, an enumeration  should  *precede*  it;  for
`ACEStats' (and `ACECosetTableFromGensAndRels') the directive  `start'
(see~"option start") required to initiate an enumeration  is  inserted
(automatically) after all the  user's  options,  except  if  the  user
herself supplies an option that initiates an enumeration (namely,  one
of `start' or `begin' (see~"option start"), `aep'  (see~"option  aep")
or `rep' (see~"option rep")). Interactively,  we  will  see  that  the
equivalent    of    the    `sr'     command     is     `ACEParameters'
(see~"ACEParameters"),  which  gives  an   output   record   that   is
immediately {\GAP}-usable. With the above  in  mind  let's  rerun  the
enumeration and get {\ACE}'s dump of the presentation and parameters:

\beginexample
gap> SetInfoACELevel(3);     
gap> ACEStats(fgens, [], [] : start, sr := 1);
#I  ACE 3.000        Wed Aug 16 14:11:12 2000
#I  =========================================
#I  Host information:
#I    name = boronia
#I  OVERFLOW (a=249998 r=83333 h=83333 n=249999; l=337 c=0.22; m=249998 t=2499\
98)
#I    #-- ACE 3.000: Run Parameters ---
#I  Group Name: G;
#I  Group Generators: ab;
#I  Group Relators: ;
#I  Subgroup Name: H;
#I  Subgroup Generators: ;
#I  Wo:1000000; Max:249998; Mess:0; Ti:-1; Ho:-1; Loop:0;
#I  As:0; Path:0; Row:1; Mend:0; No:0; Look:0; Com:10;
#I  C:0; R:0; Fi:7; PMod:3; PSiz:256; DMod:4; DSiz:1000;
#I    #--------------------------------
#I  ***
rec( index := 0, cputime := 22, cputimeUnits := "10^-2 seconds", 
  activecosets := 249998, maxcosets := 249998, totcosets := 249998 )
\endexample

Observe that at `InfoACE' level 3, one also gets {\ACE}'s  banner.  We
could have printed out the first few lines of the coset  table  if  we
had wished, using the `print' (see~"option print") option, but note as
with the `sr' option, an enumeration should *precede* it. Here's  what
happens if you disregard this (recall, we  still  have  the  `InfoACE'
level set to 3):

\beginexample
gap> ACEStats(fgens, [], [] : print := [-1, 12]);
#I  ACE 3.000        Wed Aug 16 14:24:54 2000
#I  =========================================
#I  Host information:
#I    name = boronia
#I  ** ERROR (continuing with next line)
#I     no information in table
#I  OVERFLOW (a=249998 r=83333 h=83333 n=249999; l=337 c=0.20; m=249998 t=2499\
98)
#I  ***
rec( index := 0, cputime := 20, cputimeUnits := "10^-2 seconds", 
  activecosets := 249998, maxcosets := 249998, totcosets := 249998 )
\endexample

Essentially, because {\ACE} had  not  done  an  enumeration  prior  to
getting the `print' directive, it complained with an \lq{}`** ERROR'',
recovered  and  went  on  with  the  `start'  directive  automatically
inserted by the `ACEStats' command:  no  ill  effects  at  the  {\GAP}
level, but also no table.

Now, let's do what we should have done (to get those first  few  lines
of the coset table), namely, insert  the  `start'  option  before  the
`print' option:

\beginexample
gap> ACEStats(fgens, [], [] : start, print := [-1, 12]);
#I  ACE 3.000        Wed Aug 16 14:37:21 2000
#I  =========================================
#I  Host information:
#I    name = boronia
#I  OVERFLOW (a=249998 r=83333 h=83333 n=249999; l=337 c=0.21; m=249998 t=2499\
98)
#I  co: a=249998 r=83333 h=83333 n=249999; c=+0.00
#I   coset ||      a      A      b      B   order   rep've
#I  -------+---------------------------------------------
#I       1 ||      2      3      4      5
#I       2 ||      6      1      7      8       0   a
#I       3 ||      1      9     10     11       0   A
#I       4 ||     12     13     14      1       0   b
#I       5 ||     15     16      1     17       0   B
#I       6 ||     18      2     19     20       0   aa
#I       7 ||     21     22     23      2       0   ab
#I       8 ||     24     25      2     26       0   aB
#I       9 ||      3     27     28     29       0   AA
#I      10 ||     30     31     32      3       0   Ab
#I      11 ||     33     34      3     35       0   AB
#I      12 ||     36      4     37     38       0   ba
#I  ***
rec( index := 0, cputime := 21, cputimeUnits := "10^-2 seconds", 
  activecosets := 249998, maxcosets := 249998, totcosets := 249998 )
\endexample

The values we gave to the `print' option, told  {\ACE}  to  print  the
first 12 lines and include coset  representatives.  Note  that,  since
there are no relators, the table has separate  columns  for  generator
inverses. So the default workspace of $1000000$ words allows  a  table
of $249998 = 1000000/4 - 2$ cosets. Since row  `fill'ing  (see~"option
fill") is on by default, the table is simply  filled  with  cosets  in
order. Note that a compaction phase is done before printing the table,
but that this does nothing here (the lowercase `co' tag), since  there
are no dead cosets. The coset representatives are simply all  possible
freely reduced words, in length plus lexicographic (i.e. `lenlex'; see
Section~"Coset Table Standardisation Schemes") order.

%%%%%%%%%%%%%%%%%%%%%%%%%%%%%%%%%%%%%%%%%%%%%%%%%%%%%%%%%%%%%%%%%%%%%%
*Using `ACECosetTableFromGensAndRels'*
{\undoquotes\atindex{ACECosetTableFromGensAndRels, example}
{@`ACECosetTableFromGensAndRels', example}}

The  essential  difference  between  the  functions   `ACEStats'   and
`ACECosetTableFromGensAndRels' is that `ACEStats' parses the  *results
message* from the {\ACE} binary and outputs a {\GAP} record containing
statistics  of  the  enumeration,  and  `ACECosetTableFromGensAndRels'
after parsing the *results message*, goes on to parse  {\ACE}'s  coset
table, if it can, and outputs a {\GAP} list of lists version  of  that
table. So, if we had used  `ACECosetTableFromGensAndRels'  instead  of
`ACEStats' in our examples  above,  we  would  have  observed  similar
output, except that we would have ended up in a `break'-loop  (because
the enumeration overflows) instead of obtaining  a  record  containing
enumeration statistics. We have already seen an  example  of  that  in
Section~"Using ACE Directly to Generate a Coset Table".  So,  here  we
will consider two options that prevent one  entering  a  `break'-loop,
namely   the   `silent'   (see~"option   silent")   and   `incomplete'
(see~"option  incomplete")  options.  Firstly,  let's  take  the  last
`ACEStats' example, but use `ACECosetTableFromGensAndRels' instead and
include the `silent' option. (We still have the `InfoACE' level set at
3.)

\beginexample
gap> ACECosetTableFromGensAndRels(fgens, [], [] : start, print := [-1, 12],
>                                                 silent);
#I  ACE 3.000        Thu Aug 17 11:18:09 2000
#I  =========================================
#I  Host information:
#I    name = boronia
#I  OVERFLOW (a=249998 r=83333 h=83333 n=249999; l=337 c=0.20; m=249998 t=2499\
98)
#I  co: a=249998 r=83333 h=83333 n=249999; c=+0.00
#I   coset ||      a      A      b      B   order   rep've
#I  -------+---------------------------------------------
#I       1 ||      2      3      4      5
#I       2 ||      6      1      7      8       0   a
#I       3 ||      1      9     10     11       0   A
#I       4 ||     12     13     14      1       0   b
#I       5 ||     15     16      1     17       0   B
#I       6 ||     18      2     19     20       0   aa
#I       7 ||     21     22     23      2       0   ab
#I       8 ||     24     25      2     26       0   aB
#I       9 ||      3     27     28     29       0   AA
#I      10 ||     30     31     32      3       0   Ab
#I      11 ||     33     34      3     35       0   AB
#I      12 ||     36      4     37     38       0   ba
#I  ***
fail
\endexample

Since, the enumeration overflowed and the  `silent'  option  was  set,
`ACECosetTableFromGensAndRels' simply returned `fail'.  But  hang  on,
{\ACE} at least has a partial table; we should be able to get that  in
{\GAP},  in  a  situation  like  this.  We  can.  We  simply  use  the
`incomplete' option, instead of the `silent' option.  However,  if  we
did that with the example above, the result would be an enormous table
(the number of *active cosets* is 249998); so  let  us  also  set  the
`max' (see~"option max") option, in order that we should  get  a  more
modestly sized partial table:

\beginexample
gap> ACECosetTableFromGensAndRels(fgens, [], [] : max := 12, 
>                                                 start, print := [1, -12],
>                                                 incomplete);                 
#I  ACE 3.000        Thu Aug 17 11:39:25 2000
#I  =========================================
#I  Host information:
#I    name = boronia
#I  OVERFLOW (a=12 r=4 h=4 n=13; l=5 c=0.00; m=12 t=12)
#I  co: a=12 r=4 h=4 n=13; c=+0.00
#I   coset ||      a      A      b      B   order   rep've
#I  -------+---------------------------------------------
#I       1 ||      2      3      4      5
#I       2 ||      6      1      7      8       0   a
#I       3 ||      1      9     10     11       0   A
#I       4 ||     12      0      0      1       0   b
#I       5 ||      0      0      1      0       0   B
#I       6 ||      0      2      0      0       0   aa
#I       7 ||      0      0      0      2       0   ab
#I       8 ||      0      0      2      0       0   aB
#I       9 ||      3      0      0      0       0   AA
#I      10 ||      0      0      0      3       0   Ab
#I      11 ||      0      0      3      0       0   AB
#I      12 ||      0      4      0      0       0   ba
#I  ***
#I  co: a=12 r=4 h=4 n=13; c=+0.00
#I   coset ||      a      A      b      B
#I  -------+----------------------------
#I       1 ||      2      3      4      5
#I       2 ||      6      1      7      8
#I       3 ||      1      9     10     11
#I       4 ||     12      0      0      1
#I       5 ||      0      0      1      0
#I       6 ||      0      2      0      0
#I       7 ||      0      0      0      2
#I       8 ||      0      0      2      0
#I       9 ||      3      0      0      0
#I      10 ||      0      0      0      3
#I      11 ||      0      0      3      0
#I      12 ||      0      4      0      0
#I  ACECosetTable: Warning: Coset table is incomplete.
[ [ 2, 6, 1, 12, 0, 0, 0, 0, 3, 0, 0, 0 ], 
  [ 3, 1, 9, 0, 0, 2, 0, 0, 0, 0, 0, 4 ], 
  [ 4, 7, 10, 0, 1, 0, 0, 2, 0, 0, 3, 0 ], 
  [ 5, 8, 11, 1, 0, 0, 2, 0, 0, 3, 0, 0 ] ]
\endexample

Observe, that despite the fact that {\ACE} is  able  to  define  coset
representatives for all 12 coset numbers  defined,  the  body  of  the
coset table now contains a 0 at each  place  formerly  occupied  by  a
coset number larger  than  12  (0  essentially  represents  \lq{}don't
know'). To get a table that is the same in the first 12 rows as before
we would have had to set `max' to 38, since that was the largest coset
number that appeared in the body of  the  12-line  table,  previously.
Also, note that the `max' option *preceded* the `start' option;  since
the interface respects the order in which options are put by the user,
the enumeration invoked by `start'  would  otherwise  have  only  been
restricted by the size of `workspace'  (see~"option  workspace").  The
warning that the coset table is incomplete is emitted at `InfoACE'  or
`InfoWarning' level 1, i.e.~by default, you will see it.

%%%%%%%%%%%%%%%%%%%%%%%%%%%%%%%%%%%%%%%%%%%%%%%%%%%%%%%%%%%%%%%%%%%%%%
*Using `ACEStart'*{\undoquotes\atindex{ACEStart, example}{@`ACEStart',
example}}

The     limitation     of     the     functions     `ACEStats'     and
`ACECosetTableFromGensAndRels' (on three arguments) is  that  they  do
not interact with {\ACE}; they call {\ACE}  with  user-defined  input,
and collect and parse the output for  either  statistics  or  a  coset
table. On the other hand,  the  `ACEStart'  (see~"ACEStart")  function
allows one to start up an {\ACE} process and maintain a dialogue  with
it. Moreover, via the functions `ACEStats' and `ACECosetTable'  (on  1
or no arguments), one is able to extract the same information that  we
could with the non-interactive versions of these  functions.  However,
we can also do a lot more. Each {\ACE}  option  that  provides  output
that can be used from within {\GAP} has  a  corresponding  interactive
interface function that parses and translates that output into a  form
usable from within {\GAP}.

Now we emulate our  (successful)  `ACEStats'  exchanges  above,  using
interactive  {\ACE}  interface  functions.  We  could  do  this  with:
`ACEStart(0, fgens, [], [] : start, sr := 1);'  where  the  `0'  first
argument tells `ACEStart' not to  insert  `start'  after  the  options
explicitly listed. Alternatively, we may do the following  (note  that
the `InfoACE' level is still 3):

\beginexample
gap> ACEStart( fgens, [], [] );                       
#I  ACE 3.000        Tue Aug 22 14:42:13 2000
#I  =========================================
#I  Host information:
#I    name = boronia
#I  ***
#I  OVERFLOW (a=249998 r=83333 h=83333 n=249999; l=337 c=0.24; m=249998 t=2499\
98)
1
gap> ACEParameters(1);
#I    #-- ACE 3.000: Run Parameters ---
#I  Group Name: G;
#I  Group Generators: ab;
#I  Group Relators: ;
#I  Subgroup Name: H;
#I  Subgroup Generators: ;
#I  Wo:1000000; Max:249998; Mess:0; Ti:-1; Ho:-1; Loop:0;
#I  As:0; Path:0; Row:1; Mend:0; No:0; Look:0; Com:10;
#I  C:0; R:0; Fi:7; PMod:3; PSiz:256; DMod:4; DSiz:1000;
#I    #--------------------------------
rec( enumeration := "G", subgroup := "H", workspace := 1000000, 
  max := 249998, messages := 0, time := -1, hole := -1, loop := 0, asis := 0, 
  path := 0, row := 1, mendelsohn := 0, no := 0, lookahead := 0, 
  compaction := 10, ct := 0, rt := 0, fill := 7, pmode := 3, psize := 256, 
  dmode := 4, dsize := 1000 )
\endexample

Observe that the `ACEStart' call returned an integer (1, here). All  8
forms of the `ACEStart' function, return an  integer  that  identifies
the interactive {\ACE} interface function  initiated  or  communicated
with. We may use this integer to tell any interactive {\ACE} interface
function which interactive {\ACE} process we wish to communicate with.
Above we passed `1' to the `ACEParameters' command which caused `sr :=
1' (see~"option sr") to be passed to the  interactive  {\ACE}  process
indexed by 1 (the process we just started), and  a  record  containing
the  parameter  options  (see  Section~"ACE  Parameter  Options")   is
returned. Note that  the  \lq{}Run  Parameters':  `Group  Generators',
`Group Relators' and `Subgroup Generators' are  considered  \lq{}args'
(i.e.~arguments) and a record containing  these  is  returned  by  the
`GetACEArgs' (see~"GetACEArgs")  command;  or  they  may  be  obtained
individually     via      the      commands:      `ACEGroupGenerators'
(see~"ACEGroupGenerators"),  `ACERelators'   (see~"ACERelators"),   or
`ACESubgroupGenerators' (see~"ACESubgroupGenerators").

To display 12 lines of the  coset  table  with  coset  representatives
without invoking a further enumeration we could do: `ACEStart(0,  1  :
print   :=   [-1,   12]);'.   Alternatively,   we    may    use    the
`ACEDisplayCosetTable' (see~"ACEDisplayCosetTable") (the table  itself
is emitted at `InfoACE' level 1, since by default we  presumably  want
to see it):

\beginexample
gap> ACEDisplayCosetTable(1, [-1, 12]);
#I  co: a=249998 r=83333 h=83333 n=249999; c=+0.00
#I   coset ||      a      A      b      B   order   rep've
#I  -------+---------------------------------------------
#I       1 ||      2      3      4      5
#I       2 ||      6      1      7      8       0   a
#I       3 ||      1      9     10     11       0   A
#I       4 ||     12     13     14      1       0   b
#I       5 ||     15     16      1     17       0   B
#I       6 ||     18      2     19     20       0   aa
#I       7 ||     21     22     23      2       0   ab
#I       8 ||     24     25      2     26       0   aB
#I       9 ||      3     27     28     29       0   AA
#I      10 ||     30     31     32      3       0   Ab
#I      11 ||     33     34      3     35       0   AB
#I      12 ||     36      4     37     38       0   ba
#I  ------------------------------------------------------------
\endexample

Now we obtain the enumeration statistics record, via  the  interactive
version of `ACEStats' (see~"ACEStatsI") :

\beginexample
gap> ACEStats(1); # The interactive version of ACEStats takes 1 or no arguments
rec( index := 0, cputime := 22, cputimeUnits := "10^-2 seconds", 
  activecosets := 249998, maxcosets := 249998, totcosets := 249998 )
\endexample

Still with the same interactive {\ACE} process we can now emulate  the
`ACECosetTableFromGensAndRels' exchange that  gave  us  an  incomplete
coset table. Note that it is still necessary to invoke an  enumeration
after setting  the  `max'  (see~"option  max").  We  could  just  call
`ACECosetTable' with the argument 1 and the same 4 options we used for
`ACECosetTableFromGensAndRels'.  Alternatively,  we   could   do   the
following:

\beginexample
gap> ACEStart(1 : max := 12);                                           
#I  ***
#I  OVERFLOW (a=12 r=4 h=4 n=13; l=5 c=0.00; m=12 t=12)
1
gap> ACEDisplayCosetTable(1, [-1, 12]);
#I  co: a=12 r=4 h=4 n=13; c=+0.00
#I   coset ||      a      A      b      B   order   rep've
#I  -------+---------------------------------------------
#I       1 ||      2      3      4      5
#I       2 ||      6      1      7      8       0   a
#I       3 ||      1      9     10     11       0   A
#I       4 ||     12      0      0      1       0   b
#I       5 ||      0      0      1      0       0   B
#I       6 ||      0      2      0      0       0   aa
#I       7 ||      0      0      0      2       0   ab
#I       8 ||      0      0      2      0       0   aB
#I       9 ||      3      0      0      0       0   AA
#I      10 ||      0      0      0      3       0   Ab
#I      11 ||      0      0      3      0       0   AB
#I      12 ||      0      4      0      0       0   ba
#I  ------------------------------------------------------------
\endexample
\beginexample
gap> ACECosetTable(:incomplete);
#I  start = yes, continue = yes, redo = yes
#I  ***
#I  OVERFLOW (a=12 r=4 h=4 n=13; l=4 c=0.00; m=12 t=12)
#I  co: a=12 r=4 h=4 n=13; c=+0.00
#I   coset ||      a      A      b      B
#I  -------+----------------------------
#I       1 ||      2      3      4      5
#I       2 ||      6      1      7      8
#I       3 ||      1      9     10     11
#I       4 ||     12      0      0      1
#I       5 ||      0      0      1      0
#I       6 ||      0      2      0      0
#I       7 ||      0      0      0      2
#I       8 ||      0      0      2      0
#I       9 ||      3      0      0      0
#I      10 ||      0      0      0      3
#I      11 ||      0      0      3      0
#I      12 ||      0      4      0      0
#I  ACECosetTable: Warning: Coset table is incomplete.
[ [ 2, 6, 1, 12, 0, 0, 0, 0, 3, 0, 0, 0 ],
  [ 3, 1, 9, 0, 0, 2, 0, 0, 0, 0, 0, 4 ], 
  [ 4, 7, 10, 0, 1, 0, 0, 2, 0, 0, 3, 0 ], 
  [ 5, 8, 11, 1, 0, 0, 2, 0, 0, 3, 0, 0 ] ]
\endexample

The `ACEStart' command (without `0'  as  first  argument)  inserted  a
`start'  directive   after   the   user   option   `max'.   Then   the
`ACEDisplayCosetTable' command did the equivalent  of  the  `print  :=
[-1, 12]' option. Finally, we called `ACECosetTable' with  1  argument
(thus       invoking        the         interactive         version of
`ACECosetTableFromGensAndRels') and the option  `incomplete'.  Observe
the line beginning \lq{}`\#I  start = yes,'' (the first  line  in  the
output of `ACECosetTable'). This  line  appears  in  response  to  the
option `mode' (see~"option mode") inserted  by  `ACECosetTable'  after
any user options; it is inserted  in  order  to  check  that  no  user
options  (possibly  made  before  the   `ACECosetTable'   call)   have
invalidated {\ACE}'s coset table. Since the line also says `continue =
yes', the mode `continue' (the least expensive  of  the  three  modes;
see~"option continue") is directed at {\ACE} which evokes  a  *results
message*. Then `ACECosetTable' extracts the  incomplete  table  via  a
`print' (see "option print") directive. If you wish  to  see  all  the
options that are directed to {\ACE}, set  the  `InfoACE'  level  to  4
(then all such  commands  are  `Info'-ed   behind  a   \lq{}`ToACE> ''
prompt; see~"SetInfoACELevel").

We now set things up to do the alternating group $A_5$, of order $60$.
We turn  messaging  on  via  the  `messages'  (see~"option  messages")
option, but set the interval high enough so that we will only get  the
\lq{}Run Parameters' block (the same as that obtained with `sr :=  1')
and no progress messages. Recall `F' is the free group we  defined  on
generators `fgens': `"a"' and `"b"'. Note,  that  when  {\GAP}  prints
`F.1' ($={}$`fgens[1]') it displays `a', but the *variable* `a' is (at
the moment) unassigned; so for convenience (in defining relators,  for
example) we first assign the variable `a' to  be  `F.1'  (and  `b'  to
be `F.2').

\beginexample
gap> a := F.1;; b := F.2;;
gap> # Enumerating A_5 = < a, b | a^2, b^3, (a*b)^5 > over Id (trivial subgp)
gap> ACEStart(1, fgens, [a^2, b^3, (a*b)^5], [] # last arg empty (to define Id)
>             : enumeration := "A_5", # Define the Group Name
>               subgroup := "Id",     # Define the Subgroup Name
>               max := 0,             # Set `max' back to default (no limit)
>               messages := 1000);    # Progress messages every 1000 iterations
#I  ***
#I  ***
#I    #-- ACE 3.000: Run Parameters ---
#I  Group Name: A_5;
#I  Group Generators: ab;
#I  Group Relators: (a)^2, (b)^3, (ab)^5;
#I  Subgroup Name: Id;
#I  Subgroup Generators: ;
#I  Wo:1000000; Max:333331; Mess:1000; Ti:-1; Ho:-1; Loop:0;
#I  As:0; Path:0; Row:1; Mend:0; No:3; Look:0; Com:10;
#I  C:0; R:0; Fi:6; PMod:3; PSiz:256; DMod:4; DSiz:1000;
#I    #--------------------------------
#I  INDEX = 60 (a=60 r=77 h=1 n=77; l=3 c=0.00; m=66 t=76)
1
\endexample

Since we  did  not  specify  a  strategy  the  `default'  (see~"option
default") strategy was followed and  hence  coset  number  definitions
were R (i.e.~HLT) style, and a total of $76$  coset  numbers  (`t=76')
were  defined.  Note,  that  {\ACE}  already  \lq{}knew'   the   group
generators and subgroup generators, using the `relators'  (see~"option
relators") option:

\beginexample
gap> ACEStart(1 : relators := ToACEWords(fgens, [a^2, b^3, (a*b)^5]),
>                 enumeration := "A_5",                                        
>                 subgroup := "Id",                                            
>                 max := 0,                                                    
>                 messages := 1000);                                           
#I  Detected usage of a synonym of one (or more) of the options:
#I      `group', `relators', `generators'.
#I  Discarding current values of args.
#I  (The new args will be extracted from ACE, later).
#I  No group generators saved. Setting value(s) from ACE ...
#I    #-- ACE 3.000: Run Parameters ---
#I  Group Name: A_5;
#I  Group Generators: ab;
#I  Group Relators: (a)^2, bbb, ababababab;
#I  Subgroup Name: Id;
#I  Subgroup Generators: ;
#I  Wo:1000000; Max:333331; Mess:1000; Ti:-1; Ho:-1; Loop:0;
#I  As:0; Path:0; Row:1; Mend:0; No:3; Look:0; Com:10;
#I  C:0; R:0; Fi:6; PMod:3; PSiz:256; DMod:4; DSiz:1000;
#I    #--------------------------------
#I  ***
#I  ToACE> Start;
#I    #-- ACE 3.000: Run Parameters ---
#I  Group Name: A_5;
#I  Group Generators: ab;
#I  Group Relators: (a)^2, (b)^3, (ab)^5;
#I  Subgroup Name: Id;
#I  Subgroup Generators: ;
#I  Wo:1000000; Max:333331; Mess:1000; Ti:-1; Ho:-1; Loop:0;
#I  As:0; Path:0; Row:1; Mend:0; No:3; Look:0; Com:10;
#I  C:0; R:0; Fi:6; PMod:3; PSiz:256; DMod:4; DSiz:1000;
#I    #--------------------------------
#I  INDEX = 60 (a=60 r=77 h=1 n=77; l=3 c=0.00; m=66 t=76)
1
\endexample

Note the usage  of  `ToACEWords'  (see~"ToACEWords")  to  provide  the
appropriate string value of the `relators' option. Also,  observe  the
`Info'-ed warning of the action  triggered  by  using  the  `relators'
option, that says that the current values of the \lq{}args' (i.e.~what
would be returned by `GetACEArgs'; see~"GetACEArgs")  were  discarded,
which immediately triggered the action of reinstantiating the value of
`ACEData.io[1].args' (which is what the `Info':

\beginexample
#I  No group generators saved. Setting value(s) from ACE ...
\endexample

is all about). Also observe that the  \lq{}Run  Parameters'  block  is
`Info'-ed twice; the first time was due to  `ACEStart'  emitting  `sr'
with value `1' to {\ACE} ({\ACE}'s response is used to  re-instantiate
`ACEData.io[1].args'). 

We will now define a non-trivial subgroup, and monitor all the actions
of the enumerator, by setting `messages := 1'.  As  for  defining  the
relators, we could use the 1-argument version of `ACEStart', in  which
case we would use the `subgroup' (see~"option subgroup")  option  with
the value `ToACEWords(fgens, [ a*b ])'. Instead, we will use the  more
convenient 4-argument form:

\beginexample
gap> ACEStart(1, ACEGroupGenerators(1), ACERelators(1), [ a*b ] 
gap>          : messages := 1);              
#I  ***
#I  ***
#I    #-- ACE 3.000: Run Parameters ---
#I  Group Name: A_5;
#I  Group Generators: ab;
#I  Group Relators: (a)^2, (b)^3, (ab)^5;
#I  Subgroup Name: Id;
#I  Subgroup Generators: ab;
#I  Wo:1000000; Max:333331; Mess:1; Ti:-1; Ho:-1; Loop:0;
#I  As:0; Path:0; Row:1; Mend:0; No:3; Look:0; Com:10;
#I  C:0; R:0; Fi:6; PMod:3; PSiz:256; DMod:4; DSiz:1000;
#I    #--------------------------------
#I  AD: a=2 r=1 h=1 n=3; l=1 c=+0.00; m=2 t=2
#I  SG: a=2 r=1 h=1 n=3; l=1 c=+0.00; m=2 t=2
#I  RD: a=3 r=1 h=1 n=4; l=2 c=+0.00; m=3 t=3
#I  RD: a=4 r=2 h=1 n=5; l=2 c=+0.00; m=4 t=4
#I  RD: a=5 r=2 h=1 n=6; l=2 c=+0.00; m=5 t=5
#I  RD: a=6 r=2 h=1 n=7; l=2 c=+0.00; m=6 t=6
#I  RD: a=7 r=2 h=1 n=8; l=2 c=+0.00; m=7 t=7
#I  RD: a=8 r=2 h=1 n=9; l=2 c=+0.00; m=8 t=8
#I  RD: a=9 r=2 h=1 n=10; l=2 c=+0.00; m=9 t=9
#I  CC: a=8 r=2 h=1 n=10; l=2 c=+0.00; d=0
#I  RD: a=9 r=5 h=1 n=11; l=2 c=+0.00; m=9 t=10
#I  RD: a=10 r=5 h=1 n=12; l=2 c=+0.00; m=10 t=11
#I  RD: a=11 r=5 h=1 n=13; l=2 c=+0.00; m=11 t=12
#I  RD: a=12 r=5 h=1 n=14; l=2 c=+0.00; m=12 t=13
#I  RD: a=13 r=5 h=1 n=15; l=2 c=+0.00; m=13 t=14
#I  RD: a=14 r=5 h=1 n=16; l=2 c=+0.00; m=14 t=15
#I  CC: a=13 r=6 h=1 n=16; l=2 c=+0.00; d=0
#I  CC: a=12 r=6 h=1 n=16; l=2 c=+0.00; d=0
#I  INDEX = 12 (a=12 r=16 h=1 n=16; l=3 c=0.00; m=14 t=15)
1
\endexample

Observe that we used `ACERelators(1)' (see~"ACERelators") to grab  the
value  of  the  relators  we  had  defined  earlier.  We   also   used
`ACEGroupGenerators(1)' (see~"ACEGroupGenerators") to  get  the  group
generators (`fgens' would do instead, except if, as  in  our  sequence
above, the `relators' option was used, since  this  causes  {\GAP}  to
lose  the  knowledge  that  `fgens'  and  `ACEGroupGenerators(1)'  are
equal).

The  run  ended  with  12  active   (see   Section~"Coset   Statistics
Terminology") coset numbers (`a=12') after defining a total number  of
15 coset numbers (`t=15'). So there must  have  been  3  coincidences:
observe  that  there  were  3  progress  messages  (see  Appendix~"The
Meanings of ACE's Output Messages") with a `CC:' tag.

We now dump out the  statistics  accumulated  during  the  run,  using
`ACEDumpStatistics'  (see~"ACEDumpStatistics"),  which   `Info's   the
{\ACE}  output  of  the  `statistics'  (see~"option  statistics")   at
`InfoACE' level 1.

\beginexample
gap> ACEDumpStatistics(1);
#I    #- ACE 3.000: Level 0 Statistics --
#I  cdcoinc=0 rdcoinc=2 apcoinc=0 rlcoinc=0 clcoinc=0
#I    xcoinc=2 xcols12=4 qcoinc=3
#I    xsave12=0 s12dup=0 s12new=0
#I    xcrep=6 crepred=0 crepwrk=0 xcomp=0 compwrk=0
#I  xsaved=0 sdmax=0 sdoflow=0
#I  xapply=1 apdedn=1 apdefn=1
#I  rldedn=0 cldedn=0
#I  xrdefn=1 rddedn=5 rddefn=13 rdfill=0
#I  xcdefn=0 cddproc=0 cdddedn=0 cddedn=0
#I    cdgap=0 cdidefn=0 cdidedn=0 cdpdl=0 cdpof=0
#I    cdpdead=0 cdpdefn=0 cddefn=0
#I    #----------------------------------
\endexample

The statistic `qcoinc=3' states what we had already observed,  namely,
that there  were  three  coincidences.  Of  these,  two  were  primary
coincidences  (`rdcoinc=2').  Since  `t=15',   there   were   fourteen
non-trivial coset number definitions; one was during  the  application
of coset 1 to the subgroup generator (`apdefn=1'), and  the  remainder
occurred during applications of the  coset  numbers  to  the  relators
(`rddefn=13'). For more details on the meanings of the  variables  you
will need to read the C code comments.

%Note how the pre-printout compaction phase now  does  some  work  (the
%upper-case `CO' tag), since there were coincidences,  and  hence  dead
%cosets. Note how `b/B' have been used as the first two columns,  since
%these must be occupied by  a  generator/inverse  pair  or  a  pair  of
%involutions. The `a' column is also the  `A'  column,  as  `a'  is  an
%involution.
%
%\begintt
%  print  TABLE : -1, 12 ;
%CO: a=12 r=13 h=1 n=13; c=+0.00
% coset |      b      B      a   order   rep've
%-------+--------------------------------------
%     1 |      3      2      2
%     2 |      1      3      1       3   B
%     3 |      2      1      4       3   b
%     4 |      8      5      3       5   ba
%     5 |      4      8      6       2   baB
%     6 |      9      7      5       5   baBa
%     7 |      6      9      8       3   baBaB
%     8 |      5      4      7       5   bab
%     9 |      7      6     10       5   baBab
%    10 |     12     11      9       3   baBaba
%    11 |     10     12     12       2   baBabaB
%    12 |     11     10     11       3   baBabab
%\endtt
%
%If we define the generator order to be that of the columns,  then  the
%table above is not in canonic form, and the coset representatives  are
%not in order. We now standardise the table; note the compaction  phase
%before standardisation, although it does nothing  in  this  particular
%case. Note how, if we read through the table in row-major  order,  new
%cosets are introduced using the smallest available  number,  and  that
%the representatives are now in order and are minimal.
%
%\begintt
%st;
%co/ST: a=12 r=13 h=1 n=13; c=+0.00
%pr:-1,12;
%co: a=12 r=13 h=1 n=13; c=+0.00
% coset |      b      B      a   order   rep've
%-------+--------------------------------------
%     1 |      2      3      3
%     2 |      3      1      4       3   b
%     3 |      1      2      1       3   B
%     4 |      5      6      2       5   ba
%     5 |      6      4      7       5   bab
%     6 |      4      5      8       2   baB
%     7 |      8      9      5       5   baba
%     8 |      9      7      6       5   baBa
%     9 |      7      8     10       3   babaB
%    10 |     11     12      9       3   babaBa
%    11 |     12     10     12       3   babaBab
%    12 |     10     11     11       2   babaBaB
%\endtt
%
%%%%%%%%%%%%%%%%%%%%%%%%%%%%%%%%%%%%%%%%%%%%%%%%%%%%%%%%%%%%%%%%%%%%%%
\Section{Example where ACE is made the Standard Coset Enumerator}

If {\ACE} is made the standard coset enumerator, one simply  uses  the
method of passing arguments normally used  with  those  commands  that
invoke `CosetTableFromGensAndRels', but one is able to use all options
available via the {\ACE} interface. As an example  we  use  {\ACE}  to
compute the permutation representation of a  perfect  group  from  the
data library (in this case, an automorphic  extension  of  the  simple
alternating group $A_5$):

\beginexample
gap> SetInfoACELevel(3); # Just to see what's going on behind the scenes
gap> TCENUM:=ACETCENUM;; # Make ACE the standard coset enumerator
gap> G := PerfectGroup(IsPermGroup, 16*60, 1   # Arguments ... as per usual
>                      : max := 50, mess := 10 # ... but we use ACE options
>                      );
#I  ACE 3.000        Mon Aug 14 15:50:09 2000
#I  =========================================
#I  Host information:
#I    name = boronia
#I    #-- ACE 3.000: Run Parameters ---
#I  Group Name: G;
#I  Group Generators: abstuv;
#I  Group Relators: (a)^2, (s)^2, (t)^2, (u)^2, (v)^2, (b)^3, (st)^2, (uv)^2, 
#I    (su)^2, (sv)^2, (tu)^2, (tv)^2, asau, atav, auas, avat, Bvbu, Bsbvt, 
#I    Bubvu, Btbvuts, (ab)^5;
#I  Subgroup Name: H;
#I  Subgroup Generators: a, b;
#I  Wo:1000000; Max:50; Mess:10; Ti:-1; Ho:-1; Loop:0;
#I  As:0; Path:0; Row:1; Mend:0; No:21; Look:0; Com:10;
#I  C:0; R:0; Fi:11; PMod:3; PSiz:256; DMod:4; DSiz:1000;
#I    #--------------------------------
#I  SG: a=1 r=1 h=1 n=2; l=1 c=+0.00; m=1 t=1
#I  RD: a=11 r=1 h=1 n=12; l=2 c=+0.00; m=11 t=11
#I  RD: a=21 r=2 h=1 n=22; l=2 c=+0.00; m=21 t=21
#I  CC: a=29 r=4 h=1 n=31; l=2 c=+0.00; d=0
#I  CC: a=19 r=4 h=1 n=31; l=2 c=+0.00; d=0
#I  CC: a=19 r=6 h=1 n=36; l=2 c=+0.00; d=0
#I  INDEX = 16 (a=16 r=36 h=1 n=36; l=3 c=0.00; m=30 t=35)
#I  ***
#I  CO: a=16 r=17 h=1 n=17; c=+0.00
\endexample
\beginexample
#I   coset ||      b      B      a      s      t      u      v
#I  -------+-------------------------------------------------
#I       1 ||      1      1      1      2      3      4      5
#I       2 ||     11     14      4      1      6      8      9
#I       3 ||     13     15      5      6      1     10     11
#I       4 ||      7      5      2      8     10      1      7
#I       5 ||      4      7      3      9     11      7      1
#I       6 ||      8     10      7      3      2     12     14
#I       7 ||      5      4      6     15     16      5      4
#I       8 ||     10      6      8      4     12      2     15
#I       9 ||     16     12     10      5     14     15      2
#I      10 ||      6      8      9     12      4      3     16
#I      11 ||     14      2     11     14      5     16      3
#I      12 ||      9     16     15     10      8      6     13
#I      13 ||     15      3     13     16     15     14     12
#I      14 ||      2     11     16     11      9     13      6
#I      15 ||      3     13     12      7     13      9      8
#I      16 ||     12      9     14     13      7     11     10
A5 2^4
gap> GeneratorsOfGroup(G); # Just to show we indeed have a perm'n rep'n
[ ( 2, 4)( 3, 5)( 7,12)( 9,11)(13,14)(15,16), 
  ( 2, 6,13)( 3,10,16)( 4,12, 5)( 7, 8,11)( 9,14,15), 
  ( 1, 2)( 3, 7)( 4, 8)( 5, 9)( 6,13)(10,14)(11,15)(12,16), 
  ( 1, 3)( 2, 7)( 4,11)( 5, 6)( 8,15)( 9,13)(10,16)(12,14), 
  ( 1, 4)( 2, 8)( 3,11)( 5,12)( 6,14)( 7,15)( 9,16)(10,13), 
  ( 1, 5)( 2, 9)( 3, 6)( 4,12)( 7,13)( 8,16)(10,15)(11,14) ]
gap> Order(G);
960
\endexample

%%%%%%%%%%%%%%%%%%%%%%%%%%%%%%%%%%%%%%%%%%%%%%%%%%%%%%%%%%%%%%%%%%%%%%
\Section{Example of Using ACECosetTableFromGensAndRels}

The following example calls {\ACE} for up to 800 coset  numbers  using
Mendelsohn style relator processing and sets the message level to 500.
The value of `table', i.e.~the {\GAP} coset table, immediately follows
the last {\ACE} message (\lq{}`\#I '') line, but both the coset  table
from {\ACE} and the  {\GAP}  coset  table  have  been  abbreviated.  A
slightly modified version of this example, which includes  the  `echo'
option is available on-line via `table  :=  ACEExample("perf602p5");'.
You may wish to peruse the notes  in  the  `ACEExample'  index  first,
however, by executing `ACEExample();'.

\beginexample
gap> SetInfoACELevel(3);
gap> G := PerfectGroup(2^5*60, 2);;
gap> fgens := FreeGeneratorsOfFpGroup(G);;
gap> table := ACECosetTableFromGensAndRels(
>                 # arguments
>                 fgens, RelatorsOfFpGroup(G), fgens{[1]}
>                 # options
>                 : mendelsohn, max:=800, mess:=500);
#I  ACE 3.000        Mon Aug 14 15:56:42 2000
#I  =========================================
#I  Host information:
#I    name = boronia
#I    #-- ACE 3.000: Run Parameters ---
#I  Group Name: G;
#I  Group Generators: abstuvd;
#I  Group Relators: (s)^2, (t)^2, (u)^2, (v)^2, (d)^2, aad, (b)^3, (st)^2, 
#I    (uv)^2, (su)^2, (sv)^2, (tu)^2, (tv)^2, Asau, Atav, Auas, Avat, Bvbu, 
#I    dAda, dBdb, (ds)^2, (dt)^2, (du)^2, (dv)^2, Bubvu, Bsbdvt, Btbvuts, 
#I    (ab)^5;
#I  Subgroup Name: H;
#I  Subgroup Generators: a;
#I  Wo:1000000; Max:800; Mess:500; Ti:-1; Ho:-1; Loop:0;
#I  As:0; Path:0; Row:1; Mend:1; No:28; Look:0; Com:10;
#I  C:0; R:0; Fi:13; PMod:3; PSiz:256; DMod:4; DSiz:1000;
#I    #--------------------------------
#I  SG: a=1 r=1 h=1 n=2; l=1 c=+0.00; m=1 t=1
#I  RD: a=321 r=68 h=1 n=412; l=5 c=+0.01; m=327 t=411
#I  CC: a=435 r=162 h=1 n=719; l=9 c=+0.01; d=0
#I  CL: a=428 r=227 h=1 n=801; l=13 c=+0.04; m=473 t=800
#I  DD: a=428 r=227 h=1 n=801; l=14 c=+0.00; d=534
#I  DD: a=428 r=227 h=1 n=801; l=14 c=+0.01; d=32
#I  CO: a=428 r=192 h=243 n=429; l=15 c=+0.00; m=473 t=800
#I  INDEX = 480 (a=480 r=210 h=484 n=484; l=18 c=0.08; m=480 t=855)
#I  ***
#I  CO: a=480 r=210 h=481 n=481; c=+0.00
#I   coset ||      a      A      b      B      s      t      u      v      d
#I  -------+---------------------------------------------------------------
#I       1 ||      1      1      7      6      2      3      4      5      1
#I       2 ||      4      4     22     36      1      8     10     11      2
#I       3 ||      5      5     30     23      8      1     12     13      3
#I       4 ||      2      2     17     14     10     12      1      9      4
#I       5 ||      3      3     15     19     11     13      9      1      5
... 470 lines omitted here ...
#I     476 ||    469    469    407    406    478    472    466    480    476
#I     477 ||    477    477    314    313    474    471    467    468    477
#I     478 ||    473    473    380    379    476    467    471    470    478
#I     479 ||    479    479    384    383    475    468    470    471    479
#I     480 ||    480    480    421    420    470    469    475    476    480
[ [ 1, 7, 5, 6, 3, 4, 2, 27, 25, 26, 23, 24, 39, 21, 15, 18, 46, 16, 19, 51, 
      14, 52, 11, 12, 9, 10, 8, 68, 69, 66, 67, 75, 64, 59, 62, 77, 60, 79, 
... 30 lines omitted here ...
      478, 476, 441, 475, 473, 480, 472, 471, 477 ],
[ [ 1, 7, 5, 6, 3, 4, 2, 27, 25, 26, 23, 24, 39, 21, 15, 18, 46, 16, 19, 51, 
      14, 52, 11, 12, 9, 10, 8, 68, 69, 66, 67, 75, 64, 59, 62, 77, 60, 79, 
      478, 476, 441, 475, 473, 480, 472, 471, 477 ], 
... 396 lines omitted here ...
  [ 1, 2, 3, 4, 5, 6, 7, 8, 9, 10, 11, 12, 13, 14, 15, 16, 17, 18, 19, 20, 
      21, 22, 23, 24, 25, 26, 27, 28, 29, 30, 31, 32, 33, 34, 35, 36, 37, 38, 
... 30 lines omitted here ...
      472, 473, 474, 475, 476, 477, 478, 479, 480 ] ]
\endexample

%%%%%%%%%%%%%%%%%%%%%%%%%%%%%%%%%%%%%%%%%%%%%%%%%%%%%%%%%%%%%%%%%%%%%%
\Section{Example of Using ACE Interactively (Using ACEStart)}

Now we illustrate a simple interactive process, with an enumeration of
an index 12 subgroup (isomorphic to $C_5$) within $A_5$. Observe  that
we  have  relied  on  the  default  level  of  messaging  from  {\ACE}
(`messages' = 0) which gives a result  line  only,  without  parameter
information. We have however used the option `echo', so  that  we  can
see how the interface handled the arguments and options. On-line  try:
`SetInfoACELevel(3); ACEExample("A5-C5", ACEStart);'  to  emulate  the
session prior to the `ACEQuit' command.

\beginexample
gap> SetInfoACELevel(3);
gap> F := FreeGroup("a","b");; a := F.1;;  b := F.2;;
gap> G := F / [a^2, b^3, (a*b)^5 ];
<fp group on the generators [ a, b ]>
gap> ACEStart(FreeGeneratorsOfFpGroup(G), RelatorsOfFpGroup(G), [a*b]
>          # Options
>          : echo, # Echo handled by GAP (not ACE)
>            enum := "A_5",  # Give the group G a meaningful name
>            subg := "C_5"); # Give the subgroup a meaningful name
ACEStart called with the following arguments:
 Group generators : [ a, b ]
 Group relators : [ a^2, b^3, a*b*a*b*a*b*a*b*a*b ]
 Subgroup generators : [ a*b ]
#I  ACE 3.000        Sun Aug 13 17:21:24 2000
#I  =========================================
#I  Host information:
#I    name = boronia
ACEStart called with the following options:
 echo := true (not passed to ACE)
 enum := A_5
 subg := C_5
#I  ***
#I  INDEX = 12 (a=12 r=16 h=1 n=16; l=3 c=0.00; m=14 t=15)
1
gap> # The return value on the last line identifies the interactive process
gap> # ... which we use with functions that need to interact with it:      
gap> ACEStats(1);    
rec( index := 12, cputime := 0, cputimeUnits := "10^-2 seconds", 
  activecosets := 12, maxcosets := 14, totcosets := 15 )
gap> # Actually, we didn't need to pass an argument to ACEStats()          
gap> # ... we could have relied on the default:                            
gap> ACEStats();                                                 
rec( index := 12, cputime := 0, cputimeUnits := "10^-2 seconds", 
  activecosets := 12, maxcosets := 14, totcosets := 15 )
gap> # Similarly, we can use ACECosetTable() (which returns the 
gap> # `standardised' coset table) with or without an argument:  
gap> ACECosetTable(); # Interactive version of ACECosetTableFromGensAndRels()
#I  CO: a=12 r=13 h=1 n=13; c=+0.00
#I   coset ||      b      B      a
#I  -------+---------------------
#I       1 ||      3      2      2
#I       2 ||      1      3      1
#I       3 ||      2      1      4
#I       4 ||      8      5      3
#I       5 ||      4      8      6
#I       6 ||      9      7      5
#I       7 ||      6      9      8
#I       8 ||      5      4      7
#I       9 ||      7      6     10
#I      10 ||     12     11      9
#I      11 ||     10     12     12
#I      12 ||     11     10     11
[ [ 2, 1, 4, 3, 6, 5, 8, 7, 10, 9, 12, 11 ], 
  [ 2, 1, 4, 3, 6, 5, 8, 7, 10, 9, 12, 11 ], 
  [ 3, 1, 2, 5, 7, 8, 4, 9, 6, 11, 12, 10 ], 
  [ 2, 3, 1, 7, 4, 9, 5, 6, 8, 12, 10, 11 ] ]
gap> # To terminate the interactive process we do:
gap> ACEQuit(1); # Again, we could have omitted the 1
gap> # If we had more than one interactive process we could have 
gap> # terminated them all in one go with:
gap> ACEQuitAll();
\endexample

%%%%%%%%%%%%%%%%%%%%%%%%%%%%%%%%%%%%%%%%%%%%%%%%%%%%%%%%%%%%%%%%%%%%%%
\Section{Fun with ACEExample}

First let's see the `ACEExample' index  (obtained  with  no  argument,
with  `"index"'  as  argument,  or  with  a  non-existent  example  as
argument):

\beginexample
gap> ACEExample();
#I                             ACEExample Index
#I                             ----------------
#I  This index is displayed when calling ACEExample with no arguments, or
#I  with the argument: "index", or with a non-existent example name.
#I  
#I  The following ACE examples are available (in each case, for a subgroup
#I  H of a group G, the cosets of H in G are enumerated):
#I  
#I    Example          G                      H              strategy
#I    -------          -                      -              --------
#I    "A5"             A_5                    Id             default
#I    "A5-C5"          A_5                    C_5            default
#I    "C5-fel0"        C_5                    Id             felsch := 0
#I    "F27-purec"      F(2,7) = C_29          Id             purec
#I    "F27-fel0"       F(2,7) = C_29          Id             felsch := 0
#I    "F27-fel1"       F(2,7) = C_29          Id             felsch := 1
#I    "M12-hlt"        M_12 (Matthieu group)  Id             hlt
#I    "M12-fel1"       M_12 (Matthieu group)  Id             felsch := 1
#I    "SL219-hard"     SL(2,19)               ||G : H|| = 180  hard
#I    "perf602p5"      PerfectGroup(60*2^5,2) ||G : H|| = 480  default
#I  * "2p17-fel1"      ||G|| = 2^17             ||G : H|| = 1    felsch := 1
#I  * "2p18-fel1"      ||G|| = 2^18             ||G : H|| = 2    felsch := 1
#I  * "big-fel1"       ||G|| = 2^18.3           ||G : H|| = 6    felsch := 1
#I  * "big-hard"       ||G|| = 2^18.3           ||G : H|| = 6    hard
#I    "2p17-id-fel1"   ||G|| = 2^17             Id             felsch := 1
#I    "2p17-2p14-fel1" ||G|| = 2^17             ||G : H|| = 2^14 felsch := 1
#I    "2p17-2p3-fel1"  ||G|| = 2^17             ||G : H|| = 2^3  felsch := 1
#I    "2p17-fel1a"     ||G|| = 2^17             ||G : H|| = 1    felsch := 1
#I  
#I  Notes
#I  -----
#I  1. The example (first) argument of  ACEExample()  is  a  string; each
#I     example above is in double quotes to remind you to include them.
#I  2. The enumeration for each of the *-ed examples fails. (See Note 3.)
#I  3. Try altering the ACE function used, by calling  ACEExample with  a
#I     2nd argument; choose from: ACECosetTableFromGensAndRels (default),
#I     or ACEStats, or ACEStart. The 2nd argument is *not* quoted.
#I  4. Try `SetInfoACELevel(3);' before calling  ACEExample,  to  see the
#I     effect of setting the "mess" (= "messages") option.
#I  5. To suppress a long output, use a double semicolon (`;;') after the
#I     ACEExample command.
#I  6. Also, try `SetInfoACELevel(2);' before calling ACEExample.
gap>
\endexample

Observe that the example we first met in Section~"Using  ACE  Directly
to Generate a Coset Table", the Fibonacci group F(2,7),  is  available
via examples `"F27-purec"',  `"F27-fel0"',  and  `"F27-fel1"',  except
each of these enumerate the cosets of its trivial subgroup  (of  index
29). Let's experiment with the first of these F(2,7)  examples;  since
this  example  uses  the  `messages'  option,  we  ought  to  set  the
`InfoLevel' of `InfoACE' to  3,  first,  but  since  the  coset  table
(default output) is  quite  long,  we'll  pass  `ACEStats'  as  second
argument:

\beginexample
gap> SetInfoACELevel(3);
gap> ACEExample("F27-purec", ACEStats);
#I  # ACEExample "F27-purec" : enumeration of cosets of H in G,
#I  # where G = F(2,7) = C_29, H = Id, using purec strategy.
#I  #
#I  # F, G, a, b, c, d, e, x, y are local to ACEExample
#I  # We define F(2,7) on 7 generators
#I  F := FreeGroup("a","b","c","d","e", "x", "y"); 
#I       a := F.1;  b := F.2;  c := F.3;  d := F.4; 
#I       e := F.5;  x := F.6;  y := F.7;
#I  G := F / [a*b*c^-1, b*c*d^-1, c*d*e^-1, d*e*x^-1, 
#I            e*x*y^-1, x*y*a^-1, y*a*b^-1];
#I  ACEStats(
#I      FreeGeneratorsOfFpGroup(G), 
#I      RelatorsOfFpGroup(G), 
#I      [] # Generators of identity subgroup (empty list)
#I      # Options that don't affect the enumeration
#I      : echo, enum := "F(2,7), aka C_29", subg := "Id",
#I      # Other options
#I      wo := "2M", mess := 25000, purec);
ACEStats called with the following arguments:
 Group generators : [ a, b, c, d, e, x, y ]
 Group relators : [ a*b*c^-1, b*c*d^-1, c*d*e^-1, d*e*x^-1, e*x*y^-1, 
  x*y*a^-1, y*a*b^-1 ]
 Subgroup generators : [  ]
ACEStats called with the following options:
 echo := true (not passed to ACE)
 enum := F(2,7), aka C_29
 subg := Id
 wo := 2M
 mess := 25000
 purec (no value)
#I  ACE 3.000        Mon Aug 14 16:02:45 2000
#I  =========================================
#I  Host information:
#I    name = boronia
#I    #-- ACE 3.000: Run Parameters ---
#I  Group Name: F(2,7), aka C_29;
#I  Group Generators: abcdexy;
#I  Group Relators: abC, bcD, cdE, deX, exY, xyA, yaB;
#I  Subgroup Name: Id;
#I  Subgroup Generators: ;
#I  Wo:2M; Max:142855; Mess:25000; Ti:-1; Ho:-1; Loop:0;
#I  As:0; Path:0; Row:0; Mend:0; No:0; Look:0; Com:100;
#I  C:1000; R:0; Fi:1; PMod:0; PSiz:256; DMod:4; DSiz:1000;
#I    #--------------------------------
#I  DD: a=5290 r=1 h=1050 n=5291; l=8 c=+0.03; d=2
#I  CD: a=10410 r=1 h=2149 n=10411; l=13 c=+0.03; m=10410 t=10410
#I  DD: a=15428 r=1 h=3267 n=15429; l=18 c=+0.03; d=0
#I  DD: a=20430 r=1 h=4386 n=20431; l=23 c=+0.03; d=1
#I  DD: a=25397 r=1 h=5519 n=25399; l=28 c=+0.02; d=1
#I  CD: a=30313 r=1 h=6648 n=30316; l=33 c=+0.03; m=30313 t=30315
#I  DS: a=32517 r=1 h=7326 n=33240; l=36 c=+0.02; s=2000 d=997 c=4
#I  DS: a=31872 r=1 h=7326 n=33240; l=36 c=+0.01; s=4000 d=1948 c=53
#I  DS: a=29077 r=1 h=7326 n=33240; l=36 c=+0.01; s=8000 d=3460 c=541
#I  DS: a=23433 r=1 h=7326 n=33240; l=36 c=+0.02; s=16000 d=5940 c=2061
#I  DS: a=4163 r=1 h=7326 n=33240; l=36 c=+0.08; s=32000 d=447 c=15554
#I  INDEX = 29 (a=29 r=1 h=33240 n=33240; l=37 c=0.41; m=33237 t=33239)
#I  ***
rec( index := 29, cputime := 41, cputimeUnits := "10^-2 seconds", 
  activecosets := 29, maxcosets := 33237, totcosets := 33239 )
gap>
\endexample

Observe that the first group of `Info' lines list  the  commands  that
were executed; these lines are followed by the result  of  the  `echo'
option (see~"option echo"); which  in  turn  are  followed  by  `Info'
messages from {\ACE} courtesy of the non-zero value of the  `messages'
option (and we see these because  we  first  set  the  `InfoLevel'  of
`InfoACE' to 3); and finally,  we  get  the  output  (record)  of  the
`ACEStats' command.

Observe also that {\ACE} has used the same generators as  were  input;
this will always occur if you stick to single  lowercase  letters  for
your generator names. Note, also that capitalisation is used by {\ACE}
as a short-hand for inverses, e.g.~`C = c^-1' (see `Group Relators' in
the {\ACE} \lq{}Run Parameters' block).

Now let's observe that we can add some new  options,  even  ones  that
over-ride the example's options; but first we'll reduce the  output  a
bit by setting the `InfoLevel' of `InfoACE' to 2 and since we are  not
going to observe any progress messages from {\ACE} with that `InfoACE'
level we'll set `messages := 0'; also we'll use the  default  function
`ACECosetTableFromGensAndRels' and so it's like  our  first  encounter
with F(2,7) we'll add the subgroup generator `c'  via  `sg  :=  ["c"]'
(see "option sg"). Observe that `"c"' is a string not a  {\GAP}  group
generator; to convert a list of {\GAP}  words  <sgens>  in  generators
<fgens>, suitable for  an  assignment  of  the  `sg'  option  use  the
construction: `ToACEWords(<fgens>, <sgens>)' (see~"ToACEWords").  Note
that if only single lowercase letter strings are used to identify  the
{\GAP} group generators, the same strings are used to  identify  those
generators in {\ACE}. (It's actually fortunate that we could pass  the
value of  `sg'  as  a  string  here,  since  the  generators  of  each
`ACEExample' example are *local* variables and so are not accessible.)
For good measure, we also change the string identifying  the  subgroup
(since it will no longer be the trivial  group),  via  the  `subgroup'
option (see "option subgroup").

\beginexample
gap> SetInfoACELevel(2);
gap> ACEExample("F27-purec" : sg := ["c"], subgroup := "< c >", messages := 0);
#I  # ACEExample "F27-purec" : enumeration of cosets of H in G,
#I  # where G = F(2,7) = C_29, H = Id, using purec strategy.
#I  #
#I  # F, G, a, b, c, d, e, x, y are local to ACEExample
#I  # We define F(2,7) on 7 generators
#I  F := FreeGroup("a","b","c","d","e", "x", "y"); 
#I       a := F.1;  b := F.2;  c := F.3;  d := F.4; 
#I       e := F.5;  x := F.6;  y := F.7;
#I  G := F / [a*b*c^-1, b*c*d^-1, c*d*e^-1, d*e*x^-1, 
#I            e*x*y^-1, x*y*a^-1, y*a*b^-1];
#I  ACECosetTableFromGensAndRels(
#I      FreeGeneratorsOfFpGroup(G), 
#I      RelatorsOfFpGroup(G), 
#I      [] # Generators of identity subgroup (empty list)
#I      # Options that don't affect the enumeration
#I      : echo, enum := "F(2,7), aka C_29", subg := "Id", 
#I      # Other options
#I      wo := "2M", mess := 25000, purec, 
#I      # User Options
#I        sg := [ "c" ],
#I        subgroup := "< c >",
#I        messages := 0);
ACECosetTableFromGensAndRels called with the following arguments:
 Group generators : [ a, b, c, d, e, x, y ]
 Group relators : [ a*b*c^-1, b*c*d^-1, c*d*e^-1, d*e*x^-1, e*x*y^-1, 
  x*y*a^-1, y*a*b^-1 ]
 Subgroup generators : [  ]
ACECosetTableFromGensAndRels called with the following options:
 aceexampleoptions := true (inserted by ACEExample, not passed to ACE)
 echo := true (not passed to ACE)
 enum := F(2,7), aka C_29
 wo := 2M
 purec (no value)
 sg := [ "c" ] (brackets are not passed to ACE)
 subgroup := < c >
 messages := 0
#I  INDEX = 1 (a=1 r=2 h=2 n=2; l=4 c=0.00; m=332 t=332)
[ [ 1 ], [ 1 ], [ 1 ], [ 1 ], [ 1 ], [ 1 ], [ 1 ], [ 1 ], [ 1 ], [ 1 ], 
  [ 1 ], [ 1 ], [ 1 ], [ 1 ] ]
gap>
\endexample

Observe that in the block of `Info' output all  the  original  example
options are listed along with our new options `sg := [ "c" ], messages
:= 0' after the tag \lq{}`\#  User  Options''.  Following  the  `Info'
block there is the block due to `echo'; in its listing of the  options
first up there is `aceexampleoptions' alerting us that we passed  some
`ACEExample' options. Observe that `subg := Id' and  `mess  :=  25000'
have disappeared (they were over-ridden by  `subgroup  :=  "\<  c  >",
messages := 0', but the quotes for the value  of  `subgroup'  are  not
visible); note that we didn't have to use the  same  abbreviations  to
over-ride them. Also observe that our  new  options  *are*  last.

Now following on from our last example we shall  demonstrate  how  one
can recover  from  a  `break'-loop{\undoquotes  \atindex  {break-loop}
{@`break'-loop}} (see Section~"Using ACE Directly to Generate a  Coset
Table"). To force the `break'-loop we pass  `max  :=  2'  (see~"option
max"),   while   using   the   default   {\ACE}   interface   function
`ACECosetTableFromGensAndRels' of `ACEExample'; in  this  way,  {\ACE}
will not be able to complete  the  enumeration,  and  hence  enters  a
`break'-loop when it tries to provide a complete  coset  table.  While
we're at it we'll pass the `hlt' (see~"option  hlt")  strategy  option
(which will over-ride `purec'). (The `InfoACE' level is still 2.) Note
that there are some \lq{}user-input' comments inserted at  the  `brk>'
prompt.

\beginexample
gap> ACEExample("F27-purec" : sg := ["c"], subgroup := "< c >", max := 2, hlt);
#I  # ACEExample "F27-purec" : enumeration of cosets of H in G,
#I  # where G = F(2,7) = C_29, H = Id, using purec strategy.
#I  #
#I  # F, G, a, b, c, d, e, x, y are local to ACEExample
#I  # We define F(2,7) on 7 generators
#I  F := FreeGroup("a","b","c","d","e", "x", "y"); 
#I       a := F.1;  b := F.2;  c := F.3;  d := F.4; 
#I       e := F.5;  x := F.6;  y := F.7;
#I  G := F / [a*b*c^-1, b*c*d^-1, c*d*e^-1, d*e*x^-1, 
#I            e*x*y^-1, x*y*a^-1, y*a*b^-1];
#I  ACECosetTableFromGensAndRels(
#I      FreeGeneratorsOfFpGroup(G), 
#I      RelatorsOfFpGroup(G), 
#I      [] # Generators of identity subgroup (empty list)
#I      # Options that don't affect the enumeration
#I      : echo, enum := "F(2,7), aka C_29", subg := "Id", 
#I      # Other options
#I      wo := "2M", mess := 25000, purec, 
#I      # User Options
#I        sg := [ "c" ],
#I        subgroup := "< c >",
#I        max := 2,
#I        hlt := true);
ACECosetTableFromGensAndRels called with the following arguments:
 Group generators : [ a, b, c, d, e, x, y ]
 Group relators : [ a*b*c^-1, b*c*d^-1, c*d*e^-1, d*e*x^-1, e*x*y^-1, 
  x*y*a^-1, y*a*b^-1 ]
 Subgroup generators : [  ]
ACECosetTableFromGensAndRels called with the following options:
 aceexampleoptions := true (inserted by ACEExample, not passed to ACE)
 echo := true (not passed to ACE)
 enum := F(2,7), aka C_29
 wo := 2M
 mess := 25000
 purec (no value)
 sg := [ "c" ] (brackets are not passed to ACE)
 subgroup := < c >
 max := 2
 hlt (no value)
#I  OVERFLOW (a=2 r=1 h=1 n=3; l=4 c=0.00; m=2 t=2)
Error : No coset table ... at
Error( ": No coset table ..." );
#I  The `ACE' coset enumeration failed with the result:
#I  OVERFLOW (a=2 r=1 h=1 n=3; l=4 c=0.00; m=2 t=2)
#I  Try relaxing any restrictive options:
#I  type: 'DisplayACEOptions();' to see current ACE options;
#I  type: 'SetACEOptions(:<option1> := <value1>, ...);'
#I  to set <option1> to <value1> etc.
#I  (i.e. pass options after the ':' in the usual way)
#I  ... and then, type: 'return;' to continue.
#I  Otherwise, type: 'quit;' to quit the enumeration.
Entering break read-eval-print loop, you can 'quit;' to quit to outer loop,
or you can return to continue
brk> # Let's give ACE enough coset numbers to work with ...                    
brk> # and while we're at it see the effect of 'echo := 2' :                   
brk> SetACEOptions(: max := 0, echo := 2);                                     
brk> # Let's check what the options are now:                                   
brk> DisplayACEOptions();                                                      
rec(
  enum := "F(2,7), aka C_29",
  wo := "2M",
  mess := 25000,
  purec := true,
  sg := [ "c" ],
  subgroup := "< c >",
  hlt := true,
  max := 0,
  echo := 2 )
brk> # That's ok ... so now we 'return;' to escape the break-loop              
brk> return;                                                                   
ACECosetTableFromGensAndRels called with the following arguments:
 Group generators : [ a, b, c, d, e, x, y ]
 Group relators : [ a*b*c^-1, b*c*d^-1, c*d*e^-1, d*e*x^-1, e*x*y^-1, 
  x*y*a^-1, y*a*b^-1 ]
 Subgroup generators : [  ]
ACECosetTableFromGensAndRels called with the following options:
 aceexampleoptions := true (inserted by ACEExample, not passed to ACE)
 enum := F(2,7), aka C_29
 wo := 2M
 mess := 25000
 purec (no value)
 sg := [ "c" ] (brackets are not passed to ACE)
 subgroup := < c >
 hlt (no value)
 max := 0
 echo := 2 (not passed to ACE)
Other options set via ACE defaults:
 asis := 0
 compaction := 10
 ct := 0
 dmode := 0
 dsize := 1000
 fill := 1
 hole := -1
 lookahead := 1
 loop := 0
 mendelsohn := 0
 no := 0
 path := 0
 pmode := 0
 psize := 256
 row := 1
 rt := 1000
 time := -1
#I  INDEX = 1 (a=1 r=2 h=2 n=2; l=3 c=0.01; m=2049 t=3127)
[ [ 1 ], [ 1 ], [ 1 ], [ 1 ], [ 1 ], [ 1 ], [ 1 ], [ 1 ], [ 1 ], [ 1 ], 
  [ 1 ], [ 1 ], [ 1 ], [ 1 ] ]
gap>
\endexample

Observe that  `purec'  did  *not*  disappear  from  the  option  list;
nevertheless, it *is* over-ridden by the `hlt' option (at  the  {\ACE}
level). Observe the \lq{}`Other options set via ACE defaults'' list of
options  that  has  resulted  from  having  the  `echo  :=  2'  option
(see~"option echo"). Observe, also, that  `hlt'  is  nowhere  near  as
good, here, as  `purec'  (refer  to  Section~"Using  ACE  Directly  to
Generate  a  Coset  Table"):  whereas  `purec'  completed   the   same
enumeration with a total number of coset numbers  of  332,  the  `hlt'
strategy required 3127.

Of course, running `ACEExample' with  `ACEStart'  as  second  argument
opens up far more flexibility. Try it! Have fun!  Play  with  as  many
options as you can. Also, note that the `*'-ed examples of  the  index
fail  to  give  a  coset  table;  so  these  give  you  non-artificial
`break'-loop examples for you to try.

%%%%%%%%%%%%%%%%%%%%%%%%%%%%%%%%%%%%%%%%%%%%%%%%%%%%%%%%%%%%%%%%%%%%%%
\Section{Using ACEReadResearchExample}

Here I will describe how  `ACEReadResearchExample'  defines  functions
such  as  `PGRelFind',  and  how  these   functions   were   used   in
\cite{CHHR00} to show that the group $L_3(5)$ has deficiency 0.

%%%%%%%%%%%%%%%%%%%%%%%%%%%%%%%%%%%%%%%%%%%%%%%%%%%%%%%%%%%%%%%%%%%%%%%%%
%%
%E
