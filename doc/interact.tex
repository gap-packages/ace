%%%%%%%%%%%%%%%%%%%%%%%%%%%%%%%%%%%%%%%%%%%%%%%%%%%%%%%%%%%%%%%%%%%%%%%%%
%%
%W  interact.tex   ACE documentation - interactive fns  Joachim Neub"user
%W                                                            Greg Gamble
%%
%Y  Copyright (C) 2000  Centre for Discrete Mathematics and Computing
%Y                      Department of Information Tech. & Electrical Eng.
%Y                      University of Queensland, Australia.
%%

%%%%%%%%%%%%%%%%%%%%%%%%%%%%%%%%%%%%%%%%%%%%%%%%%%%%%%%%%%%%%%%%%%%%%%%%%
\Chapter{Functions for Using ACE Interactively}

The user will probably benefit most from interactive use of {\ACE}  by
setting   the   `InfoLevel'   of   `InfoACE'    to    at    least    3
(see~"InfoACE"),  particularly  if  she  uses  the  `messages'
option with a non-zero value.

Have you read the various options warnings yet? If  not,  please  take
the time to  read  Section~"General  Warnings  regarding  the  Use  of
Options"  which  directs  you  to  various   important   sections   of
Chapter~"Options for ACE".

We describe in this chapter the functions that manipulate and initiate
interactive {\ACE} processes.

An  interactive  {\ACE}  process  is  initiated  by   `ACEStart'   and
terminated  via  `ACEQuit';   these   functions   are   described   in
Section~"Starting and Stopping Interactive ACE Processes".  `ACEStart'
also has forms that manipulate an already started  interactive  {\ACE}
process. `ACEStart' always  returns  a  positive  integer  <i>,  which
identifies the  interactive  {\ACE}  process  that  was  initiated  or
manipulated.

Most functions (there is one `ACEStart' exception), that manipulate an
already started interactive {\ACE} process,  have  a  form  where  the
first  argument  is  the  integer  <i>  returned  by  the   initiating
`ACEStart' command, and a second form with one argument  fewer  (where
the integer <i> is  discovered  by  a  default  mechanism,  namely  by
determining the least integer <i>  for  which  there  is  a  currently
active interactive {\ACE} process). We will  thus  commonly  say  that
``for the <i>th (or default) interactive  {ACE}  process''  a  certain
function performs a given action. In each case, it is an error, if <i>
is not the index of an active interactive process,  or  there  are  no
current active interactive processes.

*Notes*: 
The global method of passing options (via `PushOptions'),  should  not
be  used  with  any  of  the  interactive  functions.  In  fact,   the
`OptionsStack' should be empty at the  time  any  of  the  interactive
functions is called.

On `quit'ting {\GAP}, `ACEQuitAll();' is  executed,  which  terminates
all active interactive {\ACE} processes. If {\GAP} is  killed  without
`quit'ting, before all interactive {\ACE}  processes  are  terminated,
*zombie* processes (still living  *child*  processes  whose  *parents*
have died), will result. Since zombie processes do consume  resources,
in such an event, the responsible computer user should  seek  out  and
kill the still living `ace' children (e.g.~by piping the output  of  a
`ps' with appropriate options, usually `aux' or `ef', to  `grep  ace',
to find the process ids, and then using `kill'; try `man ps' and  `man
kill' if these hints are unhelpful).

%%%%%%%%%%%%%%%%%%%%%%%%%%%%%%%%%%%%%%%%%%%%%%%%%%%%%%%%%%%%%%%%%%%%%%%
\Section{Starting and Stopping Interactive ACE Processes}

\>ACEStart( <fgens>, <rels>, <sgens> [:<options>] )!{details} F
\>ACEStart( <i> [:<options>] ) F
\>ACEStart( [:<options>] ) F
\>ACEStart( <i>, <fgens>, <rels>, <sgens> [:<options>] ) F
\>ACEStart( 0 [:<options>] ) F
\>ACEStart( 0, <fgens>, <rels>, <sgens> [:<options>] ) F
\>ACEStart( 0, <i> [:<options>] ) F
\>ACEStart( 0, <i>, <fgens>, <rels>, <sgens> [:<options>] ) F

The variables are: <i>, a  positive  integer  numbering  from  1  that
represents (indexes) an already running  interactive  {\ACE}  process;
<fgens>, a list of free generators; <rels>, a list  of  words  in  the
generators <fgens> giving relators for a finitely presented group; and
<sgens>, a list of subgroup generators, again expressed  as  words  in
the free generators <fgens>. Each of <fgens>, <rels> and  <sgens>  are
given in the standard {\GAP} format for finitely presented groups (See
Chapter~"ref:Finitely  Presented  Groups"  of  the  {\GAP}   Reference
Manual).

All forms of `ACEStart' accept options described in  Chapters~"Options
for ACE" and~"Strategy  Options  for  ACE",  and  Appendix~"Other  ACE
Options", which are listed behind  a  colon  in  the  usual  way  (see
"ref:Function Calls" in the {\GAP} Reference Manual).  The  reader  is
strongly   encouraged   to   read   the   introductory   sections   of
Chapter~"Options  for  ACE",  with  regard  to  options.  The   global
mechanism (via `PushOptions') of passing options is *not*  recommended
for use with the interactive {\ACE} interface functions; please ensure
the `OptionsStack' is  empty  before  calling  an  interactive  {\ACE}
interface function. 

The  return  value  (for  all  forms  of  `ACEStart')  is  an  integer
(numbering from  1)  which  represents  the  running  process.  It  is
possible to have more than one interactive process  running  at  once.
The integer returned may be used to index which of these processes  an
interactive {\ACE} interface function should be applied to.

The first four forms  of  `ACEStart'  insert  a  `start'  (see~"option
start") directive after the user's options to invoke  an  enumeration.
The last four forms, with `0' as  first  argument,  do  not  insert  a
`start' directive. Moreover, the last 3 forms of `ACEStart', with  `0'
as  first  argument  only  differ  from  the  corresponding  forms  of
`ACEStart' without the `0' argument, in that  they  do  not  insert  a
`start' directive. `ACEStart(0)',  however,  is  special;  unlike  the
no-argument form of `ACEStart' it invokes  a  new  interactive  {\ACE}
process. We will now further describe each form of `ACEStart', in  the
order they appear above.

The first form of `ACEStart' (on three arguments) is the usual way  to
start an interactive {\ACE} process.

When `ACEStart' is called with one positive integer  argument  <i>  it
starts a new enumeration on the <i>th running process, i.e.~it  scrubs
a previously generated table and starts from  scratch  with  the  same
parameters (i.e.~the same arguments and options); except that  if  new
options are included these will modify  those  given  previously.  The
only reason for doing such a thing, without new options, is to perhaps
compare timings of runs (a second run is quicker  because  memory  has
already been allocated).  If  you  are  interested  in  this  sort  of
information, however, you may be better off dealing directly with  the
standalone.

When `ACEStart' is  called  with  no  arguments  it  finds  the  least
positive integer <i> for which an interactive process is  running  and
applies `ACEStart(<i>)'. (Most users will  only  run  one  interactive
process at a time. Hence, `ACEStart()' will be a useful  shortcut  for
`ACEStart(1)'.)

The fourth form  of  `ACEStart'  on  four  arguments,  invokes  a  new
enumeration on the <i>th running process, with new generators <fgens>,
relators <rels> and subgroup generators <sgens>. This is provided so a
user can re-use an already running process, rather than  start  a  new
process. This may be useful when pseudo-ttys are  a  scarce  resource.
See   the    notes    for    the    non-interactive    `ACECosetTable'
("ACECosetTableFromGensAndRels") which demonstrates an application  of
a usage of this and the following form of `ACEStart' in a loop.

The fifth form of `ACEStart' (on the one argument: `0')  initiates  an
interactive {\ACE} process, processes any user options, but  does  not
insert a `start' (see~"option start") directive. This form  is  mainly
for gurus who are familiar with the {\ACE} standalone and who wish, at
least initially,  to  communicate  with  {\ACE}  using  the  primitive
read/write tools of  Section~"Primitive ACE Read/Write Functions".  In
this  case,  after  the  group  generators,  relators,  and   subgroup
generators have been set in the {\ACE} process, invocations of any  of
`ACEGroupGenerators'     (see~"ACEGroupGenerators"),     `ACERelators'
(see~"ACERelators"),                           `ACESubgroupGenerators'
(see~"ACESubgroupGenerators"),           or            `ACEParameters'
(see~"ACEParameters") will establish the corresponding {\GAP}  values.
Be warned, though, that unless one of the general  {\ACE}  modes  (see
Section~"General ACE modes"): `ACEStart' (without  a  zero  argument),
`ACERedo' (see~"ACERedo") or `ACEContinue' (see~"ACEContinue"), or one
of the mode options: `start' (see~"option start"), `redo' (see~"option
redo") or `continu' (see~"option continu"), has been invoked since the
last change of any parameter options (see~"ACEParameterOptions"), some
of the values reported by `ACEParameters' may well be *incorrect*.

The sixth form of `ACEStart' (on four arguments), is  like  the  first
form of `ACEStart' (on three  arguments),  except  that  it  does  not
insert a `start'  (see~"option  start")  directive.  It  initiates  an
interactive {\ACE} process, with a presentation defined by its last  3
arguments.

The seventh form of `ACEStart' (on two arguments), is like the  second
form of `ACEStart' (on one argument), except that it does not insert a
`start' (see~"option start") directive. It processes any  new  options
for  the  <i>th  interactive  {\ACE}  process.  `ACEStart(0,  <i>   [:
<options> ] )', is similar to `SetACEOptions(<i>  [:  <options>  ]  )'
(see~"SetACEOptions"), but unlike the latter does not invoke a general
mode (see Section~"General ACE Modes").

The last form of `ACEStart' (on five arguments), is  like  the  fourth
form of `ACEStart' (on four arguments), except that it does not insert
a `start' (see~"option  start")  directive.  It  re-uses  an  existing
interactive {\ACE} process, with a new presentation. There is no  form
of `ACEStart' with the same functionality as this form, where the  <i>
argument is omitted.

*Note:*
When an interactive  {\ACE}  process  is  initiated  by  `ACEStart'  a
process number <i> is assigned to it (the integer <i> returned by  the
`ACEStart' command), an {\ACE} (binary) process (in the UNIX sense) is
started up, a {\GAP} iostream is  assigned  to  communicate  with  the
{\ACE}  (binary)  process,  and  the   essential   ``defining''   data
associated  with  the  interactive  {\ACE}   process   is   saved   in
`ACEData.io[<i>]' (see~"ACEData" for precisely what is saved).

\>ACEQuit( <i> )!{details} F
\>ACEQuit() F

terminate an interactive {\ACE} process,  where  <i>  is  the  integer
returned by `ACEStart' when the process was  started.  If  the  second
form is used (i.e.~without arguments) then the interactive process  of
least index that is still running is terminated.

*Note:*
`ACEQuit(<i>)' terminates the {\ACE} (binary) process  of  interactive
{\ACE} process <i>, and closes its {\GAP} iostream,  and  unbinds  the
record `ACEData.io[<i>]' (see~"ACEStart" note).

It can happen that the {\ACE} (binary) process, and hence  the  {\GAP}
iostream assigned to communicate with it, can die,  e.g.~by  the  user
typing a `Ctrl-C' while the {\ACE} (binary) process is  engaged  in  a
long  calculation.  `IsACEProcessAlive'  (see~"IsACEProcessAlive")  is
provided to check the status of the {\GAP}  iostream  (and  hence  the
status of the {\ACE} (binary) process it was communicating  with);  in
the   case   that   it   is   indeed    dead,    `ACEResurrectProcess'
(see~"ACEResurrectProcess") may be used to start a new {\ACE} (binary)
process and assign a new {\GAP} iostream to communicate  with  it,  by
using the ``defining'' data of the interactive {\ACE} process saved in
`ACEData.io[<i>]'.

\>ACEQuitAll() F

is provided as a convenience,  to  terminate  all  active  interactive
{\ACE} processes with a single command. It is equivalent to  executing
`ACEQuit(<i>)'  for  all  active  interactive  {\ACE}  processes   <i>
(see~"ACEQuit").

%%%%%%%%%%%%%%%%%%%%%%%%%%%%%%%%%%%%%%%%%%%%%%%%%%%%%%%%%%%%%%%%%%%%%%
\Section{General ACE Modes}

For our purposes, we define an interactive {\ACE} interface command to
be  an  {\ACE}  mode,  if  it  delivers  an  enumeration  result  (see
Section~"Results messages"). Thus, `ACEStart' (see~"ACEStart"), except
when called with the  argument  0,  and  the  commands  `ACERedo'  and
`ACEContinue' which we describe below are {\ACE} modes; we call  these
*general {\ACE} modes*. Additionally, there  are  two  other  commands
which   deliver   enumeration   results:    `ACEAllEquivPresentations'
(see~"ACEAllEquivPresentations")   and   `ACERandomEquivPresentations'
(see~"ACERandomEquivPresentations"); we call these ``experimentation''
{\ACE} modes and describe them in Section~"Experimentation ACE Modes".

*Guru Note:*
The {\ACE} standalone uses the term `mode'  in  a  slightly  different
sense. There, the commands: `start', `redo'  or  `continue',  put  the
{\ACE} enumerator in `start mode', `redo mode' or `continue mode'.  In
this manual, we have used the term to mean  the  command  itself,  and
generalised it  to  include  any  command  that  produces  enumeration
results.

After changing any of  {\ACE}'s  parameters,  one  of  three  *general
modes* is possible: one may be able to ``continue'' via  `ACEContinue'
(see~"ACEContinue"), or ``redo'' via `ACERedo' (see~"ACERedo"), or  if
neither of  these  is  possible  one  may  have  to  re-``start''  the
enumeration   via   `ACEStart'   (see~"ACEStart").   Generally,    the
appropriate mode is invoked automatically when options are changed; so
most users should be able to ignore the following three functions.

\>ACEModes( <i> ) F
\>ACEModes() F

for the <i>th (or default) interactive {\ACE} process, return a record
whose fields are the modes `ACEStart',  `ACEContinue'  and  `ACERedo',
and whose values are `true' if the mode is possible  for  the  process
and `false' otherwise.

\>ACEContinue( <i> [:<options>] ) F
\>ACEContinue( [:<options>] ) F

for the <i>th (or  default)  interactive  {\ACE}  process,  apply  any
<options> and then ``continue'' the current enumeration, building upon
the existing table. If a previous  run  stopped  without  producing  a
finite index you can, in principle, change  any  of  the  options  and
continue on. Of course, if you make any changes which  invalidate  the
current table, you won't be allowed to `ACEContinue' and an error will
be raised. However, after `quit'ting the `break'-loop, the interactive
{\ACE} process should normally still be active; after  doing  so,  run
`ACEModes' (see~"ACEModes") to see which of `ACERedo' or `ACEStart' is
possible.

\>ACERedo( <i> [:<options>] ) F
\>ACERedo( [:<options>] ) F

for the <i>th (or  default)  interactive  {\ACE}  process,  apply  any
<options> and then ``redo''  the  current  enumeration  from  coset  1
(i.e., the subgroup), keeping any existing information in the table.

*Notes:*
The  difference  between  `ACEContinue'  and  `ACERedo'  is   somewhat
technical, and the user should regard it as a mode that  is  a  little
more  expensive  than  `ACEContinue'  but  cheaper  than   `ACEStart'.
`ACERedo' is really intended for the case  where  additional  relators
and/or subgroup generators have been introduced;  the  current  table,
which may be incomplete or exhibit a finite index, is still ``valid''.
However, the new data may allow the enumeration to complete, or  cause
a collapse to a smaller index. In some cases,  `ACERedo'  may  not  be
possible and an error  will  be  raised;  in  this  case,  `quit'  the
`break'-loop, and try `ACEStart', which will discard the current table
and re-``start'' the enumeration.

%%%%%%%%%%%%%%%%%%%%%%%%%%%%%%%%%%%%%%%%%%%%%%%%%%%%%%%%%%%%%%%%%%%%%%
\Section{Interactive ACE Process Utility Functions and Interruption of
an Interactive ACE Process}

\>ACEProcessIndex( <i> ) F
\>ACEProcessIndex() F

With argument <i>, which must be a positive integer, `ACEProcessIndex'
returns <i> if it corresponds to an  active  interactive  process,  or
raises an error. With no  arguments  it  returns  the  default  active
interactive process or returns `fail' and emits a warning  message  to
`Info' at `InfoACE' or `InfoWarning' level 1.

*Note:*
Essentially, an  interactive  {\ACE}  process  <i>  is  ``active''  if
`ACEData.io[<i>]' is bound (i.e.~we still have some  data  telling  us
about it). Also see~"ACEStart" note.

\>ACEProcessIndices() F

returns the list of integer indices of all active  interactive  {\ACE}
processes (see~"ACEProcessIndex" for the meaning of ``active'').

\>IsACEProcessAlive( <i> ) F
\>IsACEProcessAlive() F

return `true' if  the  {\GAP}  iostream  of  the  <i>th  (or  default)
interactive {\ACE} process started by `ACEStart'  is  alive  (i.e.~can
still  be  written  to),  or  `false',  otherwise.  (See   the   notes
for~"ACEStart" and~"ACEQuit".)

\atindex{interruption}{@interruption of an interactive ACE process}
\atindex{break-loop}{@\noexpand`break'-loop}
If the user does not yet have a `gap>' prompt then usually  {\ACE}  is
still away doing something and an {\ACE} interface function  is  still
waiting for a reply from {\ACE}. Typing a `Ctrl-C' (i.e.~holding  down
the `Ctrl' key and typing `c') will stop the waiting and  send  {\GAP}
into a `break'-loop, from which one has no option but to `quit;'.  The
typing of `Ctrl-C', in such a circumstance, usually causes the  stream
of the interactive {\ACE} process to die; to  check  this  we  provide
`IsACEProcessAlive' (see~"IsACEProcessAlive"). If  the  stream  of  an
interactive {\ACE} process, indexed by <i>,  say,  has  died,  it  may
still be possible to recover enough of the state, before the death  of
the stream, from  the  information  stored  in  the  `ACEData.io[<i>]'
record (see Section~"The ACEData Record"). For such a purpose, we have
provided `ACEResurrectProcess' (see~"ACEResurrectProcess").

The {\GAP} iostream of an interactive {\ACE} process will also die  if
the {\ACE} binary has a segmentation fault. We do hope that this never
happens to you, but if it does and the failure is  reproducible,  then
it's a bug and we'd like to know about it. Please read the file `README.md'
that comes with the {\ACE} package to find out what to  include  in  a
bug report and who to email it to.

\>ACEResurrectProcess( <i> [: <options>] ) F
\>ACEResurrectProcess( [: <options>] ) F

re-generate the {\GAP} iostream of the <i>th (or default)  interactive
{\ACE} process started by `ACEStart' (see~"ACEStart", the final  note,
in particular), and try to recover as much as possible of the previous
state from saved values  of  the  process's  arguments  and  parameter
options. The possible <options> here are `use' and `useboth' which are
described in detail below.

The arguments of the <i>th interactive {\ACE} process  are  stored  in
`ACEData.io[<i>].args', a  record  with  fields  `fgens',  `rels'  and
`sgens', which are the {\GAP} group generators, relators and  subgroup
generators, respectively (see Section~"The  ACEData  Record").  Option
information is saved in `ACEData.io[<i>].options' when a user uses  an
interactive  {\ACE}  interface   function   with   options   or   uses
`SetACEOptions' (see~"SetACEOptions"). Parameter option information is
saved in `ACEData.io[<i>].parameters' if `ACEParameters' (see~"ACEParameters")
is used to extract from  {\ACE}  the  current  values  of  the  {\ACE}
parameter options (this is generally less reliable unless one  of  the
general {\ACE} modes (see Section~"General ACE Modes"), has  been  run
previously).

By   default,   `ACEResurrectProcess'   recovers   parameter    option
information from `ACEData.io[<i>].options' if it  is  bound,  or  from
`ACEData.io[<i>].parameters'  if   it   is   bound,   otherwise.   The
`ACEData.io[<i>].options'  record,  however,  is  first  filtered  for
parameter and  strategy  options  (see  Sections~"ACEParameterOptions"
and~"The ACEStrategyOptions list") and the `echo' option  (see~"option
echo"). To alter this behaviour, the user is provided two options:

\beginitems

\quad`use := <useList>'& <useList>  may   contain   one  or  both   of
`"options"'  and  `"parameters"'.  By  default,  `use  =   ["options",
"parameters"]'.

\quad`useboth' & (A boolean option). By default, `useboth = false'.

\enditems

If `useboth = true', `SetACEOptions' (see~"SetACEOptions") is  applied
to    the    <i>th    interactive    {\ACE}    process    with    each
`ACEData.io[<i>].(<field>)'   for   each   <field>   (`"options"'   or
`"parameters"') that is bound and in <useList>, in the  order  implied
by <useList>. If `useboth = false', `SetACEOptions'  is  applied  with
`ACEData.io[<i>].(<field>)' for only the first <field> that  is  bound
in <useList>. The current value of the `echo' option is also preserved
(no matter what values the user chooses for the  `use'  and  `useboth'
options).

*Notes:*
Do not use general {\ACE}  options  with  `ACEResurrectProcess';  they
will  only   be   superseded   by   those   options   recovered   from
`ACEData.io[<i>].options'     and/or     `ACEData.io[<i>].parameters'.
Instead, call `SetACEOptions' first (or afterwards). When called prior
to `ACEResurrectProcess', `SetACEOptions' will emit a warning that the
stream is dead; despite this, the `ACEData.io[<i>].options' *will*  be
updated.

`ACEResurrectProcess'  does  *not*  invoke   an   {\ACE}   mode   (see
Section~"General ACE  Modes").  This  leaves  the  user  free  to  use
`SetACEOptions' (which does invoke an {\ACE} mode) to  further  modify
options afterwards.

\>ToACEGroupGenerators( <fgens> ) F

This function produces, from a  {\GAP}  list  <fgens>  of  free  group
generators, the  {\ACE}  directive  string  required  by  the  `group'
(see~"option group") option. (The `group' option may be used to define
the equivalent of <fgens> in an {\ACE} process.)

\>ToACEWords( <fgens>, <words> ) F

This function produces, from a  {\GAP}  list  <words>  in  free  group
generators <fgens>, a string  that  represents  those  <words>  as  an
{\ACE} list of words. `ToACEWords' may be  used  to  provide  suitable
values   for   the   options   `relators'   (see~"option   relators"),
`generators' (see~"option generators"), `sg'  (see~"option  sg"),  and
`rl' (see~"option rl").

%%%%%%%%%%%%%%%%%%%%%%%%%%%%%%%%%%%%%%%%%%%%%%%%%%%%%%%%%%%%%%%%%%%%%%
\Section{Experimentation ACE Modes}

Now we describe the two *experimentation modes*. The term  *mode*  was
defined in Section~"General ACE Modes".

\>ACEAllEquivPresentations( <i>, <val> ) F
\>ACEAllEquivPresentations( <val> ) F

for the <i>th (or default) interactive {\ACE} process,  generates  and
tests an enumeration for combinations  of  relator  ordering,  relator
rotations, and relator inversions; <val> is in the integer range 1  to
7.

The argument <val> is considered as a binary number.  Its  three  bits
are treated as flags, and control relator rotations (the  $2^0$  bit),
relator inversions (the $2^1$ bit) and relator  orderings  (the  $2^2$
bit),  respectively;  where  $1$  means  ``active''  and   $0$   means
``inactive''. (See below for an example).

Before we describe the {\GAP} output of `ACEAllEquivPresentations' let
us spend some time considering what happens before the  {\ACE}  binary
output is parsed.

The `ACEAllEquivPresentations'  command  first  performs  a  ``priming
run'' using the options as they stand. In particular, the  `asis'  and
`messages' options are honoured.

It  then  turns  `asis'  (see~"option   asis")   on   and   `messages'
(see~"option messages") off (i.e.~sets `messages' to 0), and generates
and tests the requested  equivalent  presentations.  The  maximum  and
minimum values attained by `m' (the maximum number  of  coset  numbers
defined at any stage) and `t'  (the  total  number  of  coset  numbers
defined) are tracked, and each time the  statistics  are  better  than
what we've already seen, the {\ACE} binary emits a summary result line
for the relators used. See  Appendix~"The  Meanings  of  ACE's  output
messages" for a discussion of the statistics `m' and `t'.  To  observe
these messages set  the  `InfoLevel'  of  `InfoACE'  to  3;  and  it's
*recommended* that you do this so that you get some idea of  what  the
{\ACE} binary is doing.

The order in which the  equivalent  presentations  are  generated  and
tested has no particular significance, but note that the  presentation
as given *after* the initial priming run) is the  *last*  presentation
to be generated and tested, so that  the  group's  relators  are  left
``unchanged'' by running the `ACEAllEquivPresentations' command.

As discussed by Cannon, Dimino, Havas  and  Watson  \cite{CDHW73}  and
Havas and Ramsay \cite{HR01} such equivalent presentations  can  yield
large variations in  the  number  of  coset  numbers  required  in  an
enumeration. For this command, we are interested in this variation.

After  the  final  presentation  is  run,   some   additional   status
information messages are printed to the {\ACE} output:

\beginlist%unordered

\item{--}  the number of runs which yielded a finite index; 

\item{--}  the total number of runs (excluding the priming run); and 

\item{--}  the range of values observed for `m' and `t'.

\endlist

As an example (drawn from the discussion in \cite{HR99ace}) consider the
enumeration   of   the   $448$   coset   numbers   of   the   subgroup
$\langle  a^2,a^{-1}b \rangle$ of the group
$$
(8,7 \mid 2,3) 
    = \langle a,b \mid a^8 = b^7 = (ab)^2 = (a^{-1}b)^3 = 1 \rangle.
$$
There are $4!=24$  relator  orderings  and  $2^4=16$  combinations  of
relator or inverted relator. Exponents are  taken  into  account  when
rotating relators, so the relators given give rise to 1, 1,  2  and  2
rotations respectively, for a total of $1.1.2.2=4$  combinations.  So,
for  <val>${} = 7$   (resp.~$3$),   $24.16.4=1536$   (resp.~$16.4=64$)
equivalent presentations are tested.

Now we describe the output  of  `ACEAllEquivPresentations';  it  is  a
record with fields:

\beginitems

\quad`primingResult' & the  {\ACE}  enumeration  result  message  (see
Section~"Results Messages") of the priming run;

\quad`primingStats' & the enumeration result of the priming run  as  a
{\GAP}  record  with  fields   `index',   `cputime',   `cputimeUnits',
`activecosets', `maxcosets' and `totcosets', exactly as for the record
returned by `ACEStats' (see~"ACEStats");

\quad`equivRuns' & a list of data records, one for each  progressively
``best'' run, where each record has fields:

\qquad`rels'& the relators in the order used for the run,

\qquad`enumResult'&  the  {\ACE}  enumeration  result   message   (see
Section~"Results Messages") of the run, and

\qquad`stats'& the enumeration result as a {\GAP} record exactly  like
the record returned by `ACEStats' (see~"ACEStats");

\quad`summary' & a record with fields:

\qquad`successes'& the total number of successful (i.e.~having  finite
enumeration index) runs,

\qquad`runs'&  the  total  number  of  equivalent  presentation   runs
executed,

\qquad`maxcosetsRange'& the range of values as a  {\GAP}  list  inside
which each `equivRuns[<i>].maxcosets' lies, and

\qquad`totcosetsRange'& the range of values as a  {\GAP}  list  inside
which each `equivRuns[<i>].totcosets' lies.

\enditems

*Notes:*
In general, the length of the `equivRuns' field list will be less than
the number of runs executed.

There is no way to stop the `ACEAllEquivPresentations' command  before
it has completed, other than killing the task. So do a  reality  check
beforehand on the size of the search  space  and  the  time  for  each
enumeration. If you are interested in finding a ``good''  enumeration,
it can be very helpful, in terms of running time, to put a tight limit
on the number of coset numbers via the `max' option. You may also have
to set `compaction'  =  $100$  to  prevent  time-wasting  attempts  to
recover space via compaction. This maximises throughput by causing the
``bad'' enumerations, which are in the majority, to  overflow  quickly
and abort. If you wish to explore a very large search-space,  consider
firing up many copies of {\ACE}, and starting each with  a  ``random''
equivalent   presentation.   Alternatively,   you   could   use    the
`ACERandomEquivPresentations' command.


\>ACERandomEquivPresentations( <i>, <val> ) F
\>ACERandomEquivPresentations( <val> ) F
\>ACERandomEquivPresentations( <i>, [<val>] ) F
\>ACERandomEquivPresentations( [<val>] ) F
\>ACERandomEquivPresentations( <i>, [<val>, <Npresentations>] ) F
\>ACERandomEquivPresentations( [<val>, <Npresentations>] ) F

for the <i>th (or default) interactive {\ACE} process,  generates  and
tests up to <Npresentations> (or 8,  in  the  first  4  forms)  random
presentations; <val>, an integer in the range 1  to  7,  acts  as  for
`ACEAllEquivPresentations' and <Npresentations>, when given, should be
a positive integer.

The routine first turns `asis' (see~"option asis") on  and  `messages'
(see~"option messages") off (i.e.~sets  `messages'  to  0),  and  then
generates  and  tests  the  requested  number  of  random   equivalent
presentations. For  each  presentation,  the  relators  used  and  the
summary result line are printed by {\ACE}. To observe  these  messages
set the `InfoLevel' of `InfoACE' to at least 3.

`ACERandomEquivPresentations' parses the {\ACE} messages,  translating
them to {\GAP}, and thus returns a list of  records  (similar  to  the
field     `equivRuns'     of     the      returned      record      of
`ACEAllEquivPresentations'). Each record of the returned list  is  the
data derived from a presentation run and has fields:

\beginitems

\quad`rels'& the relators in the order used for the run,

\quad`enumResult'&  the  {\ACE}  enumeration   result   message   (see
Section~"Results Messages") of the run, and

\quad`stats'& the enumeration result as a {\GAP} record  exactly  like
the record returned by `ACEStats' (see~"ACEStats").

\enditems

*Notes:*
The relator inversions and rotations  are  ``genuinely''  random.  The
relator permuting is a little bit of a kludge, with the ``quality'' of
the permutations tending to  improve  with  successive  presentations.
When  the   `ACERandomEquivPresentations'   command   completes,   the
presentation active is the *last* one generated.

*Guru Notes:*
It might appear that neglecting to restore the  original  presentation
is an error. In fact, it is a useful feature! Suppose that  the  space
of equivalent presentations is too  large  to  exhaustively  test.  As
noted in the entry for `ACEAllEquivPresentations',  we  can  start  up
multiple copies of `ACEAllEquivPresentations' at random points in  the
search-space. Manually generating `random' equivalent presentations to
serve   as   starting-points   is   tedious   and   error-prone.   The
`ACERandomEquivPresentations'  command  provides  a  simple  solution;
simply    run    `ACERandomEquivPresentations(<i>,     7);'     before
`ACEAllEquivPresentations(<i>, 7);'.

%%%%%%%%%%%%%%%%%%%%%%%%%%%%%%%%%%%%%%%%%%%%%%%%%%%%%%%%%%%%%%%%%%%%%%%%
\Section{Interactive Query Functions and an Option Setting Function}

\>ACEGroupGenerators( <i> ) F
\>ACEGroupGenerators() F

return  the  {\GAP}  group  generators  of  the  <i>th  (or   default)
interactive {\ACE} process. If no generators have been saved  for  the
interactive {\ACE} process, possibly because the process  was  started
via   `ACEStart(0);'   (see~"ACEStart"),   the   {\ACE}   process   is
interrogated, and the equivalent in  {\GAP}  is  saved  and  returned.
Essentially,   `ACEGroupGenerators(<i>)'   interrogates   {\ACE}   and
establishes `ACEData.io[<i>].args.fgens', if  necessary,  and  returns
`ACEData.io[<i>].args.fgens'.  As  a  side-effect,  if  any   of   the
remaining       fields       of       `ACEData.io[<i>].args'        or
`ACEData.io[<i>].acegens' are unset, they  are  also  set.  Note  that
{\GAP} provides  `GroupWithGenerators'  (see~"ref:GroupWithGenerators"
in the {\GAP} Reference Manual) to establish a free group on  a  given
set of already-defined generators.

\>ACERelators( <i> ) F
\>ACERelators() F

return the {\GAP} relators  of  the  <i>th  (or  default)  interactive
{\ACE} process. If no relators have been  saved  for  the  interactive
{\ACE}  process,  possibly  because  the  process  was   started   via
`ACEStart(0);' (see~"ACEStart"), the {\ACE} process  is  interrogated,
the  equivalent  in  {\GAP}  is  saved  and   returned.   Essentially,
`ACERelators(<i>)'     interrogates     {\ACE}     and     establishes
`ACEData.io[<i>].args.rels',     if     necessary,     and     returns
`ACEData.io[<i>].args.rels'. As a side-effect, if any of the remaining
fields  of  `ACEData.io[<i>].args'  or  `ACEData.io[<i>].acegens'  are
unset, they are also set.

\>ACESubgroupGenerators( <i> ) F
\>ACESubgroupGenerators() F

return the {\GAP}  subgroup  generators  of  the  <i>th  (or  default)
interactive {\ACE} process. If no subgroup generators have been  saved
for the interactive {\ACE} process, possibly because the  process  was
started via `ACEStart(0);' (see~"ACEStart"),  the  {\ACE}  process  is
interrogated,  the  equivalent  in  {\GAP}  is  saved  and   returned.
Essentially,  `ACESubgroupGenerators(<i>)'  interrogates  {\ACE}   and
establishes `ACEData.io[<i>].args.sgens', if  necessary,  and  returns
`ACEData.io[<i>].args.sgens'.  As  a  side-effect,  if  any   of   the
remaining       fields       of       `ACEData.io[<i>].args'        or
`ACEData.io[<i>].acegens' are unset, they are also set.

\>DisplayACEArgs( <i> ) F
\>DisplayACEArgs() F

display the arguments (i.e.~<fgens>, <rels> and <sgens>) of the  <i>th
(or   default)   process   started    by    `ACEStart'.    In    fact,
`DisplayACEArgs(<i>)'   is    just    a    pretty-printer    of    the
`ACEData.io[<i>].args' record. Use `GetACEArgs'  (see~"GetACEOptions")
in          assignments.          Unlike          `ACEGroupGenerators'
(see~"ACEGroupGenerators"),  `ACERelators'   (see~"ACERelators")   and
`ACESubgroupGenerators'                 (see~"ACESubgroupGenerators"),
`DisplayACEArgs' does not have the side-effect of setting any  of  the
fields of `ACEData.io[<i>].args' if they are unset.

\>GetACEArgs( <i> ) F
\>GetACEArgs() F

return a record of the current  arguments  (i.e.~<fgens>,  <rels>  and
<sgens>) of the <i>th (or default) process started by  `ACEStart'.  In
fact, `GetACEOptions(<i>)' simply returns  the  `ACEData.io[<i>].args'
record,  or  an  empty  record  if  that  record  is  unbound.  Unlike
`ACEGroupGenerators'     (see~"ACEGroupGenerators"),     `ACERelators'
(see~"ACERelators")            and             `ACESubgroupGenerators'
(see~"ACESubgroupGenerators"),  `GetACEOptions'  does  not  have   the
side-effect of setting any of the fields of `ACEData.io[<i>].args'  if
they are unset.

\>DisplayACEOptions( <i> ) F
\>DisplayACEOptions() F

display the options, explicitly set by the user, of the  <i>th  (or  default)  process  started  by
`ACEStart'. In fact, `DisplayACEOptions(<i>)' is just a pretty-printer
of   the   `ACEData.io[<i>].options'   record.   Use   `GetACEOptions'
(see~"GetACEOptions") in assignments. Please note that no-value {\ACE}
options  will   appear   with   the   assigned   value   `true'   (see
Section~"Interpretation of ACE Options" for how the  {\ACE}  interface
functions interpret such options). 

*Notes:* 
Any  options  set  via  `ACEWrite'  (see~"ACEWrite")  will  *not*   be
displayed. Also, recall that if {\ACE} is not  given  any  options  it
uses the `default' strategy  (see  Section~"What  happens  if  no  ACE
Strategy Option or if no ACE  Option  is  passed").  To  discover  the
various    settings    of     the     {\ACE}     Parameter     Options
(see~"ACEParameterOptions") in  vogue  for  the  {\ACE}  process,  use
`ACEParameters' (see~"ACEParameters").

\>GetACEOptions( <i> ) F
\>GetACEOptions() F

return  a  record  of  the  current  options  (those  that  have  been
explicitly set by the user) of the <i>th (or default) process  started
by `ACEStart'. Please note that no-value {\ACE}  options  will  appear
with the assigned value `true'  (see  Section~"Interpretation  of  ACE
Options"  for  how  the  {\ACE}  interface  functions  interpret  such
options).    The     notes     applying     to     `DisplayACEOptions'
(see~"DisplayACEOptions") also apply here.

\>SetACEOptions( <i> [:<options>] ) F
\>SetACEOptions( [:<options>] ) F

modify the current options of the <i>th (or default)  process  started
by `ACEStart'. Please ensure that the `OptionsStack' is  empty  before
calling `SetACEOptions', otherwise the options already present on  the
`OptionsStack' will also be ``seen''. All interactive {\ACE} interface
functions that accept options, actually call an  internal  version  of
`SetACEOptions';  so,  it  is  generally   important   to   keep   the
`OptionsStack' clear while working with {\ACE} interactively.

After setting the options passed, the first  mode  of  the  following:
`ACEContinue'  (see~"ACEContinue"),   `ACERedo'   (see~"ACERedo")   or
`ACEStart' (see~"ACEStart"), that may  be  applied,  is  automatically
invoked.

Since  a  user  will   sometimes have options in the form  of a record 
(e.g.~via `GetACEOptions'),   we    provide  an  alternative  to   the
behind-the-colon  syntax,  in a  manner .like  `PushOptions', for  the
passing of options via `SetACEOptions':

\>SetACEOptions( <i>, <optionsRec> )!{record version} F
\>SetACEOptions( <optionsRec> )!{record version} F

In this form, the record <optionsRec> is used to  update  the  current
options of the <i>th (or default) process started by `ACEStart'.  Note
that since <optionsRec> is a record each field must have  an  assigned
value; in particular, no-value {\ACE} options should be  assigned  the
value `true' (see Section~"Interpretation  of  ACE  Options").  Please
don't mix these two forms of `SetACEOptions'  with  the  previous  two
forms; i.e.~do *not* pass both a record argument  and  options,  since
this will lead to options appearing in the wrong order; if you want to
do this, make two separate calls to  `SetACEOptions',  e.g.~let's  say
you have a process like that started by:

\beginexample
gap> ACEExample("A5", ACEStart);

\endexample

then the following demonstrates both usages of `SetACEOptions':

\beginexample
gap> SetACEOptions( rec(echo := 2) );
gap> SetACEOptions( : hlt);

\endexample

Each of the three commands above generates output; for brevity it  has
not been included.

*Notes:*

\atindex{break-loop}{@\noexpand`break'-loop}
When    `ACECosetTableFromGensAndRels'    enters    a     `break'-loop
(see~"ACECosetTable"), local versions of the second form  of  each  of
`DisplayACEOptions' and `SetACEOptions' become available. (Even though
the names are similar and their function is analogous they are in fact
different functions.)

\>ACEParameters( <i> ) F
\>ACEParameters() F

return a record of the current values of the {\ACE} Parameter  Options
(see~"ACEParameterOptions") of the <i>th (or default) process  started
by `ACEStart', according to {\ACE}. Please note that some options  may
be reported with incorrect values if they have been  changed  recently
without following up with one of the modes `ACEContinue', `ACERedo' or
`ACEStart'. Together the commands `ACEGroupGenerators', `ACERelators',
`ACESubgroupGenerators' and `ACEParameters' give the equivalent {\GAP}
information that is obtained in {\ACE} with  `sr  :=  1'  (see~"option
sr"), which is the ``Run Parameters'' block obtained in the  messaging
output (observable when the `InfoLevel' of  `InfoACE'  is  set  to  at
least 3), when `messages' (see~"option messages") is  set  a  non-zero
value.

*Notes:*
One use for this function might be to determine the  options  required
to replicate a previous run,  but  be  sure  that,  if  this  is  your
purpose, any recent change in the parameter  option  values  has  been
followed by an invocation of one of `ACEContinue' (see~"ACEContinue"),
`ACERedo' (see~"ACERedo") or `ACEStart' (see~"ACEStart").

As a side-effect, for  {\ACE}  process  <i>,  any  of  the  fields  of
`ACEData.io[<i>].args' or `ACEData.io[<i>].acegens'  that  are  unset,
are set.

\>IsCompleteACECosetTable( <i> ) F
\>IsCompleteACECosetTable() F

return, for the <i>th (or  default)  process  started  by  `ACEStart',
`true' if  {\ACE}'s  current  coset  table  is  complete,  or  `false'
otherwise.

*Note:*
The completeness of the coset table of the  <i>th  interactive  {\ACE}
process      is       determined       by       checking       whether
`ACEData.io[<i>].stats.index' is positive; a value of  zero  indicates
the   last   enumeration    failed    to    complete.    The    record
`ACEData.io[<i>].stats'  is  what  is  returned   by   `ACEStats(<i>)'
(see~"ACEStats").

\>ACEDisplayCosetTable( <i> ) F
\>ACEDisplayCosetTable() F
\>ACEDisplayCosetTable( <i>, [<val>] ) F
\>ACEDisplayCosetTable( [<val>] ) F
\>ACEDisplayCosetTable( <i>, [<val>, <last>] ) F
\>ACEDisplayCosetTable( [<val>, <last>] ) F
\>ACEDisplayCosetTable( <i>, [<val>, <last>, <by>] ) F
\>ACEDisplayCosetTable( [<val>, <last>, <by>] ) F

compact and display the (possibly incomplete) coset table of the <i>th
(or default) process started by `ACEStart'; <val> must be an  integer,
and <last> and <by> must be positive integers. In the first two  forms
of the command, the entire coset table is displayed, without orders or
coset representatives. In the third and  fourth  forms,  the  absolute
value of <val> is taken to be  the  last  line  of  the  table  to  be
displayed (and 1 is taken to be the first); in  the  fifth  and  sixth
forms, `|<val>|' is taken to be the first line  of  the  table  to  be
displayed, and <last> is taken to be the number of the last line to be
displayed. In the last two forms, the table  is  displayed  from  line
`|<val>|' to line <last> in steps of <by>. If <val> is negative,  then
the  orders   modulo   the   subgroup   (if   available)   and   coset
representatives are displayed also.

*Note:*
The coset table displayed will normally only be `lenlex'  standardised
if   the   call   to    `ACEDisplayCosetTable'    is    preceded    by
`ACEStandardCosetNumbering'   (see~"ACEStandardCosetNumbering").   The
options `lenlex' (see~"option lenlex") and  `semilenlex'  (see~"option
semilenlex")     are     only     executed     by      `ACECosetTable'
(see~"ACECosetTable"). The {\ACE} binary does not provide `semilenlex'
standardisation, and hence `ACEDisplayCosetTable' will never display a
`semilenlex' standard coset table.

\>ACECosetRepresentative( <i>, <n> ) F
\>ACECosetRepresentative( <n> ) F

return, for the <i>th (or default) process started by `ACEStart',  the
coset representative of coset <n> of the current coset table  held  by
{\ACE}, where <n> must be a positive integer.

\>ACECosetRepresentatives( <i> ) F
\>ACECosetRepresentatives() F

return, for the <i>th (or default) process started by `ACEStart',  the
list of coset representatives of  the  current  coset  table  held  by
{\ACE}.

\>ACETransversal( <i> ) F
\>ACETransversal() F

return, for the <i>th (or default) process started by `ACEStart',  the
list of coset representatives of  the  current  coset  table  held  by
{\ACE}, if the  current  table  is  complete,  and  `fail'  otherwise.
Essentially, `ACETransversal(<i>) = ACECosetRepresentatives(<i>)'  for
a complete table.

\>ACECycles( <i> ) F
\>ACECycles() F
\>ACEPermutationRepresentation( <i> ) F
\>ACEPermutationRepresentation() F

return, for the <i>th (or default) process started  by  `ACEStart',  a
list of permutations corresponding to the group generators, (i.e., the
permutation representation), if the current coset table held by {\ACE}
is complete or `fail', otherwise. In the event of failure a message is
emitted to `Info' at `InfoACE' or `InfoWarning' level 1.

\>ACETraceWord( <i>, <n>, <word> ) F
\>ACETraceWord( <n>, <word> ) F

for the <i>th (or  default)  interactive  {\ACE}  process  started  by
`ACEStart', trace <word> through {\ACE}'s  coset  table,  starting  at
coset <n>, and return the final coset number if the  trace  completes,
and `fail' otherwise. In Group  Theory  terms,  if  the  cosets  of  a
subgroup $H$ in a group $G$ are  the  subject  of  interactive  {\ACE}
process <i> and the coset identified by that process  by  the  integer
<n> corresponds to some coset $Hx$, for some $x$ in  $G$,  and  <word>
represents the element $g$ of $G$, then, providing the  current  coset
table is complete enough, `ACETraceWord( <i>, <n>, <word>  )'  returns
the integer identifying the coset $Hxg$.

*Notes:*
You may wish to compact {\ACE}'s coset table first, either  explicitly
via `ACERecover' (see~"ACERecover"), or, implicitly, via any  function
call that invokes {\ACE}'s compaction routine (see~"ACERecover" note).

If you actually wanted {\ACE}'s  coset  representative,  then,  for  a
*compact*   table,   feed   the   output    of    `ACETraceWord'    to
`ACECosetRepresentative' (see~"ACECosetRepresentative").

\>ACEOrders( <i> ) F
\>ACEOrders() F
\>ACEOrders( <i> : suborder := <suborder> ) F
\>ACEOrders(: suborder := <suborder> ) F

for the <i>th (or  default)  interactive  {\ACE}  process  started  by
`ACEStart', search for all coset numbers whose representatives' orders
(modulo the subgroup) are either  finite,  or,  if  invoked  with  the
`suborder' option,  are  multiples  of  <suborder>,  where  <suborder>
should be a positive integer. `ACEOrders' returns a  (possibly  empty)
list of records, each with fields `coset', `order'  and  `rep',  which
are respectively, the coset number, its order modulo the subgroup, and
a representative for each coset number satisfying the criteria of  the
search.

*Note:*
You may wish to compact {\ACE}'s coset table first, either  explicitly
via `ACERecover' (see~"ACERecover"), or, implicitly, via any  function
call that invokes {\ACE}'s compaction routine (see~"ACERecover" note).

\>ACEOrder( <i>, <suborder> ) F
\>ACEOrder( <suborder> ) F

for the <i>th (or  default)  interactive  {\ACE}  process  started  by
`ACEStart', search for coset  number(s)  whose  coset  representatives
have  order  modulo  the  subgroup  a  multiple  of  <suborder>.  When
<suborder> is a positive integer, `ACEOrder' returns just  one  record
with fields `coset', `order' and `rep', which  are  respectively,  the
coset number, its order modulo the subgroup, and a representative  for
the first coset number satisfying  the  criteria  of  the  search,  or
`fail' if there is no such coset number. The value of  <suborder>  may
also be a negative integer, in which case, `ACEOrder( <i>,  <suborder>
)' is equivalent to `ACEOrders( <i> : suborder :=  |<suborder>|)';  or
<suborder> may be zero,  in  which  case,  `ACEOrder(  <i>,  0  )'  is
equivalent to `ACEOrders( <i> )'.

*Note:*
You may wish to compact {\ACE}'s coset table first, either  explicitly
via `ACERecover' (see~"ACERecover"), or, implicitly, via any  function
call that invokes {\ACE}'s compaction routine (see~"ACERecover" note).

\>ACECosetOrderFromRepresentative( <i>, <cosetrep> ) F
\>ACECosetOrderFromRepresentative( <cosetrep> ) F

for the <i>th (or default) interactive {\ACE} process return the order
(modulo the subgroup) of the coset with representative  <cosetrep>,  a
word in the free group generators.

*Note:*   
`ACECosetOrderFromRepresentative' calls  `ACETraceWord'  to  determine
the coset (number) to which <cosetrep> belongs,  and  then  scans  the
output of `ACEOrders' to determine the order of the coset (number).

\>ACECosetsThatNormaliseSubgroup( <i>, <n> ) F
\>ACECosetsThatNormaliseSubgroup( <n> ) F

for the <i>th (or  default)  interactive  {\ACE}  process  started  by
`ACEStart', determine non-trivial  (i.e.~other  than  coset  1)  coset
numbers whose representatives normalise the subgroup.

\beginlist%unordered

\item{--} If <n> $> 0$, the list of the first  <n>  non-trivial  coset
numbers whose representatives normalise the subgroup is returned.

\item{--} If <n> $\< 0$, a list of records  with  fields  `coset'  and
`rep'  which  represent  the  coset  number  and   a   representative,
respectively,  of  the  first  <n>  non-trivial  coset  numbers  whose
representatives normalise the subgroup is returned.

\item{--} If <n> $= 0$, a list of  records  with  fields  `coset'  and
`rep'  which  represent  the  coset  number  and   a   representative,
respectively, of all non-trivial coset numbers  whose  representatives
normalise the subgroup is returned.

\endlist

*Note:*
You may wish to compact {\ACE}'s coset table first, either  explicitly
via `ACERecover' (see~"ACERecover"), or, implicitly, via any  function
call that invokes {\ACE}'s compaction routine (see~"ACERecover" note).

\>ACEStyle( <i> ) F
\>ACEStyle() F

returns the current enumeration style as one of  the  strings:  `"C"',
`"Cr"', `"CR"', `"R"', `"R*"', `"Rc"', `"R/C"', or `"R/C (defaulted)"'
(see Section~"Enumeration Style").

The next two functions of this section are really intended for  {\ACE}
standalone gurus. To fully understand their output you  will  need  to
consult the standalone manual and the C source code.

%%%%%%%%%%%%%%%%%%%%%%%%%%%%%%%%%%%%%%%%%%%%%%%%%%%%%%%%%%%%%%%%%%%%%%
\Section{Interactive Versions of Non-interactive ACE Functions}

\>ACECosetTable( <i> [:<options>] )!{interactive} F
\>ACECosetTable( [:<options>] )!{interactive} F

return a coset table  as  a  {\GAP}  object,  in  standard  form  (for
{\GAP}).   These   functions   perform   the    same    function    as
`ACECosetTableFromGensAndRels' and `ACECosetTable' on three  arguments
(see~"ACECosetTable"), albeit interactively, on the <i>th (or default)
process started by `ACEStart'. If options are passed then an  internal
version of `ACEModes' is run to determine which of the general  {\ACE}
modes (see Section~"General ACE Modes")  `ACEContinue',  `ACERedo'  or
`ACEStart' is possible; and (an internal version of) the first mode of
these that is allowed is executed, to ensure the  resultant  table  is
correct for the current options.

\>ACEStats( <i> [:<options>] )!{interactive} F
\>ACEStats( [:<options>] )!{interactive} F

perform  the  same  function  as   `ACEStats'   on   three   arguments
(see~"ACEStats" --- non-interactive version), albeit interactively, on
the <i>th (or default) process started by `ACEStart'. If  options  are
passed then an internal version of  `ACEModes'  is  run  to  determine
which of the general {\ACE} modes (see  Section~"General  ACE  Modes")
`ACEContinue', `ACERedo' or `ACEStart' is possible; and  (an  internal
version of) the first mode of these that is allowed  is  executed,  to
ensure the resultant statistics are correct for the current options.

See Section~"Example of Using ACE Interactively (Using ACEStart)"  for
an example demonstrating both these functions  within  an  interactive
process.

\>IsACEGeneratorsInPreferredOrder( <i> )!{interactive} F
\>IsACEGeneratorsInPreferredOrder()!{interactive} F

for the <i>th (or  default)  interactive  {\ACE}  process  started  by
`ACEStart', return `true' if the group generators of that process, are
in an order that will not be changed by {\ACE}, and `false' otherwise.
This function has greatest relevance to users who call `ACECosetTable'
(see~"ACECosetTable!interactive"),   with    the    `lenlex'    option
(see~"option lenlex"). For more details, see the  discussion  for  the
non-interactive    version    of     `IsACEGeneratorsInPreferredOrder'
("IsACEGeneratorsInPreferredOrder"),  which   is   called   with   two
arguments.

%%%%%%%%%%%%%%%%%%%%%%%%%%%%%%%%%%%%%%%%%%%%%%%%%%%%%%%%%%%%%%%%%%%%%%
\Section{Steering ACE Interactively}

\index{dead coset (number)}
\>ACERecover( <i> ) F
\>ACERecover() F

invoke the compaction routine on the coset  table  of  the  <i>th  (or
default) interactive {\ACE} process started by `ACEStart', in order to
recover the space used by the dead coset numbers. A `CO' message  line
is printed if any rows of the coset table were recovered, and  a  `co'
line if  none  were.  (See  Appendix~"The  Meanings  of  ACE's  output
messages" for the meanings of these messages.)

*Note:*
The  compaction  routine  is  called   automatically   when   any   of
`ACEDisplayCosetTable'                   (see~"ACEDisplayCosetTable"),
`ACECosetRepresentative'               (see~"ACECosetRepresentative"),
`ACECosetRepresentatives'             (see~"ACECosetRepresentatives"),
`ACETransversal'          (see~"ACETransversal"),          `ACECycles'
(see~"ACECycles"),                         `ACEStandardCosetNumbering'
(see~"ACEStandardCosetNumbering"),                     `ACECosetTable'
(see~"ACECosetTable")    or    `ACEConjugatesForSubgroupNormalClosure'
(see~"ACEConjugatesForSubgroupNormalClosure"), is invoked.

\atindex{lenlex standardisation scheme}%
{@\noexpand`lenlex' standardisation scheme}
\>ACEStandardCosetNumbering( <i> ) F
\>ACEStandardCosetNumbering() F

compact and then do a  `lenlex'  standardisation  (see  Section~"Coset
Table Standardisation Schemes") of the  numbering  of  cosets  in  the
coset table of the  <i>th  (or  default)  interactive  {\ACE}  process
started by `ACEStart'. That is, for a given ordering of the generators
in the columns of the table, they produce a  canonic  table.  A  table
that  includes  a  column  for  each  generator  inverse   immediately
following the column for  the  corresponding  generator,  standardised
according to the `lenlex' scheme, has the property  that  a  row-major
scan (i.e.~a scan of the successive rows of the *body*  of  the  table
row by row, from left to right) encounters previously unseen cosets in
numeric order. This function does  not  display  the  new  table;  use
`ACEDisplayCosetTable' (see~"ACEDisplayCosetTable") for that.

*Notes:*
In a `lenlex' canonic table, the  coset  representatives  are  ordered
first according to length and then the lexicographic order defined  by
the order the generators and their inverses  head  the  columns.  Note
that,  unless  special  action  is  taken,  {\ACE}  avoids  having  an
involutory generator in the first column (by swapping  the  first  two
generators), except when there is only  one  generator,  or  when  the
second generator is also an involution;  so  the  lexicographic  order
used by {\ACE} need not necessarily correspond with the order in which
the generators were first put  to  {\ACE}.  (We  have  used  the  term
``involution'' above; what we really mean is  a  generator  `x'  for
which there is a relator `x*x' or `x^2'.  Such  a  generator  may,  of
course,  turn  out  to  actually  be  the  identity.)   The   function
`IsACEGeneratorsInPreferredOrder'                                 (see
"IsACEGeneratorsInPreferredOrder") detects  cases  when  {\ACE}  would
swap the first two generators.

Standardising the coset numbering within {\ACE} does *not* affect  the
{\GAP} coset table obtained via `ACECosetTable' (see~"ACECosetTable").
If `ACECosetTable' is called  without  the  `lenlex'  option  {\GAP}'s
default  standardisation  is  applied  after  conversion  of  {\ACE}'s
output, which undoes an {\ACE} standardisation. On the other hand,  if
`ACECosetTable' is called with the `lenlex' option then after a  check
and  special  action,  if  required,  the  equivalent  of  a  call  to
`ACEStandardCosetNumbering' is invoked, irrespective of whether it has
been done by the user beforehand. The check that is done is a call  to
`IsACEGeneratorsInPreferredOrder'                                 (see
"IsACEGeneratorsInPreferredOrder")  to  ensure  that  {\ACE}  has  not
swapped the first two generators. The special action  taken  when  the
call to  `IsACEGeneratorsInPreferredOrder'  returns  `false',  is  the
setting of  the  `asis'  option  (see~"option  asis")  to  1  and  the
resubmission of the relators to {\ACE} taking care not to  submit  the
relator that determines the first generator as an involution  as  that
generator squared (these two actions together avert {\ACE}'s  swapping
of the first two generators), followed by  the  re-`start'ing  of  the
enumeration.

*Guru Notes:*
In  five  of  the  ten  standard  enumeration   strategies   of   Sims
\cite{Sim94} (i.e.~the five Sims strategies not provided  by  {\ACE}),
the   table   is   standardised   repeatedly.   This   is    expensive
computationally, but can result in fewer cosets being  necessary.  The
effect of doing this can be investigated  in  {\ACE}  by  (repeatedly)
halting the enumeration (via restrictive options),  standardising  the
coset numbering, and continuing (see Section~"Emulating Sims"  for  an
example).

\>ACEAddRelators( <i>, <wordlist> ) F
\>ACEAddRelators( <wordlist> ) F

add, for the <i>th (or default) interactive {\ACE} process started  by
`ACEStart', the words in the list <wordlist> to any  relators  already
present, and automatically invoke `ACERedo', if it can be applied,  or
otherwise `ACEStart'. Note that {\ACE}  sorts  the  resultant  relator
list, unless the `asis' option (see~"option asis") has been set to  1;
don't assume, unless `asis = 1',  that  the  new  relators  have  been
appended in user-provided order to  the  previously  existing  relator
list.  `ACEAddRelators'  also  returns  the  new  relator  list.   Use
`ACERelators' (see~"ACERelators") to  determine  the  current  relator
list.

\>ACEAddSubgroupGenerators( <i>, <wordlist> ) F
\>ACEAddSubgroupGenerators( <wordlist> ) F

add, for the <i>th (or default) interactive {\ACE} process started  by
`ACEStart',  the  words  in  the  list  <wordlist>  to  any   subgroup
generators already present, and automatically invoke `ACERedo', if  it
can be applied, or otherwise `ACEStart'. Note that  {\ACE}  sorts  the
resultant  subgroup  generator  list,   unless   the   `asis'   option
(see~"option asis") has been set to 1; don't assume,  unless  `asis  =
1',  that  the  new  subgroup  generators  have   been   appended   in
user-provided order to  the  previously  existing  subgroup  generator
list.  `ACEAddSubgroupGenerators'  also  returns  the   new   subgroup
generator          list.          Use          `ACESubgroupGenerators'
(see~"ACESubgroupGenerators")  to  determine  the   current   subgroup
generator list.

\>ACEDeleteRelators( <i>, <list> ) F
\>ACEDeleteRelators( <list> ) F

for the <i>th (or  default)  interactive  {\ACE}  process  started  by
`ACEStart', delete <list> from the current relators, if list is a list
of words in the group generators, or those current relators indexed by
the integers in <list>, if <list> is a list of positive integers,  and
automatically invoke `ACEStart'. `ACEDeleteRelators' also returns  the
new relator list. Use `ACERelators' (see~"ACERelators")  to  determine
the current relator list.

\>ACEDeleteSubgroupGenerators( <i>, <list> ) F
\>ACEDeleteSubgroupGenerators( <list> ) F

for the <i>th (or  default)  interactive  {\ACE}  process  started  by
`ACEStart', delete <list> from the  current  subgroup  generators,  if
list is a list of words in the  group  generators,  or  those  current
subgroup generators indexed by the integers in <list>, if <list> is  a
list  of  positive  integers,  and  automatically  invoke  `ACEStart'.
`ACEDeleteSubgroupGenerators' also returns the new subgroup  generator
list.  Use  `ACESubgroupGenerators'  (see~"ACESubgroupGenerators")  to
determine the current subgroup generator list.

\>ACECosetCoincidence( <i>, <n> ) F
\>ACECosetCoincidence( <n> ) F

for the <i>th (or  default)  interactive  {\ACE}  process  started  by
`ACEStart', return the representative of coset <n>, where <n> must  be
a positive integer, and add  it  to  the  subgroup  generators;  i.e.,
equates  this  coset  with  coset  1,  the  subgroup.   `ACERedo'   is
automatically invoked.

\>ACERandomCoincidences( <i>, <subindex> ) F
\>ACERandomCoincidences( <subindex> ) F
\>ACERandomCoincidences( <i>, [<subindex>] ) F
\>ACERandomCoincidences( [<subindex>] ) F
\>ACERandomCoincidences( <i>, [<subindex>, <attempts>] ) F
\>ACERandomCoincidences( [<subindex>, <attempts>] ) F

for the <i>th (or  default)  interactive  {\ACE}  process  started  by
`ACEStart', attempt up to <attempts> (or, in the first four forms,  8)
times to find nontrivial subgroups with index a multiple of <subindex>
by repeatedly making random coset numbers coincident with coset 1  and
seeing what happens. The starting coset table must be  non-empty,  but
must  *not*  be   complete   (use   `ACERandomlyApplyCosetCoincidence'
(see~"ACERandomlyApplyCosetCoincidence")  if  your  table  is  already
complete).  For  each  attempt,  by  applying  {\ACE}'s  `rc'   option
(see~"option rc") random coset representatives are repeatedly added to
the subgroup and the enumeration `redo'ne. If the  table  becomes  too
small, the  attempt  is  aborted,  the  original  subgroup  generators
restored, and another attempt made. If an attempt succeeds,  then  the
new set of subgroup generators  is  retained.  `ACERandomCoincidences'
returns   the   list   of   new   subgroup   generators   added.   Use
`ACESubgroupGenerators' (see~"ACESubgroupGenerators") to determine the
current subgroup generator list.

*Notes:* 
`ACERandomCoincidences' may add subgroup generators even if it  failed
to  determine  a  nontrivial  subgroup  with  index  a   multiple   of
<subindex>; in such a case, the original status  may  be  restored  by
applying                                 `ACEDeleteSubgroupGenerators'
(see~"ACEDeleteSubgroupGenerators")  with   the   list   returned   by
`ACERandomCoincidences'.

`ACERandomCoincidences' applies the `rc' option (see~"option  rc")  of
{\ACE} which takes  the  line  that  if  an  enumeration  has  already
obtained a finite index then either, <subindex> is already  a  divisor
of that finite index, or the request is impossible. Thus an invocation
of `ACERandomCoincidences', in the  case  where  the  coset  table  is
already complete, is an error.

*Guru  Notes:*  A  coset  can  have  many  different  representatives.
Consider              running              `ACEStandardCosetNumbering'
(see~"ACEStandardCosetNumbering") before  `ACERandomCoincidences',  to
canonicise the table and the representatives.

\>ACERandomlyApplyCosetCoincidence( <i> [: <controlOptions>]) F
\>ACERandomlyApplyCosetCoincidence( [: <controlOptions>]) F

for the <i>th (or  default)  interactive  {\ACE}  process  started  by
`ACEStart', try to find a larger proper subgroup (i.e.~a  subgroup  of
smaller   but    nontrivial    index),    by    repeatedly    applying
`ACECosetCoincidence'  (see~"ACECosetCoincidence")  and  seeing   what
happens;  `ACERandomlyApplyCosetCoincidence'  returns  the   (possibly
empty) list of new subgroup generators added. The starting coset table
must    already    be    complete     (use     `ACERandomCoincidences'
(see~"ACERandomCoincidences") if your table is not already  complete).
`ACERandomlyApplyCosetCoincidence' provides the following four options
(<controlOptions>).

\beginitems

\quad`subindex := <subindex>' & 
Sets the restriction that the final index  should  be  a  multiple  of
<subindex>; <subindex> must be  a  positive  integer  divisor  of  the
initial subgroup index.

\quad`hibound := <hibound>' &
Sets the restriction that the final index should  be  (strictly)  less
than <hibound>; <hibound> must be an integer that is  greater  than  1
and at most the initial subgroup index.

\quad`lobound := <lobound>' &
Sets the restriction that the final index should be (strictly) greater
than <lobound>; <lobound> must be an integer that is at  least  1  and
(strictly) less than the initial subgroup index.

\quad`attempts := <attempts>' &
Sets  the   restriction   that   the   number   of   applications   of
`ACECosetCoincidence' should be at most <attempts>; <attempts> must be
a positive integer.

\enditems

By default, `<subindex> =  1',  <hibound>  is  the  existing  subgroup
index, `<lobound> = 1' and `<attempts> = 8'. If after an  attempt  the
new index is a multiple of <subindex>, less than <hibound> and greater
than <lobound> then the  process  terminates  (and  the  list  of  new
subgroup generators is returned). Otherwise, if an attempt  reaches  a
stage where the criteria cannot be satisfied, the attempt is  aborted,
the original subgroup generators restored, and another  attempt  made.
If  no  attempt  is  successful  an  empty  list  is   returned.   Use
`ACESubgroupGenerators' (see~"ACESubgroupGenerators") to determine the
current subgroup generator list.

\>ACEConjugatesForSubgroupNormalClosure( <i> ) F
\>ACEConjugatesForSubgroupNormalClosure() F
\>ACEConjugatesForSubgroupNormalClosure( <i> : add ) F
\>ACEConjugatesForSubgroupNormalClosure(: add ) F

for the <i>th (or  default)  interactive  {\ACE}  process  started  by
`ACEStart', test each conjugate of a subgroup  generator  by  a  group
generator for membership in the subgroup,  and  return  the  (possibly
empty) list of conjugates that were determined to not  belong  to  the
subgroup (coset 1); and,  if  called  with  the  `add'  option,  these
conjugates are also added to the existing list of subgroup generators.

*Notes:* A conjugate of a subgroup generator is tested for  membership
of the subgroup, by checking whether it can be traced from coset 1  to
coset 1  (see  `ACETraceWord':~"ACETraceWord").  For  an  *incomplete*
coset  table,  such  a  trace  may  not  complete,   in   which   case
`ACEConjugatesForSubgroupNormalClosure' may return an empty list  even
though the subgroup is *not* normally closed within the group.

The `add' option does *not* guarantee that the resultant  subgroup  is
normally closed. It is still possible  that  some  conjugates  of  the
newly added subgroup generators will not be elements of the subgroup.

*Example:*
To      demonstrate      the      usage      and      features      of
`ACEConjugatesForSubgroupNormalClosure', let us  consider  an  example
where we know pretty well what to expect.

Let $G$ be the group, isomorphic to the symmetric  group  $S_6$,  with
the presentation
$$
\{ a, b \mid a^2, b^6, (ab^{-1}ab)^3, (ab^{-1}ab^2)^4, (ab)^5, 
             (ab^{-2}ab^2)^2 \}
$$
(from  \cite{CM72}),  and  let  $H$  be  the  subgroup  $\langle  ab^3
\rangle$. There is an isomorphism $\phi$ from $G$ to $S_6$ mapping $a$
onto $(1,2)$ and $ab^3$ onto $(1,2,3,4,5,6)$. It follows  that  $\phi$
maps $ab^3$ into $A_6$. So we know that the normal closure of $H$  has
index 2 in $G$. Let us observe this via {\ACE}.

First we start an enumeration with `max' set to 80.

\beginexample
gap> F := FreeGroup( "a", "b" );;
gap> a := F.1;; b := F.2;;
gap> fgens := GeneratorsOfGroup( F );;
gap> rels := [ a^2, b^6, (a*b^-1*a*b)^3, (a*b^-1*a*b^2)^4, (a*b)^5,
>              (a*b^-2*a*b^2)^2 ];;
gap> sgens := [ a*b^3 ];;
gap> i := ACEStart( fgens, rels, sgens : max := 80 );;
gap> IsCompleteACECosetTable( i );
false
gap> ACEConjugatesForSubgroupNormalClosure( i : add );
#I  ACEConjugatesForSubgroupNormalClosure: All (traceable) conjugates in subgp
[  ]

\endexample

Though we know that $H$ is not equal to its normal closure, we did not
get any new elements (we had warned  above  of  such  a  possibility).
Apparently our incomplete table is too small. So let us increase `max'
to 100 and continue.

\beginexample
gap> ACEContinue( i : max := 100 );;
gap> IsCompleteACECosetTable( i );
false
gap> ACEConjugatesForSubgroupNormalClosure( i : add );
[ b^-1*a*b^4 ]
gap> IsCompleteACECosetTable( i );
true
gap> ACEStats( i ).index;
20

\endexample

This time we got a new element, and after adding it  to  the  subgroup
generators  we  obtained  a  complete  table.  However  the  resulting
subgroup need not yet be the normal closure of $H$  (and  in  fact  we
know that it is not). So we continue with another call to the function
`ACEConjugatesForSubgroupNormalClosure'.

\beginexample
gap> ACEConjugatesForSubgroupNormalClosure( i : add );
[ b^-2*a*b^5, b*a*b^2 ]
gap> ACEStats( i ).index;
2

\endexample

Now we have the index that we expected. Another call to  the  function
`ACEConjugatesForSubgroupNormalClosure'  should  not  yield  any  more
conjugates. We ensure that this is indeed the case  and  then  display
the resulting list of subgroup generators.

\beginexample
gap> ACEConjugatesForSubgroupNormalClosure( i : add );
#I  ACEConjugatesForSubgroupNormalClosure: All (traceable) conjugates in subgp
[  ]
gap> ACESubgroupGenerators( i );
[ a*b^3, b*a*b^2, b^-1*a*b^4, b^-2*a*b^5 ]

\endexample

%%%%%%%%%%%%%%%%%%%%%%%%%%%%%%%%%%%%%%%%%%%%%%%%%%%%%%%%%%%%%%%%%%%%%%
\Section{Primitive ACE Read/Write Functions}

For those familiar with the  workings  of  the  {\ACE}  standalone  we
provide primitive read/write tools to  communicate  directly  with  an
interactive {\ACE} process,  started  via  `ACEStart'  (possibly  with
argument 0, but this is not essential). For the most part, it is up to
the user to translate the output  strings  from  {\ACE}  into  a  form
useful in {\GAP}. However, after the group generators,  relators,  and
subgroup  generators  have  been  set  in  the  {\ACE}  process,   via
`ACEWrite',    invocations    of    any    of     `ACEGroupGenerators'
(see~"ACEGroupGenerators"),     `ACERelators'     (see~"ACERelators"),
`ACESubgroupGenerators'       (see~"ACESubgroupGenerators"),        or
`ACEParameters' (see~"ACEParameters") will establish the corresponding
{\GAP} values.  Be  warned  though,  that  unless  one  of  the  modes
`ACEStart'  (without  a  zero  argument;  see~"ACEStart"),   `ACERedo'
(see~"ACERedo")  or  `ACEContinue'   (see~"ACEContinue"),   or   their
equivalent  for  the  standalone   {\ACE}   (`start;',   `redo;',   or
`continue;'), has been invoked since the last change of any  parameter
options (see~"ACEParameterOptions"), some of the  values  reported  by
`ACEParameters' may well be *incorrect*.

\>ACEWrite( <i>, <string> ) F
\>ACEWrite( <string> ) F

write <string> to the <i>th or  default  interactive  {\ACE}  process;
<string> must be in exactly the form the  {\ACE}  standalone  expects.
The command is  echoed  via  `Info'  at  `InfoACE'  level  4  (with  a
```ToACE> ''' prompt); i.e.~`SetInfoLevel(InfoACE, 4);'  will  allow  you to
see what is transmitted to the {\ACE} binary. `ACEWrite' returns `true'
if successful in writing to the stream of the interactive {\ACE} process,
and `fail' otherwise.

*Note:*
If `ACEWrite' returns `fail' (which means that the {\ACE} process  has
died), you may like to try resurrecting the interactive {\ACE} process
via `ACEResurrectProcess' (see~"ACEResurrectProcess").

\>ACERead( <i> ) F
\>ACERead() F

read a complete line of {\ACE}  output,  from  the  <i>th  or  default
interactive {\ACE} process, if there is output to be read and  returns
`fail' otherwise. When successful, the line is returned  as  a  string
complete with trailing newline  character.  Please  note  that  it  is
possible to be ``too quick'' (i.e.~the return  can  be  `fail'  purely
because the output from {\ACE} is not there  yet),  but  if  `ACERead'
finds any output at all, it waits for a complete line.

\>ACEReadAll( <i> ) F
\>ACEReadAll() F

read and return as many *complete* lines of {\ACE}  output,  from  the
<i>th or default interactive {\ACE} process, as there are to be  read,
*at the time of the call*, as a list  of  strings  with  the  trailing
newlines removed and returns the empty  list  otherwise.  `ACEReadAll'
also writes each line read via `Info' at `InfoACE' level  3.  Whenever
`ACEReadAll' finds only a partial line,  it  waits  for  the  complete
line, thus increasing the probability that it  has  captured  all  the
output to be had from {\ACE}.

\>ACEReadUntil( <i>, <IsMyLine> ) F
\>ACEReadUntil( <IsMyLine> ) F
\>ACEReadUntil( <i>, <IsMyLine>, <Modify> ) F
\>ACEReadUntil( <IsMyLine>, <Modify> ) F

read complete lines of  {\ACE}  output,  from  the  <i>th  or  default
interactive {\ACE} process, ``chomps'' them (i.e.~removes any trailing
newline character), emits them to `Info' at  `InfoACE'  level  3,  and
applies the function <Modify> (where <Modify>  is  just  the  identity
map/function for the first two forms) until a ``chomped'' line  <line>
for which `<IsMyLine>( <Modify>(<line>)  )'  is  true.  `ACEReadUntil'
returns the list of <Modify>-ed ``chomped'' lines read.

*Notes:* 
When provided by the user, <Modify> should be a function that  accepts
a single string argument.

<IsMyLine> should be a function that is able to accept the  output  of
<Modify> (or take a  single  string  argument  when  <Modify>  is  not
provided) and should return a boolean.

If `<IsMyLine>( <Modify>(<line>) )' is never true, `ACEReadUntil' will
wait indefinitely.

%%%%%%%%%%%%%%%%%%%%%%%%%%%%%%%%%%%%%%%%%%%%%%%%%%%%%%%%%%%%%%%%%%%%%%%%%
%%
%E
